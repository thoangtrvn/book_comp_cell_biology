\chapter{Gene Editing}


\section{CRISPR-Cas9}
\label{sec:CRISPR-Cas9}


CRISPR-Cas9 is an experimental gene editing technique used to make {\it precise
changes} in DNA. Other researchers are refining CRISPR-Cas9 to be more
efficient, specific, and safe.

This experimental technique is not ready to try in humans, but it has moved
quickly from test tubes to living cells to organisms.

To fix genes on a microscopic scale, one cell at a time, the faulty code has to
be located and physically cut - and that's what CRISPR-Cas9 does.
This cutting requires two components: (1) a guide RNA and (2) a cutting enzyme
called Cas9. The guide RNA finds and presents the right spot on the DNA, and
the Cas9 acts as the scissors, actually cutting the DNA.

CRISPR-Cas9 can operate in either: {\bf delete} mode, and {\bf editing} mode.



\begin{enumerate}
  \item   CRISPR-Cas9 can be used to edit the HD gene in the brain of a living
  mouse
  
  The pecific guide RNAs will show Cas9 where to cut twice, on both sides
  of the extra long stretch of C-A-G repeats in the HD gene. Then the new ends can
  be patched together, permanently removing the offending part.   
  
  The guide RNA and Cas9 'scissors' are carried by specially designed viruses
  that must be injected into the brain. Li's group applied this technique to the
  striatum. A few weeks later, the CRISPR-cas9 components had spread to many
  cells, disabling the dysfunctional HD gene, and signs of stress on the neurons
  had diminished.
  
  
  
\end{enumerate}