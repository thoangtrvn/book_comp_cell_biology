\chapter{What to do next?}


\section{RyR}

\begin{enumerate}
  \item produce the nonlinear relationship between fractional SR $\Ca$
  increase and SR $\Ca$ load \citep{bassani1995fsr, shannon2000pfs}.
  
  \item RyR-CSQ interaction: \citep{Lee2008, tania2010}
\end{enumerate}


\section{Floderus equation (1944)}

This equation is used to estimate the number of cells (or other cell's
component) from thin slices images.
The count correlation factor
\begin{equation}
\frac{(t-h)}{t+D-2*h}
\end{equation}
with $t=$ section thickness; $D=$ vesicle diameter (in detecting
vesicle in axonal terminal), and $h = $ the percentage of radius lost as caps
(estimated at 10\%, or 2nm).

However, this method has certain assumptions and it may not be accurate for some
cases. 
Why was the Floderus method of estimating Leydig cell number applicable to
control rats but not to the TE-implanted rats? 
\url{http://www.ncbi.nlm.nih.gov/pubmed/1447753}
