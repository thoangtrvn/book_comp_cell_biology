\chapter{Plasma $\Ca$-ATPase (PMCA) models}
\label{chap:PMCA}

% \section{PMCA pump}
% \label{sec:pmca-pump}


The plasma membrane $\ca$-ATPase pump (PMCA) is ATP-driven pump to expell $\ca$
out of the cell. It was discovered in 1966 by Schatzmann (Sect.\ref{sec:PMCA}).
For many eukaryotic cells, PMCA is the only mechanism to export $\Ca$ out of the cell. In some cell
types (e.g. neurons and cardiac cells), PMCA is the secondary mechanism, after
Na/Ca exchanger (NCX) (Chap.\ref{chap:NCX}).
\begin{itemize}
  \item NCX is a low $\Ca$ affinity, yet with high capacity
  
  \item PMCA is high $\Ca$ affinity ($\Km = 0.2-1.0 \muM$), yet with low
  capacity (30Hz, or pumping 30 ions/sec per PMCA molecule)
  
The energy cost is 1ATP for 1$\Ca$ transport; though some data showed 2$\Ca$
transport. 
\end{itemize}



% \begin{framed}
%   PMCA pump is inhibited by \ce{La^3+} and vanadate. 
% \end{framed}

\section{Introduction}


Most model of the PMCA use Michaelis-Menten formula of the rate or current
(Sect.\ref{sec:Michaelis-Menten-modified}) due to the small contribution of
PMCA compared to NCX.



\section{Simple model}


\subsection{Constant clearance (Traub, Llinas - 1997)}
\label{sec:PMCA-Traub-Llinas-1997}
\label{sec:PMCA-constant-clearance}

The ATP-driven PMCA is modeled using (Traub, LLinas, 1997)

\def\equil{{\text{equil}}}
\def\msec{{\text{msec}}}
\begin{equation}
\frac{d[\Ca]}{dt} = \frac{[\Ca]_\equil - [\Ca]_i}{\tau}
\end{equation}
with $\tau=0.9-2.0$ ([msec]).

\subsection{Vm-dependency clearance (Zador, Koch, Brown - 1990)}
\label{sec:PMCA-Vm-dependent-clearance-rate}

The ATP-driven PMCA is modeled using (Zador, Koch, Brown, 1990)
\def\equil{{\text{equil}}}
\def\msec{{\text{msec}}}
\begin{equation}
\frac{d[\Ca]}{dt} = \frac{[\Ca]_\equil - [\Ca]_i}{\tau(\Vm)}
\end{equation}
with
\begin{equation}
\tau(\Vm) = 17.7 \exp (\Vm/35) \qquad ([\msec])
\end{equation}

The appropriate ODE
\begin{equation}
\frac{[\Ca]_{n,t+\Delta t} - [\Ca]_{n,t}}{\Delta t} = 
\frac{[\Ca]_\equil}{\tau(\Vm)} - 
\frac{[\Ca]_{n,t+\Delta t} + [\Ca]_{n,t}}{2.\tau(\Vm)}
\end{equation}

These membrane-bound pump proteins have to move Ca2+ against an extremely large
calcium gradient (five orders of magnitude).

\begin{equation}
I_\pmca = 
\end{equation}

\section{Michaelis-Menten formula of PMCA}

Under optimal conditions, Michaelis constant $K_m$ of the pump for $\Ca$ drops
from 10-30$\mu$M range (at rest) to 0.2-0.5$\mu$M range.
\textcolor{red}{It also operates as a $\ca$/H$^+$ exchanger, but the
stoichiometry is neutral, i.e. Ca:H=1:2}.

\subsection{Jafri et al. (1998)}
\label{sec:winslow-et-al-1}

~\citep{jafri1998cad,winslow1999} modeled the sarcolemmal $\Ca$-ATPase pump
using the Michaelis-Menten formula
\begin{equation}
  \label{eq:1250}
  I_{p(\ca)} =
  \overline{I_{p(\ca)}}\frac{([\Ca]_i)^\eta}{(K_{m,p(\ca)})^\eta + ([\Ca]_i)^\eta}
\end{equation}
with $K_{m,p(\ca)}=0.5$ ($\muM$) is half-saturation constant, and Hill
coefficient $\eta = 1$, and $\overline{I_{p(\ca)}}=0.05\mu$A/cm$^2$.

As the pump takes 2 $\H$ inside for each $\Ca$ ion pumping out, there is no net
charge change in PMCA for this model.

\subsection{Greenstein-et-al-2002}

Later,~\citep{greenstein2002} modified it with $\eta =2$ with the fact that
every 2 $\Ca$ ions are extruded.
As the pump takes 2 $\H$ inside for each $\Ca$ ion pumping out, so the net
charge change in PMCA for this model is also $I_{p(\ca)}$.
\begin{equation}
  \label{eq:1250}
  I_{p(\ca)} =
  \overline{I_{p(\ca)}}\frac{([\Ca]_i)^\eta}{(K_{m,p(\ca)})^\eta + ([\Ca]_i)^\eta}
\end{equation}
with $\eta=2$.


% \subsection{Inhibitors}
% \label{sec:inhibitors}

% We created a first order kinetics model of the PMCA
% that mimics its activation and recovery as function of [Ca2+] i (Vmax increased
% from 30 to 60 nM s
% 
% 1 , Km decreased from 500 to 400 nM when fully activated) and incor- porated
% into a mathematical description of Ca2+ signaling in T cells (Lympho- LAB).
% Our results indicate that modulation of the PMCA activity improves the stability
% of Ca2+ signaling by adjusting the pump rate to Ca2+ influx even at high [Ca
%2+ ] i levels (preventing a harmful Ca2+ overload). Moreover the delay in
% modulation permits small Ca2+ fluxes to increase transiently enhancing Ca2+
% signaling dynamic

\section{Caride et al. (2001)}


\section{Graupner et al. (2005)}
