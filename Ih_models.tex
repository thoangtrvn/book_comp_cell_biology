\chapter{Ih models}
\label{chap:Ih-models}
\label{chap:HCN-models}

\def\fast{{{\text{fast}}}}
\def\slow{{{\text{slow}}}}
\def\open{{{\text{open}}}}


\section{Huguenard-McCormick (1992) - guinea pig dorsal lateral geniculate relay
neurons}
\label{sec:Ih-Huguenard-McCormick-1992}


Mathematical equation, using Hodgkin-Huxley-type formula, was fitted using data
collected from voltage-clamp of guinea pig dorsal lateral geniculate relay
neurons maintained {\it in vitro} as a thalamocortical slice (McCormick and
Pape, 1990).
\begin{itemize}
  \item Voltage-clamp: from holding potential -65mV to different stepping
  voltage (-55mV to -100 mV) for 5 seconds.

At the holding potential -65mV, a small amount of Ih is active, which deactivate
when stepping to -55 mV.

The activation current
\begin{equation}
\frac{I}{I_\max} = \frac{1}{1 + \exp\left[ \frac{\Vm - V_{1/2}}{k} \right]}^N
\end{equation}
with $N$ = a power factor that influences the amplitude time-course of the
current (e.g. reflects the delay of activation in voltage-clamp) (if there is
no delay, we set $N=1$); $V_{1/2}$ = the voltage at which channels are half
activated/inactivated; and $k$ = the slope factor of the sigmoidal curve
(related to the valence of charge that must be displaced to switch between
closed and permissive states).

The power factor N may be thought of as the
number of (in)activation gates, each of which must be in the permissive
state for the channel to be conductive.

  \item The tail current is generated when stepping back to holding potential:
  generated by Ih deactivation
\end{itemize}

The Ih current is modeled with a single activation gate $m$, and no inactivation
is found, i.e. $h=1$.

\begin{equation}
\tau_m = \frac{1}{\exp(-0.086 \Vm - 14.6) + \exp(0.0701 \Vm - 1.87)}
\end{equation}



\subsection{Schweighoer et al. (1999)}
\label{sec:Ih-Schweighofer-1999}

Schweighoer et al. (1999 - Sect.\ref{sec:Schweighofer-1999} adopted the formula
developed by Huguenard-McCormick (1992 -
Sect.\ref{sec:Ih-Huguenard-McCormick-1992}) and modified the time constant
$\tau_m$

\begin{equation}
\tau_m = \frac{1}{\exp(-0.086 \Vm - 14.6) + \exp(0.07 \Vm - 1.87)}
\end{equation}



\section{Kole-Hallermann-Stuart (2006) - single channel: layer 5 pyramidal
neuron (HCN1)}
\label{sec:Ih-Kole2006}

The authors developed a stochastic model of Ih single-channel gating into a
morphologically realistic model of a layer 5 neuron. This model is for HCN1
channel (Sect.\ref{sec:HCN-channels-HCN1}).

The conductance density of Ih (i.e. gh ) was distributed across compartments
using the following exponential function
\begin{verbatim}
gh = y0 + A * exp(d/lambda)

y0 = -2 pS/um^2
A  = 4.28   [pS/um^2]
lambda = 323   [um]
d  = distance from soma  [um]
\end{verbatim}
to gave rise to a range of Ih densities between 2.3 pS/$\mum^2$ at the soma to
93 pS/$\mum^2$ in the distal apical dendrites about 1000 $\mum$ from the soma.
This enables the change in Vm that was found experimentally.

Stochastic simulation for single channel Ih was done with NEURON-extension
NMODL (Sect.\ref{sec:NMODL}). The number of channels in each segment, was
calculated from the current density and the surface area.

\begin{mdframed}
The time-step $dt$ was set to 10 or 100$\mu$s for simulations using stochastic
Ih or 1 $\mu$s for simulations using stochastic Na+ channels.
\end{mdframed}

A simple 2-state model using Hodgkin-Huxley formula
\begin{equation}
\ce{C <=>[\alpha][\beta] O}
\end{equation}
with single activation gate
$m$ with single exponential (i.e. power = 1) and no inactivation gate (h=1).
\begin{equation}
\begin{split}
P_\open &= \int_{t=0}^{dt} \alpha(\Vm) \exp(-\alpha(\Vm)) dt \\
        &= 1 - \exp( -\alpha(\Vm)dt)
\end{split}
\end{equation}
with $\alpha(\Vm)$ is the rate of channel opening, as a function of the local
transmembrane potential $\Vm$.

\begin{mdframed}
The time course of Ih current onset at -100 mV hyperpolarizing step was best fit
with a double-exponential function
\begin{itemize}
  \item $\tau_\fast = 27.0\pm 1.5$ ms

  \item $\tau_\slow = 155.0 \pm 8.0$ ms
\end{itemize}
The fast component dominate the with the peak amplitude current ratio fast:slow
is 4.9$\pm$0.3 times.

\end{mdframed}


\begin{equation}
\begin{split}
\alpha(\Vm) &= \frac{A(\Vm + B)}{\exp\left(  \frac{\Vm + B}{C}\right) - 1} \\
\beta(\Vm) &= D \exp(\frac{\Vm}{E})
\end{split}
\end{equation}
NOTE: $m_\infty = \frac{\alpha_m}{\alpha_m + \beta_m}$;
$\tau_m = \frac{1}{\alpha_m + \beta_m}$.
%and the closing rate $\beta(\Vm) = D \exp(\frac{\Vm}{E})$.



\begin{mdframed}

The 5 free parameters were fit by simultaneously fitting
\begin{itemize}
  \item $\tau_m = \frac{1}{\alpha(\Vm) + \beta(\Vm)}$ to the voltage-dependent
time-constant of $I_h$ activation and deactivation; and

  \item the fraction of opening channels
  $\frac{\alpha(\Vm)}{\alpha(\Vm) + \beta(\Vm)}$ to the activation curve.
\end{itemize}
using Levenberg-Marquardt method for sum of squared errors.

The weighting factors:


RESULT: A = 6.43 $s^{-1}$, B = 154 mV, C = 11.9 mV, D = 193 $s^{-1}$,
and E  =  33.1 mV.

\end{mdframed}



The channel noise increases with channel expression at distal site (i.e. tufted
dendrite) and the noise increased threefold with a 10-fold increased
single-channel conductance (6.8 pS) but constant Ih current density.

They suggested that in the face of high Ih current densities, the small
single-channel conductance of Ih is critical for maintaining the fidelity of
action potential output.


