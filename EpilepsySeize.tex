\chapter{Epilepsy (i.e. seizure disorder) vs. Seizure}
\label{sec:epilepsy_seizure_detection}

It is important to note the distinction between seizures and "epilepsy" (often
called a "seizure disorder"). \textcolor{blue}{A seizure is an event}, i.e. a
brief disturbance in the electrical activity of the brain (whose disturbance can
be detected using EEG - Sect.\ref{sec:EEG}) that causes temporary changes in
movement, awareness, feelings, behavior, or other bodily functions. Under
certain conditions, epilepsy is likely a cause of seizures, and such seizure
event is called {\bf ictal epileptic activity}.

\section{Epilepsy}
\label{sec:epilepsy}

The term epilepsy implies an abnormally high tendency to have seizures
(Sect.\ref{sec:seizure}). Epilepsy can be the result of a variety of {\it
neurological conditions} characterized by {\it recurrent unprovoked seizures}.

\begin{itemize}
  
  \item  Epilepsy is "an occasional, an excessive and a disorderly discharge of
  nervous tissue" induced by any process involving the cerebral cortex that
  pathologically increases the likelihood of depolarization and synchronized
  firing of groups of neurons (John Hughlings Jackson, 1889).
  
Many potential underlying causes: such as metabolic disorders of nerve cells or
virtually any disorder that damages cortical tissue including trauma,
  hemorrhage, ischemia, anoxia, infection, hyperthermia, or the presence of scar
  tissue relating to prior injury.
  
  
\end{itemize}



Even still, not all areas of the cerebral cortex have the same tendency to
epileptic activity: most of the neocortex is relatively resistant, while the temporal lobes and frontal lobes (particularly the limbic areas) are highly susceptible.


\subsection{Prevalance} 

It is the 4-th most common neurological disorder in US, after migraine, stroke,
and Alzheimer's disease. Anyone can develop epilepsy at any time. Incidence is
highest among the very young and the very old. About 1\% of American have some
form of epilepsy (Sect.\ref{sec:epilepsy_classification}).  10\% of Americans
will have at least one seizure at some point in their lives.
 
Epilepsy is found link to other disease. Epilepsy is prevalent among other
disability groups such as autism (25.5\%), cerebral palsy (13\%), Down syndrome
(13.6\%), and intellectual disability (25.5\%). For people with both cerebral
palsy and intellectual disability the prevalence of epilepsy is 40\%.

\subsection{Symptoms}

In about 70 percent of epilepsy cases, there is no known cause.

Among the remaining 30 percent, the following causes are most frequent:
traumatic brain injury, brain tumor, stroke, Alzheimer's disease, poisoning
(e.g. lead poisoning, alcohol or drug abuse, etc.), infection (e.g. meningitis,
encephalitis, and others), prenatal or birth trauma, and developmental or
congenital disabilities. Genetic factors also play a role in some types of
epilepsy, but we still have a great deal to learn about this.
 



\subsection{Epileptiform in hippocampus}

In roden hippocampal CA3 region, the powerful inter-pyramidal synaptic
connections predispose the cell population to epileptic burst and
after-discharges (Miles and Wong, 1987).

By design, to restrain this epileptiform tendency, there should be a system of
recurrent inhibition. As such, reliable activation of inhibitory pathways is
essential for maintaining the balance between excitation and inhibition during
cortical activity.


Interneuron, generate disynaptic IPSP in the projected neuron characterizes a
connection between two neurons as involving an intermediate neuron.
 

\section{Seizure}
\label{sec:seizure}

A seizure is an event which can be detected using EEG (Sect.\ref{sec:EEG}) or
MRI (Sect.\ref{sec:MRI}), i.e. a brief disturbance in the electrical activity of
the brain that causes temporary changes in movement, awareness, feelings,
behavior, or other bodily functions. Under certain conditions, epilepsy is
likely a cause of seizures, in that the seizure is called {\bf epileptic
seizure} or {\bf ictal epileptic activity} (Sect.\ref{sec:epileptic-seizure});
however, not all seizures are due to epilepsy (Sect.\ref{sec:epilepsy}), e.g.
due to diabetes.

All human cerebral cortices have the potential to generate seizures given enough
of a stimulus; with in nature, nearly 10\% of people will have seizure at some
point in life.


There are several types of seizures; with two main classes 
\begin{itemize}
  \item primary generalized seizures: the seizure involves all of the cerebral
  cortex simultaneously.
  
  \item focal onset (localization-related) seizures:  it involves a localized
  cluster of neurons having epileptiform activity.
\end{itemize}
While most seizures present with motor correlates, some can present with mainly
inhibitory phenomena.

There are more than 20 types of seizures.

\subsection{ictal epileptic activity (ictal epileptic event (IEE), epileptic
seizure)}
\label{sec:epileptic-seizure}
\label{sec:epileptic-seizures}

In the evolution of the cerebral cortex, the sophisticated organization in a steady state far away
from thermodynamic equilibrium has produced the side effect of two fundamental pathological
network events: ictal epileptic activity and spreading depolarization
(Sect.\ref{sec:cortical-spreading-depression}).

{\bf Epileptic seizure} is the seizure as the result of epilepsy
(Sect.\ref{sec:epilepsy}). The fact that epileptic seizures always start in the
brain is important when considering the EEG. Ictal epileptic activity describes
the partial disruption.


\subsection{drug for seizure}
\label{sec:seizure-drug}

Most common anticonvulsants affect Na+ channels, calcium channels, and/or the
GABAA-benzodiazepine receptor complex.


FBM (Sect.\ref{sec:FBM}) has showed its effectiveness in the treatment of
partial onset seizures.
\begin{enumerate}
  
  \item FBM has an effect on $\Na$ currents. However, FBM's primary
  antiepileptic effect appears to be at the NMDA receptor channel complex, where
  FBM displaces 5,7-[3H]-dichlorokynurenate from the strychnine-insensitive
  glycine binding site (McCabe et al., 1993).
  
  This effect has been demonstrated also in human postmortem brains (Wamsley et
  al., 1994).

FBM may have a potentiation of GABAA responses by FBM (Rho et al., 1994);
however a clear action on GABA receptors is controversial (Ticku et al., 1991; 
Domenici et al., 1994). Furthermore, an action ofFBM on non-
NMDAglutamate receptors has been described (De Sarro et al.,
1994; Domenici et al., 1994).

  \item  In rodents, FBM inhibits seizures induced by maximal electroshock, by
picrotoxin, pentylentetrazol and 4-aminopyridine, but not by bicuculline or
strychnine (Swinyard et al., 1986; for review see Burdette \& Sackellares,
1994).

  \item In Rhesus monkeys, FBM reduced seizures produced by aluminium hydroxide
injections into cortical regions (Lockard et al., 1987).
  
  \item FBM antagonizes seizures induced by N-methyl-D-aspartate (NMDA) and
  kainate-induced status epilepticus (White et al., 1992; Chronopoulos et al.,
  1993). 
  
  
\end{enumerate}
\section{Classification: Epilepsy}
\label{sec:epilepsy_classification}

It's important for patients to ask their neurologists what type of seizures they
are experiencing and what type of epilepsy is suspected. 

There are more than 40 types of epilepsy.


