\chapter{Chronic Pain}

Neuropathic pain is a major disabilities that affect millions of people.
Chronic pain is a long-lasting form of neuropathic pain, i.e. lasting for
months, years or even a life-time.

Chronic pain is a major disability faced by more than 100 million in U.S. 
The state of the art treatment is non-specific and is not effective.
The result, in part, due to the lack of the fundamental understanding of the
mechanism of chronic pain's etiology in the CNS (Sect.\ref{chap:CNS}).

Experimental studies have suggested:
\begin{enumerate}
  \item an increase in the synaptic efficacy leads to a reduction in pain
  threshold in the spinal cord and the brain
  
  This suggests a key role of synaptic plasticity in the development of chronic
  pain.
  
\end{enumerate}


Pain is modeled as the change in peak frequency (range 16-189\%) and the change
in the total number of spikes elicited (range: 24-372\%), similar to what
observed in dorsal horn neurons (Sect.\ref{sec:dorsal-horn-neurons}) after the
persion is exposured with mustard oil.

\section{Neuropathic pain}
\label{sec:neuropathic-pain}

Peripheral nerve injury and tissue inflammation often induce a state of abnormal
pain known as neuropathic pain, which includes hyperalgesia and allodynia.

Chronic constriction injury (CCI) of the sciatic nerve increases Met-enkephalin
immunoreactivity in the spinal cord
(Sect.\ref{sec:enkephalin-containing-neurons}). Increases in enkephalin have
also been described in spinal cord injury, polyarthritis, electrical
stimulation, and various other preparations/
Thus, the 
