\chapter{Writing a Simulator}


Simulation is the imitation of the operation of a real-world process or system
over time. This include
\begin{itemize}
  \item building a mathematical model for the system, i.e. in the forms of ODE
  and/or PDE and/or algebraic equations.
  
  The models have time-evolving parameters
  that represent key characteristics of different parts of the physical system.
   
  \item writing a code, choose the right numerical methods to solve such
  ODE/PDE.
  
  The several packages/libraries have been developed which make writing
  simulating code easier. The packages started as a domain-specific packages;
  and then become generic, which can be applied to different fields.
  
  These packages support: Monte Carlo simulation, stochastic modeling,
  multimethod modeling.
  
  \item run the simulation, giving an initial condition, i.e. initial values for
  all parameters
   
   \item (optional) a visualization toolkit showing the dynamical changes in the
   system.
\end{itemize}

There are different levels of simulation:
\url{https://en.wikipedia.org/wiki/Simulation}


\section{Types of simulators}

\subsection{Discrete-event simulators}
\label{sec:discrete-event-simulator}

A discrete-event simulatior (DES) models the operation of a system as a
discrete sequence of events in time.
\begin{itemize}
  \item  Each event occurs at a particular instant in time and marks a change of
  state in the system
  
  \item Between two consecutive events, no change in the system is assumed. So
  if it know the time for the next event, it can directly jump in time from one
  event to the next. This is in contrast to continuous simulation
  (Sect.\ref{sec:continuous-simulator}) where you need to keep track the
  system dynamics over time.
  
  Because discrete-event simulations do not have to simulate every time slice,
  they can typically run much faster than the corresponding continuous simulation.
\end{itemize}

The system is modeled via a number of components:
\begin{enumerate}
  \item {\bf States}: a set of variables whose values captures the different
  states of the system
  
  The number of these variables in {\bf States} is called the dimension of the
  system.
  
  \item {\bf Clock}:
  to keep track the current simulation time (in whatever measurement units
  suitable for the system, e.g second, hours, month\ldots)
   
  \item {\bf Event}: an event is described by the tuple: 
  \begin{itemize}
    \item Instantaneous event: (time at which it
  occurs, the code representing the type of event)
  
    \item Interval-based event model: 
    If the event takes some times, we need to
    have a time interval:
  (giving the start time, the end time of each event, the code representing the
  type of event)
  \end{itemize}
  
  To simulate the randomness of the time the event to occur, the simulation
  needs to generate random variables of various kinds, depending on the system
  model. The simulation that has randomness, is called {\bf Monte Carlo
  simulation}.
  
  When studying any system, we need to study its characteristics at
  steady-state. However, one of the problems with the random number
  distributions used in discrete-event simulation is that the steady-state
  distributions of event times may not be known in advance. As a result, the
  initial set of events placed into the pending event set will not have arrival
  times representative of the steady-state distribution. This problem is
  typically solved by {\bf bootstrapping the simulation model}.
  
  Bootstrapping means we temporarily substitute the empirical probability
  distribution induced by the sample for the probability distribution defined by
  the population.
  
% Here, it refers to a technique to bring the distribution of event times of
% system approaches the steady state.
  First, we bootstrap the initial data points by carefully choose an initial
  sets of pending events with realistic times. Then, these events, upon
  simulated by the program, will schedule additional events, and with time, the
  distribution of event times approaches the steady state.
  \footnote{\url{https://en.wikipedia.org/wiki/Bootstrapping\#Discrete_event_simulation}}
  
  When we analyze the result, at each simulation, we ignore (disregard) the
  events that occur before the steady state is reached.
  
  \item {\bf Activity}:
  
  NOTE: You can have the concept of {\bf Activity} which is a sequence of
  events. 
  

  \item {\bf Current event}: 
  Depending on the type of programs
  \begin{itemize}
    \item {\it single-thread program}: only one current event
    
    \item {\it multi-threaded program} or {\bf program supporting an
    interval-based event model}: it may have multiple current events
    
    In both cases, there are significant problems with synchronization between
    current events.
  \end{itemize}
  
    
  \item {\bf Event list}:
  A list of simulation events {\bf Event} which is a list of {\bf pending
  events}
  that need to be simulated. 
  
  This is often programmed as a {\bf  priority queue, sorted by event time}.
  So, regardless of the order in which events are added to the event set, they
  are removed in strictly chronological order. \textcolor{red}{The popular
  general-purpose priority queue algorithms}: splay tree, skip lists, calendar
  queue, ladder queues.
  
\end{enumerate}
IN SUMMARY: A common exercise in learning how to build discrete-event
simulations is to model a queue. The random variables that need to be
characterized to model this system stochastically is the arrival time of the
event.

A common goal is to study the {\bf Statistics} of the system via the model, e.g.
analyzing the variance or the mean of the system's output at steady-state by
running the simulation several times, each time uses a different bootstraps,
and at steady-state each time the data point(s) are collected
\begin{verbatim}
(at steady-state)
run-1    data1(1)   data2(1) ...
run-2    data1(2)   data2(2) ..
......
.....
Statistics: mean(data1), mean(data2)
           variance(data1), ...
           ...
\end{verbatim}

Example the number of bootstraps required, i.e. the number of simulation to get
a good statistics as given in
\url{http://www.researchgate.net/post/How_many_bootstraps_can_I_do_from_a_sample_of_N_elements}
\begin{enumerate}
  
  \item How many [unique] bootstrap samples can be formed from a sample of N
  elements [where each of the N elements is distinct]?
  
  
  \item for highly skewed data, none of the bootstrap methods work well with
  less than about 30 samples
  
  \item to get confidence intervals for the mean, 100 bootstrap samples would be
  sufficent, 
  
  \item for a confidence interval of the standard deviation you need more,
  like 500. 
  
  \item But for a statistic like the cost-effectiveness ratio I use 50.000 since
  the underlying distribution of the statistic is Cauchy (because the
  denominator can be zero, the function contains a discontinuity).
  
  \item If one is interested in the complete distribution of the statistic, not
  in its confidence intervals only, a large number of samples is needed to get a
  proper and precise impression (This occurs if one wants to determine
  outlyers).
  
\end{enumerate}
\url{http://www.amstat.org/publications/jse/v13n3/wood.html}

\subsection{Continuous simulators}
\label{sec:continuous-simulator}

A {\bf continuous-event simulator} is called an activitity-based simulator;
It continuously tracks the system dynamics over time; and the time is broken 
up into small time slices and the system state is updated according to the 
different activities happening in the time slice.

\subsection{Process-oriented discrete-event simulators}

We also have {\bf process-based simulator}, i.e.
each event in a system corresponds to a separate process, where a process is
typically simulated by a thread in the simulation program.

In this case, the discrete events, which are generated by threads, would cause
other threads to sleep, wake, and update the system state.

\subsection{}

\section{Simulation frameworks}


There are two important criteria before analyzing a simulation model:
\begin{enumerate}
  \item wait until it reaches stead-state
  
  So the simulation designer must decide when the simulation will end. 
  
  
  \item repeat several time, to get the mean behavior at steady-state 
\end{enumerate}

Regardless of the framework we use, the logic of the simulation engine is 
\begin{itemize}
  \item {\bf Start}
  \begin{enumerate}
  \item Initialize ending-condition to FALSE
  
  The ending-condition can be (1) the number of iteration, (2) the maximum time
  to reach, (3) the mean squared-error below a certain threshold, \ldots
  
  
  \item Initialize system state variables (with some initial value that is
  suppose to be closed to the steady-state)
  
  \item Initialize clock (usually starts at simulation time zero)
  
  
  \item Depending on system
  \begin{itemize}
    \item Discrete-event simulation: Schedule an initial event (i.e. put some
    initial event into the Events list)

    \item Continuous simulation: Either using fixed time-step or dynamic
    time-step
    
    
   \end{itemize}
  \end{enumerate}
  
  \item {bf Timing-loop}: Do-loop or while-loop
  \begin{verbatim}
  While (ending condition is FALSE)
  {
    generate next-event time (usig PSEUDO-RANDOM)
    set clock to next-event time
    do next-event and remove the Event-list
    update statistics
  }
  \end{verbatim}
  
  \item {\bf Ending}: generate statistical report
\end{itemize}



\subsection{Agent-based framework}


An {\bf agent-based framework} for performance modeling of an optimistic
parallel discrete event simulator (Sect.\ref{sec:discrete-event-simulator}) is
another example for a discrete event simulation.





\section{C++ SIM: process-based}

C++SIM, a set of  C++ class definitions  that mimic the process-based 
simulation facilities of SIMULA and the SIMSET routines.

The  co-routine facility of Simula   is implemented by various threads
(and others).  Classes  are    provided  for various   random   number
distributions.

\url{https://github.com/codehaus/javasim}

\section{SimPy: process-oriented discrete-event}
\label{sec:SimPy}

{\bf SimPy} is 
a  process-oriented discrete-event simulation.

Processes in SimPy are simple Python generator functions and are used to model
active components like customers, vehicles or agents. 

\section{Markov chain: finite-state machine of ion channel gating}

\url{https://en.wikipedia.org/wiki/Discrete_event_simulation}

\url{https://en.wikipedia.org/wiki/Finite-state_machine}