\chapter{Experiments}

\section{NMR (Nuclear Magnetic Resonance) spectroscopy or MRS (Magnetic Resonance Spectroscopy): measure metabolic fluxes}
\label{sec:MRS}
\label{sec:NMR}

A {\bf metabolic pathway} (within a cell) is a series of chemical reactions that the
cell uses to break down simple molecules such as glucose for fuel, or
alternatively to synthesize more complex biomolecules such as hormones and
neurotransmitters.
These chemical reactions are performed by enzymes produced by the expression of
genes.

The change of certain intermediate product, as represented by its {\bf current
concentration} and {\bf the metabolic flux} (rate of creation/consumption of
such molecule), in one metabolic pathway can be a good indicator of healthy
status of the human subject.

The ability of the body to perform the integrated physiological processes needed
for survival, such as providing the energy needed to sustain brain function,
depends upon the coordination of metabolic pathways in billions of cells
throughout the body.

\begin{mdframed}

Sensing pathways are also metabolic pathways: They detect the presence of
specialized biomolecules such as hormones and neurotransmitters used for
cell-to-cell communication.

At the level of the organism the coordinated response of metabolic path- ways is
critical for maintaining higher-level function such as systemic physiology,
organ function, and behavior. The demands on metabolism are constantly changing
in the face of environmental challenges.

\end{mdframed}


Advances in MRS (Sect.\ref{sec:NMR-machines}), using stable isotopes, have
provided direct noninvasive measurements of metabolic fluxes in the healthy
human brain (Duarte et al., 2012; Shulman and Rothman, 2001).
{\it Flux, or metabolic flux is the rate of turnover of molecules through a
metabolic pathway.}

Using NMR machines, we can understand quantitatively the regulation and control
of metabolic fluxes in humans by in vivo NMR measurements of brain and muscle
and to relate such metabolic understanding to normal and pathological functions.
\begin{itemize}
  \item Robert Shulman (Yale University) \url{https://medicine.yale.edu/mrrc/faculty/robert_shulman-4.profile}
  
The study using MRS started with microorganisms and perfused organs. These
experiments were extended to intact animal models in 1981 (3) and to human
subjects when the Yale School of Medicine established a Magnetic Resonance
Center in 1985. 

Since that time, in vivo $^{13}$C NMR has been applied to study a variety of
metabolic pathways and systems in animals and humans in our laboratory and
several other laboratories world-wide.

The first observations by $^{13}$C NMR of cerebral glutamate were made using
indirect detection at the MR Center at Yale in 1983.


  \item Doughlas Rothman (Yale University)
  
Advanced NMR techniques and equipment improvements now enable in vivo measures
of fluxes in several cubic centimeters of human brain by 1H-$^{13}$C NMR, in
which the carbon isotope is measured indirectly through scalar coupling to bound
proton nuclei.

This flux measurement provides valuable quantitation of metabolic rates in vivo,
not obtainable by any other methods.
  
\end{itemize}


Using stable isotope (e.g. $^{13}$C or $^{1}$H), we can measure  in vivo fluxes within an
organism that are not directly observable.
Isotopic tracers ($^2$H, $^{13}$C, $^{14}$C, $^{15}$N and others) have been used
since the 1930s to reveal metabolically active pathways as well as enzyme
mechanisms in all types of organisms.
\begin{itemize}
  
  \item $^{11}$C is short-lived (half-life is 20.5min)
  
  \item $^{14}$C is long-lived (5,760 years)
  
  \item $^{13}$C is stable in \ce{CO2} environment - Sect.\ref{sec:NMR-13C}
  
  Nowadays, $^{13}$C fluxomics, which also trades under the name $^{13}$C
metabolic flux analysis (MFA), matured as a powerful, but complex systems
biology tool.

  \item $^{1}$H - Sect.\ref{sec:NMR-1H}
  
\end{itemize}

\begin{mdframed}

Even for the most basic biological processes, such as the breakdown of glucose
for fuel, it requires the involvement of well over 1000 genes.

Molecular genetics and structural biology have identified nucleic acids and
proteins involved in an ever-increasing network of biochemical activities, and
details of molecular properties are so plentiful they can only be handled by
computers.

Biochemistry has provided chemical and physical explanations of many properties,
e.g. molecules, structures, genes, and signaling pathways.

Nowadays, modern biochemistry more than ever before sits at the juncture of
these two universes - the more complex organismic properties and the equally
complex but ever more detailed physical understanding at a molecular level.



\end{mdframed}


Metabolic fluxes cannot be directly measured but have to be inferred from other
observables. Metabolic reaction rates (fluxes) contribute fundamentally to our
understanding of metabolic phenotypes and mechanisms of cellular regulation.
All trophic systems (from single cell to global biosphere) depends on
photosynthesis and metabolism of reduced carbon substrates - which are operated
by carbon substrate fluxes. Isotope methodologies are useful tools for tracing
carbon substrate fluxes.

MCA (Sect.\ref{sec:MCA}) is a well-developed theory that defines the control of
flux in a pathway in terms of the concentrations and kinetic properties of
constitutive enzymes and shows how it can be determined by achievable
experimental protocols.


Tracing carbon fluxes means tracking carbon atoms in chemical reactions or
during displacement. 

The isotope effect on pyruvate-dehydrogenase causes a
depletion of $^{13}$C in the metabolites of acetyl-CoA and lipids, i.e. TCA
cycle (Sect.\ref{sec:TCA-pathway}).

\subsection{NMR machines}
\label{sec:NMR-machines}

High resolution NMR spectroscopy is used to follow chemical reactions and brain
activity in vivo and NMR imaging methods enable functional areas of brain
activity to be resolved.

In vivo NMR spectroscopy is similar to the better-known magnetic resonance
imaging (MRI) in that it measures the signal emitted by nuclei within the human
body when placed in a powerful magnetic field.
However, it adds a new dimension to MRI by tracking the resolved NMR signals
from different chemicals in the body, a method well developed in organic
chemistry.

In vivo NMR can measure the concentrations and synthesis rates of individual
biological molecules such as glycogen and neurotransmitters within precisely
defined areas of specific organs such as brain, liver, and muscle.


The 4.0 Tesla spectrometer capable of imaging humans for localized spectroscopic
 studies, similar spectrometers for animal studies at fields of 7T, 9.4T and
 11.74T for studying rats and mice
 

\section{$^{1}$H NMR spectroscopy}
\label{sec:NMR-1H}

The non-invasive quality of this technique makes it suitable not only for
diagnostic purposes but also for therapy monitoring paralleling an eventual
neuroprotection.


\section{[1-$^{13}$C]glucose and [1,2-$^{13}$C]acetate}
\label{sec:NMR-13C}

The use of $^{13}$C NMR to track metabolic pathways was pioneered in the 1970s,
when metabolic substrates such as \textcolor{red}{glucose} were tagged with an
NMR-visible stable isotope, $^{13}$C.

Through the appearance of the $^{13}$C isotope in other compounds, as part of
the metabolism cascades that modify $^{13}$C-tagged glucose, we can know 
the metabolism rate, and the current level of $^{13}$C-tagged glucose.

\begin{enumerate}
  
  \item  Rates of these metabolic pathways were determined by measuring the label appearance as a function of time.
  
  \item In the absence of labeled substrates, the low 1.1\% naturally abundant
  13C NMR signal can be used to measure concentrations, while labeled enriched
  precursors such as 1-$^{13}$C glucose enable fluxes to be measured even when
  concentrations are in steady state.
  
  This kind of turnover measurement has been very valuable in measuring kinetics
  during steady-state concentrations, particularly in vivo, where steady-state
  concentrations are the rule.
  
\end{enumerate}

We use $^{13}$C NMR for measuring concentrations of $^{13}$C-tagged metabolites,
e.g. glucose, and the rates through metabolic pathways in the human body.

Labeled mannitol, which is not taken up by the muscle cell, was used as a marker
to allow the signal from 1-$^{13}$C glucose in the blood and extracellular space
to be distinguished from the glucose within the muscle.

The results demonstrated that intracellular glucose was <200 $\muM$, which would
lead to a negligible back flux through the glucose transporter.




\subsection{-- regulation of muscle glycogen metabolism, tagging glucose}
\label{sec:glycogen}

The regulation of muscle and/or liver glycogen metabolism
(Sect.\ref{sec:glycogen}), by insulin, in health and diabetes.


\textcolor{red}{UP-TO-DATE}:
\begin{itemize}
  
  \item  \textcolor{red}{QUESTION: muscle vs liver glycogen metabolism}: Muscle
  therefore was shown to have the major role in insulin-stimulated glucose
  storage while a meal is being absorbed.
  
  
  \item Between meals the level of glucose in the blood is maintained by release
  of glucose from the liver
  
  Glucose is produced by the liver either by de novo synthesis via
  gluconeogenesis or by breakdown of stored glycogen (glycogenolysis)
  
  Gluconeogenesis accounted for 64\% of glucose production during the first 22 h
  of fast in normal subjects, substantially more than had previously been
  estimated by other methods
  
  Rates of gluconeogenesis in patients with NIDDM accounted for nearly all of
  the elevated rate of fasting liver glucose production.
  
  \item \textcolor{red}{Roles of 3 enzymes (GT, Hk, GSase)}: The activity of the
  muscle glucose transporter (GT) is impaired in diabetes.
  
  The question remained as to whether this was the primary defect in NIDDM or
  was secondary to the loss of glucose homeostasis early in the disease.
  
  The glucose transporter activity may be so reduced in patients with NIDDM that
  it dominates flux control even though it does not exert the majority of flux
  control in healthy subjects.?
  
  
  \item The eversal of the defect through regular exercise in these subjects
  underlines the complex interactions between the environment (lifestyle) and
  genetics in the development of the disease.
  
  Young, healthy, normal-weight offspring of NIDDM parents with normal blood
  glucose levels were studied.
  It was known that these subjects had a high risk of developing the disease.
  In the 13C NMR studies a defect in the muscle glucose transporter and an
  associated reduction in glycogen synthesis similar to that in the NIDDM
  subjects were found, suggesting that impaired glucose transporter activity was
  a primary defect in the pathogenesis of NIDDM
  
  After glycogen-depleting exercise, 13C NMR glycogen concentrations were
  measured during recovery in control subjects.
  The recovery was sharply divided into an initial insulin-insensitive period of
  1–2 h, followed for many hours by an insulin-sensitive time.
  
  Measurements of muscle glycogen repletion after intense depleting exercise in
  insulin-resistant offspring of NIDDM parents showed (a) normal rates of muscle
  glycogen synthesis during the early insulin-independent phase of recovery from
  exercise and (b) severely diminished rates of muscle glycogen synthesis during
  the subsequent insulin-dependent period (2–5 h)
  
  After six weeks of aerobic training G6P levels and rates of muscle glycogen
  synthesis normalized during a hyperglycemic clamp, thus demonstrating that
  this abnormality can be reversed with exercise training
  
  
  
   
  
\end{itemize}


\textcolor{red}{HISTORY}: In 1981, arteriovenous difference studies by DeFronzo
and co-workers in conjunction with a glucose clamp technique suggested that
muscle played the dominant role, than liver in glycogen metabolism.
However, at that time, which metabolic pathway was primarily responsible for
insulin resistance was not known, nor were the identities of the key control
enzymes in these pathways. This is resolved using NMR device with glucose were
tagged with an NMR-visible stable isotope, $^{13}$C.

Later, by enriching the $^{13}$C concentration, starting with an enriched
substrate such as 1-$^{13}$C glucose, the sensitivity and therefore the accuracy
could be increased by more than a factor of ten.


% DeFronzo RA, Jacot E, Jequier E, Maeder 23. E, Wahren J, Felber JP. 1981. The effect
% of insulin on the disposal of intravenous glucose: results from indirect calorimetry
% and hepatic and femoral venous catheteri-
% zation. Diabetes 30:1000–7


\begin{mdframed}

A failure of this regulation leads to non-insulin-dependent diabetes mellitus
(NIDDM), which affects about 15\% of the population over the age of 65 in the United
States. The disease is known to have a strong genetic component because there is
about 90\% concordance in identical twins.

In addition to the genetic contribution, patients respond to environmental
factors of exercise and diet

\end{mdframed}

In a healthy person, after a meal containing glucose, as the result of food
breakdown, glucose is released, i.e. level of glucose spikes (with peak 2hrs
after meal), the pancreas releases insulin that activates glucose metabolism in
muscle and liver.

When the body needs to eliminate excess systemic glucose, such as after a
high-carbohydrate meal, the hormone insulin is released from the pancreas.
Insulin binds cellular receptors, which leads to a cascade of signaling events.
In the final event, phosphoprotein phosphatase cleaves phosphorylated serines
from GSase, which activates it.

In diabetes the ability to maintain the balance between glucose absorption,
metabolism, and insulin secretion is lost, which results in chronically elevated
blood glucose and subsequent cellular damage.

HYPOTHESIS:
\begin{enumerate}
  \item prior to {\it in vivo NMR studies}: it was hypothesized that
  the molecular defect responsible for NIDDM was in the enzyme glycogen synthase.
  
  Among the many reactions of the pathways:
  the key role ascribed to glycogen synthase was based on the belief that
  phosphorylated enzymes in a metabolic pathway control the rate.
  
  
  Glycogen is the major storage compound for glucose in the body, Most glycogen
  is present in muscle and liver. In both organs glycogen synthesis is
  stimulated by insulin and high glucose levels.
  
  
  The loss of capability to reducing glucose was thought as the result of
  lacking the enzyme ({\it glycogen synthase} GSase) that helps to convert
  glucose to glycogen. 
  
  NOTE: The enzyme of opposite function, i.e. converting glycogen to glucose, is
  phosphorylase (GPase). 
  
  
  
  \item current knowledge (based on $^{13}$C NMR in vivo):
  
\begin{mdframed}
  Unknown question: there were also uncertainties at the level of whole-body
  physiology about the relative importance of the muscle and liver glycogen
  synthesis pathways.
  
  Studies measuring arteriovenous differences had been interpreted to support
  muscle as the primary storage point for plasma glucose. However, this
  interpretation was weakened by the difficulty of making accurate arteriovenous
  difference measurements of liver glucose up- take in humans and by the
  inaccuracy of biopsy measurements of muscle glycogen.

Glycogen, with molecular weight of several million Daltons, would, if it were a
rigid molecule, not be visible by high-resolution NMR because its resonance
lines would be broadened beyond detectability.
However, in standard solution, $^{13}$C glycogen NMR were about 100\% visible.
The intensity of the 1.1\% natural abundance $^{13}$C peak enabled the muscle
glycogen concentration to be determined more accurately than by the existing
biopsy method.

Later, by enriching the $^{13}$C concentration, starting with an enriched
substrate such as 1-$^{13}$C glucose, the sensitivity and therefore the accuracy
could be increased by more than a factor of ten.

\end{mdframed}  

EXPERIMENTS: A $^{13}$C-labeled glucose and insulin were infused into both
healthy subjects and NIDDM patients to simulate post-meal conditions.

The flow of $^{13}$C-tagged glucose into muscle glycogen was measured by
following the increase in the intensity of the $^{13}$C NMR glycogen signal.
The twofold slower muscle glycogen synthesis rate in the NIDDM patients was
found to quantitatively account for their lower insulin-stimulated glucose
uptake, thus establishing muscle glycogen synthesis as the major pathway of
insulin resistance in NIDDM


In Shulman et al. (1990), the initial $^{13}$C NMR studies showed that
insulin-stimulated glycogen synthesis in muscle was the major metabolic pathway
for disposing of excess glucose in healthy adults after a meal and that a defect
in muscle glycogen synthesis was largely responsible for the decreased insulin
sensitivity in NIDDM

% Shulman GI, Rothman DL, Jue T, Stein P, DeFronzo RA, Shulman RG. 1990. Quanti-
% tation of muscle glycogen synthesis in nor- mal subjects and subjects with
% non-insulin dependent diabetes mellitus by 13C nuclear magnetic resonance
% spectroscopy. N. Engl.
% J. Med. 322:223–28
  
  Three enzymes in the pathway between plasma glucose and muscle glycogen are
  regulated by insulin: glucose transporters, hexokinase, and glycogen synthase.
  
  The activities of these enzymes in vivo or in cell cultures correlated with
  the rate of insulin-stimulated glycogen synthesis.
  
  \item between meal glucose synthesis
  
  By using 13C NMR to measure concentrations of liver glycogen during a fast and
  measuring the total rates of hepatic glucose output by conventional
  radiolabels, the separate gluconeogenic and glycogenolytic rates were
  determined in human livers
  
  
\end{enumerate}



\subsection{-- relationship between glutamate neurotransmission and metabolism}

The relationship between glutamate neurotransmission and (glucose) metabolism is
important to know if glutamate reuptake consume what amount of energy?.

{\bf BACKGROUND}: The cycle of neurotransmitter fluxes, e.g. glutamate: from
glutamate release, then got converted into glutamine (inside astrocytic 
terminal), and glutamine (once released) got uptake into the presynaptic
terminal, where it is converted back to glutamate.

\begin{enumerate}
  
  \item how neurotransmitter fluxes support the transmission of information,
  i.e. the rate of neurotransmitter release correspond to the rate of neuronal
  information transfer.
  
  \item how this process depends upon brain energy consumption?

Because rates of release and recognition of neurotransmitters regulate brain
activity, such activity should be related to brain energy consumption. But the
question is how much of that energy is for maintaining synaptic activity.
  
  \item the implications of quantitating glutamate neurotransmitter flux 
  for the interpretation of higher-level brain function based on functional imaging studies?
  
  
  As it was found that (in 1998) there is almost a 1:1 stoichiometry between
  brain oxygen oxidation (to generate ATP in Krebs cycle) $V_{\text{TCA}}$, with
  glutamate-glutamine cycle ($V_{\text{cycle}}$), it suggests that most of
  energy increases are for recovering neurotransmitter fluxes (Shen, 1998,
  Magistretti et al (1999)).
  Importantly, the cycling model is equally valid and are similar in rat and
  human brain.
  
   These relative rates of oxidation and cycling were studied over a wide range
   of brain activities in the rat, taking the animals from deep pentobarbital
   anesthesia, with a flat EEG indicating no neuronal firing, through lighter
   states of anesthesia, ending with a lightly anesthetized state where neuronal
   firing had been enhanced by nicotine.
  
  
%   Magistretti PJ, Pellerin L, Rothman DL, Shulman RG. 1999. Perspective: neuro-
%   science “energy on demand.” Science 283:496–97
%   

% Shen J, Sibson NR, Cline G, Behar KL, Rothman DL, et al. 1998. 15N NMR
% spectroscopy studies of ammonia transport and glutamine synthesis in the
% hyperammone- mic rat brain. Dev. Neurosci. 20:438– 43

  
\end{enumerate}

Shulman and Rothman at Yale University focused on the two most abundant
neurotransmitters: GABA and Glutamte. The high concentrations of glutamate and
GABA have enabled them to observe their high-resolution NMR signals in vivo.

It is considered that neurotransmitters a class of metabolites whose
concentrations, reaction pathways, and fluxes can be studied in vivo as one
particularly important aspect of intermediary metabolism.

\begin{enumerate}
  
  \item The first observations by $^{13}$C NMR of cerebral glutamate were made
  using indirect detection at the MR Center at Yale in 1983, and the authors followed
  the flow of $^{13}$C label into pools of metabolites.

NOTE: Glucose is normally the nearly exclusive cerebral carbon source

Enriched 1-$^{13}$C glucose was particularly useful in following pathways and
quantitating fluxes that are the downstream products of pathways involving
glucose.


  \item The best established $^{13}$C NMR measurement in the brain was done in
  1990 (Fitzpatrick, \ldots Shulman (1990)), in that  the flow from 1-$^{13}$C
  glucose to 4-$^{13}$C glutamate, which was used to determine the flux through
  the tricarboxylic acid cycle ($V_{TCA}$ - Sect.\ref{sec:TCA-pathway}).
  
  Although glutamate is not a TCA-cycle intermediate, it has been shown to be in
  fast exchange with $\alpha$-ketoglutarate, which is in the cycle. Because exchange
  between these two pools is rapid in brains compared with the TCA-cycle flux,
  the turnover of large glutamate pools determines $V_{TCA}$.
  
%   Fitzpatrick SM, Hetherington HP, Behar
% KL, Shulman RG. 1990. The flux from glu- cose to glutamate in the rat brain in
% vivo as determined by 1H-observed, 13C edited 61. NMR spectroscopy. J. Cereb.
% Blood Flow Metab. 10:170–79

  \item After technical improvements to NMR were made, in 1994, it became possible to measure
in the human brain the label flow from glutamate into glutamine.

% Gruetter R, Novotny EJ, Boulware S, Ma- son GF, Rothman DL, et al. 1994. Local-
% ized 13C NMR spectroscopy in the human 63. brain of amino acid labeling from
% [1-13C] glucose. J. Neurochem. 63:1377–85

Glutamine, although not a neurotransmitter, had been proposed to be an
intermediate in the glutamate neurotransmitter cycle.
  
  \item Since 2000, advanced NMR systems enable in vivo measures of fluxes in several
  cubic centimeters of human brain by $^{1}$H-$^{13}$C NMR.
  
  The carbon isotope is measured indirectly through scalar coupling to bound
  proton nuclei (Pan et al., 2000; Hyder, Renken, Rothman (1999)). This flux
  measurement provides valuable quantitation of metabolic rates in vivo, not
  obtainable by any other methods.
   
%    Pan JW, Stein D, Mason GF, Rothman DL, Hetherington HP. 2000. Gray and white
%    matter metabolic rate in human brain by spectroscopic imaging. Magn. Reson.
%    Med. In press
%    
%    Hyder F, Renken R, Rothman DL. 1999. In
% vivo carbon detection with proton echo pla- nar spectroscopic imaging (ICED
% PEPSI): [3,4-13CH2]glutamate/glutaminetomogra- 60. phy in rat brain. Magn.
% Reson. Med. 42: 997–1003
  
  \item 
  
  
  
\end{enumerate}

\textcolor{red}{BIG QUESTION}: How neurotransmitter fluxes support the
transmission of information and how this process depends upon brain energy
consumption.

\begin{mdframed}

Dozens of neurotransmitters have been identified and characterized, and
properties of several, e.g. dopamine, serotonin, acetylcholine, and
norepinephrine, have been studied intensively for decades. Two
neurotransmitters, glutamate and gamma-amino-butyric-acid (GABA), stand out
because of their high concentra- tions (in the millimolar range).
Furthermore, in the mammalian cortex, more than 90\% of the neurons serve these
two neurotransmitters.

\end{mdframed}




\subsection{Workflows}

{\bf Workflows}:
$^{13}$C fluxomics workflows have been performed for more than two decades, but
no standard protocol is in sight. One major reason is the inherent dependency of
the biological, analytical and computational subprotocols involved:

Example: intracellular metabolic flux rates with $^{13}$C isotope labeling
experiments (ILEs).
\begin{enumerate}
  \item  precisely adjusting the environmental parameters of the cultivation
  and, then, 
  
  \item at t0 = 0 replace the substrate with an isotopically labeled one.
  
  \item samples are taken over time:
  
Intracellular and extracellular metabolites are extracted, separated and
analyzed with respect to their concentrations and labeling states using targeted
mass spectroscopy (MS) or nuclear magnetic resonance spectrometry (NMR).

\end{enumerate}

\subsection{Challenge}

Unfortunately, reliable in vivo data on intracellular enzyme kinetics is still
sparse.
To date, the main stumbling block of $^{13}$C fluxomics is still the lack of
spatial resolution of the measurements. Assuming a metabolic pseudo-steady state
(constant fluxes and pool sizes), the family of less complex metabolic
stationary $^{13}$C MFA techniques targets short isotopic transient and
stationary regimes.

\begin{itemize}
  
  \item   In the case of simple transport reactions between the cellular
  environment and cytosol or simple network topologies, gross flux rates can be
  derived by balancing extracellular metabolite concentration changes
  
  
\end{itemize}

\subsection{Effect of gliotoxins}

A series of glutamate analogues, known as {\bf gliotoxins}, are toxic to
astrocytes in culture, and are inhibitors or substrates of high affinity
sodium-dependent glutamate transporter.
The effects of these gliotoxins, e.g. (L-$\alpha$-aminoadipate,
L-serine-O-sulphate, d-aspartate and L-cysteate) on metabolism of
[1-$^{13}$C]glucose was examined in c6 glioma cells; and the result showed that
the presence of gliotoxins profoundly alters (reduce) the flux of glucose to
lactate, alanine, aspartate and glutamate (Brennan et al., 2003).

\subsection{Compartmental analysis: Pulse- vs. Dynamic- labelling}

The metabolic fluxes analysis requires modeling the network systems in one or
many compartments; and then fit the fluxes using single or multiple-exponential
components equation.


\section{Fitting 13C time-course to metabolic flux}

Many studies confirmed that cerebral metabolism can be characterized by two
distinct metabolic compartments associated with neurons and glial cells.


\section{MCA}
\label{sec:MCA}

MCA is a well-developed theory that defines the control of flux in a pathway in
terms of the concentrations and kinetic properties of constitutive enzymes and
shows how it can be determined by achievable experimental protocols


MCA provides a theoretical basis for analyzing the control of metabolism and
has proven to be extremely well suited for interpreting in vivo NMR
measurements. MCA, in conjunction with in vivo NMR measurements, offers the
opportunity to utilize existing information about biomolecules in furthering our
understanding of more complex physiological functions


The flux coefficient $C^J_i$

\begin{equation}
C^J_i = \frac{\partial J/J}{\partial E_i/E_i}
\end{equation}


