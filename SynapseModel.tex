
\chapter{Model Synaptic Plasticity}
\label{chap:synapse-model-interc}

This chapter describes models that uses tetanic burst as synaptic input.
Those that look into pre-synaptic and post-synaptic spikes time relationship
in STDP (Sect.\ref{sec:STDP}) is discussed in the next chapters.

Sect.\ref{sec:synaptic-strength-measure} discuss how we can model the
(post)synaptic input, and from that we can derive the current-voltage
relationship (Sect.\ref{sec:synapse_I-V-curve}).

\section{Different forms of synaptic plasticity}

%\subsection{}
\subsection{STDP}

Spike-timing-dependent plasticity (STDP) has emerged in recent years as
the  experimentally  most  studied  form  of  synaptic  plasticity.


\section{Model synaptic plasticity}
\label{sec:synaptic_plasticity_model}

% Based on Hebbian theory (Sect.\ref{sec:synaptic_plasticity}), the efficiency of
% on nerve cell A on firing the next cell B can change. To model the synaptic
% plasticity, an important variable is the efficiency $w_{ij}$, with $i,j$ are the
% index of two connected nerve cells and cell $i$ with axon connecting to dendrite
% of cell $j$. Hebbian theory stated that the value of $w_{ij}$ depends on
% \begin{itemize}
%   \item the state of presynaptic neuron $i$
%   \item the state of postsynaptic neuron $j$
%   \item the present efficacy $w_{ij}$
% \end{itemize}
% but not on the state of other neurons $k$,
% Fig.\ref{fig:synaptic_plasticity_efficacy}. We call it
% the connection between these neurons is {\bf potentiated}.
% 
% Let y be some measure of posite type at a given separation are roughly equal and
% the activity of the postsynaptic cell, xi a measure of the opposite to
% correlations between two same type inputs activity of the i at that separation.
% Both of these are postulated to occur th input
% 
% 
% There are different rules that can be used to explain this Hebbian learning
% scheme ($w_{ij}$ is the weight or presynaptic $A_j$'s efficiency and
% post-synaptic $B_i$, Fig.\ref{fig:artificial-neuron}) [NOTICE the order:
% postsynaptic $i$ written first]
% \begin{enumerate}
%   \item simplest rule: the change in synaptic weight connecting
%   presynaptic region $j$ to postsynaptic region $i$ 
%   \begin{equation}
%   \Delta w_{ij}= k x_i x_j
%   \end{equation}
%   with $k$ is the scaling factor (representing the {\it learning rate});
%   firing rate $x_j$ of the presynaptic region, and firing rate $x_i$ of the
%   postsynaptic region $y_i$.  
% 
%   
%   \item 
% \end{enumerate}
% 
% 
% 
% Interestingly, 
% 
% \url{http://www.scholarpedia.org/article/Models_of_synaptic_plasticity}

Graupner and Brunel, 2012, PNAS: calcium-based bistable synaptic plasticity
model.


Up-regulation of NMAD receptor (NMADR) is the basis for pain synaptic
plasticity.


\section{Model Spike timing-dependent plasticity}
\label{sec:spike-timing-dependent-plasticity_model}

STDP focus the role of individual spike, in addition to frequency (bursting), 
with relative time of postsynaptic firing in controling the synaptic plasticity
(LTP and LTD) - Sect.\ref{sec:spike-timing-dependent-plasticity}.


\subsection{'high-level': Phenomenological models}

General variables that represent the net effects of the underlying mechanisms
or by applications of 'learning rule' that changes synaptic strengths
(Sect.\ref{sec:synaptic_plasticity_model})
\begin{equation}
w_{ij} = f(\Delta t)
\end{equation}
$\Delta t$ = the interspike interval (ISI) or time delay between pre- and
post-synaptic spikes.

Details in Chap.\ref{chap:models-phenomenological_STDP}.


\subsection{'detail-level': Biological model}

These models require explicit modelling of 
\begin{enumerate}
  \item back-propagating AP (bAP) - Sect.\ref{sec:bAP}
  
  \item neuronal potential
  
  \item individual ion channels
  
  \item chemical networks
  
  \item other low-level biophysical phenomena
\end{enumerate}


Details in Chap.\ref{chap:ModelIntermediate_STDP} and
Chap.\ref{chap:Model-LowLevel_STDP}.




\section{Statistical modelling transmitter release}
\label{sec:stat-modell-transm}

Consider the neuron transmitter are released in vesicles, each vesicle
is a unit, or quantum.  Suppose one transmitter quantum cause a
postsynaptic action $p$ (in units of potentials, conductance, or
change transfer).  The prob. that one quantum will be released from a
single release site after an action potential is $q$. If there are
totally $n$ sites, then the mean amplitude of the EPSP is $npq$.
\textcolor{red}{This is a binomial distribution}.

\subsection{in CA3 cells}
\label{sec:ca3-cells}

The values of $n$ are different from statistical analysis and
morphological counting. From statistical analysis, $n$ is in the range
3-30, much lower than those from neuromusclar junctions and squid
giant synapse.

Consider a simple model of a synapse with $n=3$ release sites, the
prob. for releasing a quantum successfully is $p=0.69$, and each
quanta cause a postsynaptic potential $q=0.51$ mV. Then, the prob. for
a transmission failure at all release site is $(1-0.69)^3=0.03$.

The simple binomial model assumes the same prob. of releasing
transmitter from all sites, at a connection between 2 cells. 

A simple treatment to connectivity is: the cells are organized in
layers (classes), and cells in one layer only connect to those in
another layers.
\begin{enumerate}
\item how many cell classes
\item {\it synaptic divergence} = the number of synapses does one cell
  make
\item {\it convergence} = how many synapse does one cell receive
\item {\it connectivity} = probability that two cells from two
  different layers are connected. How does it varies with the distance
  between them.
\item is synaptic connectivity (as described above) uniform for
  similar cells (i.e. cells in the same classes).
\end{enumerate}
The divergence and convergence can be determined anatomically and
physiologically. 


\section{Bhalla - Iyengar (1999) - cAMP + CaMKII}
\label{sec:Bhalla-Iyengar-1999}

\textcolor{red}{TAKE HOME MESSAGE}:
Simple biochemical reactions can, with appropriate coupling, be used to store
information. Thus, the signaling pathways may constitute one locus for the
biochemical basis for learning and memory.

\citep{bhalla1999} developed a compartmental model of hippocampal CA1 neuron
(using GENESIS software - Sect.\ref{sec:GENESIS}).
$\Ca$ influx was modeled through NMDAR channels on the dendritic spine
(Sect.\ref{sec:NMDAR}).

Synaptic input was delivered as tetanic burst at 100 Hz, for 1 sec each,
separated by 600s to study Hebbian learning LTP (Sect.\ref{sec:LTP}).

The $\Ca$ waveform as output of the above model is fed into the kinetic model.


They linked pathways one by one, until the entire network model of interacting
pathways was formed. Pathways were linked by 2 forms of interactions:
\begin{enumerate}
  \item output (achidonic acid (AA) and diacylglycerol (DAG)) of one pathway is
  served as input to another.
  
  \item enzymes whose activation was regulated by one pathway were coupled to
  substrate belonging to other pathways.
\end{enumerate}

They studied different things, and one of them is CaMKII function in LTP, 
as persistently activated CaMKII in the postsynaptic CA1 neuron increases
synaptic responses (i.e. LTP).
\begin{itemize}
  \item CaMKII can be activated via autophosphorylation at Thr$^{286}$ even at
  low $[\Ca]$.
  
mutated CaMKII with Thr$^{286}$ replaced by Ala impaired LTP.   
\end{itemize}
However, certain forms of LTP also requires cAMP pathway.
cAMP is required, but it cannot change synaptic response by itself.
Thus, the authors suggested cAMP pathway gates CaMKII signalling through the
regulation of protein phosphastase.

PROPOSED MECHANISM: With the inflow of $\Ca$, $\Ca$/calmodulin (CaM) not only
activates CaMKII and calcineurin (CaN, or PP2B), but also elevates intracellular
cAMP throgh CaM-dependent activation of AC1 and AC8
(Sect.\ref{sec:AC_adenylyl_cyclase}).


\section{Spiros (2010) - corticostriatal}
\label{sec:Spiros-2010}

\citep{spiros2010} developed a corticostriatl synapse model 
to study activity level of postsynaptic receptor under the effect of dopamine
(DA), a D2-receptor binding drug, two additional D2-receptor binding agents
(metabolite and/or radioactive tracer) in a competitive binding model.

The model takes into account the
\begin{itemize}
  \item different firing patterns of presynaptic terminal: tonic spiking and
  high-frequency spiking
  
  \item effect of D2-like autoreceptor to DA release
  
  \item effect of  presynaptic facilitation and depression to DA release
\end{itemize}

\subsection{Mathematical model}

Initially
\begin{verbatim}
DA_zero = 

D2receptor = [D2receptor]_total  

drug = [drug]_total

tracer = [tracer]_total

...
\end{verbatim}

Variables
\begin{verbatim}
DA   // free
DAhighaffinity_D2receptor
DAlowaffinity_D2receptor
drug_D2receptor
metabolite_D2receptor
tracer_D2receptor

\end{verbatim}

The purpose of tracer is to incorporate the effect of radiotracer (e.g.
C-raclopride, I-IBZM) being used.

to emulate the effect of using antagonist (e.g.
haloperidol) on D2 receptor, i.e. increased free DA 

and we need to solve a set of ODEs
\begin{verbatim}
Example:
d[DAhighaffinity_D2receptor]/dt = kon_DAh * [DA] * [D2receptor] 
               - koff_DAh * [DAhighaffinity_D2receptor]
 NOTE: koff_DAh = kon_DAh * Kd_Ah
 Kd_Ah = 10 nM

d[DAlowaffinity_D2receptor]/dt = kon_DAl * [DA] * [D2receptor] 
               - koff_DAl * [DAhighaffinity_D2receptor]

d[drug_D2receptor]/dt = kon_drug * [drug] * [D2receptor] 
               - koff_drug * [drug_D2receptor]

d[tracer_D2receptor]/dt = kon_tracer * [tracer] * [D2receptor] 
               - koff_tracer * [tracer_D2receptor]
   NOTE: [tracer]_total = 1 pM
         Kd_tracer = 1.3 nM (for raclopride)
                     0.6 nM (for IBZM)
                     0.018 nM (for FLB457)

d[DA]/dt = f(release) - f(decay) - f(reaction_loss)

 // Dopaminergic neurons tend to switch between 
 //     low-frequencity tonic firing and high-frequency bursting firing
 // patterns
f(release) = 1. following a user-defined set of firing patterns
                modulated by D2-autoreceptor
             2. time-dependent facilitation and depression
                
 // DA removal from the cleft is modeled
 // as an exponential decay with rate is adjusted
 // to match DA kinetics in various brain regions 
 // taking into account 
 //     1. DA diffusion
 //     2. DAT and/or catabolic enzyme (e.g. COMT) mechanism   
 //    tau_halflife = 30-50ms  (rodent striatal n.accucmbens)
f(decay) = exp( - t * ln(2) / tau_halflife)

f(reaction_loss) = 
\end{verbatim}

{\bf Assumption}: the binding 'on' rate of DA is diffusion limited, i.e. the
reaction occurs quickly and depends on the size and molecular weight of the
molecules using Stokes-Einstein equation. 

\begin{verbatim}
DA_release = s
\end{verbatim}

\subsection{Model tuning}

The model was calibrated using data on DA dynamics measured with fast cyclic
voltammetry (Sect.\ref{sec:FSCV}).


\subsection{Analysis}

{\bf Limitation}: 
\begin{enumerate}
  \item The model outcome is presented as the level of postsynaptic D2
receptor activation; without taking into account any physiological intracellular
effect or pathway activation.

  \item no spatial aspect, i.e. assume all agents have perfect access to each
  other
  
  CONS: this could lead to overestimation of the amount of bound receptors. 
 
 BUT: For understanding the relative effects, e.g. inhibition fraction, the
 model behavior is resonably well. Also, the space is small in the cleft, making
 it more similar to the well-mixed solution.
  
  \item relation between presynaptic D2 autoreceptor and DA release after
  stimulation: phenomenological representation, rather than an exact biological
  way (check \ref{qi2008})
  
  \item 
\end{enumerate}


\section{Qi (2008)}
\label{sec:Qi-2008}




%%% Local Variables: 
%%% mode: latex
%%% TeX-master: "mainfile"
%%% End: 
