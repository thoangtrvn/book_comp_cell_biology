\chapter{NeuroMorphic Computing}
\label{chap:NeuroMorphic-Computing}


{\bf NeuroMorphic Computing} is a concept developed by Carver Mead in early
1980s, who is a pioneer in modern micro-electronics. In 1980s, he focused on
electronic modelling of human neurology and biology.
\footnote{\url{https://en.wikipedia.org/wiki/Carver_Mead}}
 
Observing graded synaptic transmission in the retina, Mead became interested in
the potential to treat transistors as analog devices rather than digital
switches.

In 1986, Mead and Federico Faggin founded Synaptics Inc. to develop analog
circuits based in neural networking theories, suitable for use in vision and
speech recognition.  The first product was pressure-sensitive computer touchpad.

Mead's work underlies the development of computer processors whose electronic
components are connected in ways that resemble biological synapses.
They pioneered the use of loating-gate transistors as a means of non-volatile
storage for neuromorphic and other analog circuits.

{\it They are not "programmed." Rather the connections between the circuits are
"weighted" according to correlations in data that the processor has already
"learned." Those weights are then altered as data flows in to the chip, causing
them to change their values and to "spike." That generates a signal that travels
to other components and, in reaction, changes the neural network, in essence
programming the next actions much the same way that information alters human
thoughts and actions. }
\footnote{\url{http://www.nytimes.com/2013/12/29/science/brainlike-computers-learning-from-experience.html?_r=0}}


