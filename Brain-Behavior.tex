\chapter{Brain - Behavior: How the brain function give rise to behavior?}

To maintain normal brain function, neural activity needs to be tightly
coordinated within the brain network. How this coordination is achieved and
related to behavior is largely unknown.

\begin{enumerate}
  \item It has been previously argued
that the study of the link between brain and behavior is impossible without a guiding vision

  \item formal description of behavior as a low-dimensional process emerging
  from a network's dynamics dependent on the symmetry and invariance properties
  of the network connectivity


\end{enumerate}

\section{Behavior: sum(internally coordinated actions)}

The great Dutch ethologist Nikolaas Tinbergen (1907-1988) defined behavior as
''the total movements made by the intact animal, emphasizing the need to include
the plurality of movements as opposed to a single trajectory (Tinbergen, 1951).


Levitis, Lidicker, and Freund (2000) offer a definition, based on a systematic
analysis of survey responses, in which behavior constitutes the set of {\it
internally coordinated actions} (or inactions) of an organism in the presence of
internal and/or external stimuli.

It emphasizes the manifestation of low-dimensional patterns from numerous
processes and constraints (internal and external to the actor; Huys et al.,
2014).

Scott Kelso formalized coordination as the functional ordering of interacting
components in space and time (Kelso, 1995, 2012).

Krakauer et al. (2017) emphasized, for instance, that the neural basis of
behavior cannot be understood without allowing for independent detailed study of
the behavior itself.
They pointed out that the current focus on neural circuits will not yield the
kind of insight and explanation that we ultimately demand for the understanding
of the link between brain and behavior.

Pillai, Jirsa (2017) suggested that behavior Is the Set of Coordinated Actions
in a Task-Specific Contex: the complete set of actions comprising all behavioral
patterns establishes one behavior.
In other words, one behavior comprises the full range of the dynamic repertoire.

\subsection{Mathematical representation of behavior}

Behavior is a low dimensional set of ODE 
\begin{equation}
\dot{\zeta} = N(\zeta(t)))
\end{equation}
with time-dependent behavioral state vector (or action variables) $\zeta(t)$ is
M-dimensional. The right-hand side is the {\bf flow of the system}.

Hermann Haken and Scott Kelso demonstrated that behavior (once again: the
functional ordering of its action variables) may be multistable and express
multiple coexistent attractors (Haken, 1996; Kelso, 2012).

Hermann Haken and Scott Kelso (1985) used changes in the stability of
trajectories through the mechanism of transitions between movement patterns as a
paradigm to interrogate human behavior and its dynamics.
\begin{itemize}
  
  \item  individual attractors as behavioral patterns, for instance work on
  transitions from syncopation to synchronization pattern in rhythmic movements
  of index fingers (Haken et al., 1985).

\end{itemize}

The attractors of the dynamic system, as well as the entire flow of the
attractors, establish the behavioral patterns, which as an ensemble define the
behavior and to which we refer to as structured flow on manifold.

\url{https://www.genome.gov/27545740/ajay-pillai-phd/}


\section{structured flows on manifolds (SFMs)}


structured flows on manifolds (SFMs) is a set of tools
and concepts 

Pillai and Jirsa (2017) showed that behavior emerges when appropriate conditions
imposed upon the couplings are satisfied, justifying the conductance-based
nature of synaptic couplings.

\subsection{SFM-based network mechanisms}


It is applied to a toy model - which is a classic task in motor behavior,
the Fitts' paradigm.

