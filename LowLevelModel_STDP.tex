\chapter{Low-Level Models of STDP}
\label{chap:Model-LowLevel_STDP}

This types of model incorporate all features of intermediate models of STDP
(Chap.\ref{chap:ModelIntermediate_STDP}), with explicitly include descriptions
of neuronal parameters and plasticity mechanisms.

A biophysical model of STDP must contain at least 2 primary ingredients:
\begin{enumerate}
  \item a mechanism through which the synaptic strength can exist in at least
  one of the three state: LTP state, LTD state and basal state.

To be considered as a stable state, it must be resistant to small pertubations
(for deterministic model). Example: After applying a step pulse of $[\Ca]$
(e.g. from 0.1$\muM$ as basal level to 0.5$\muM$) for a certain time (e.g.
2sec), when the pulse is terminated (i.e. $[\Ca]$ back to basal level), the
model should be able to return to the original state at the end.
  
  \item 
\end{enumerate}



\section{Graupner - Brunel (2007) + CaMKII - bistable switch}
\label{sec:Graupner-Brunel-2007}


The model incorporate CaMKII and investigate its phosphorylation effect
and help to explain {\bf bistable chemical switches}.

\section{Pi-Lisman (2008) - tristability switch}
\label{sec:Pi-Lisman-2008}


The key idea is that PP2A (Sect.\ref{sec:PP2A}) can form a phosphatase switch
that could underlie the maintenance of LTD.

ASSUMPTION:
\begin{itemize}
  \item The model did not incorporate the dynamic of $\Ca$, instead they tested
  the model by applying different pulse level of $[\Ca]$ (step size 0.1$\muM$
  for 2sec, and resting level $[\Ca]=0.1\muM$)
  
  
\end{itemize}

\subsection{Numerical analysis}

The model was implemented on MatLab, and fourth-order Runge-Kutta method.

\section{Urakubo et al. (2008)}
\label{sec:Urakubo-2008}	


\section{Carlson - Giordano (2011) - tristable switch}
\label{sec:Carlson-Giordano-2011}

The model explicitly include neuronal parameters + plasticity mechanism, based
on Graupner-Brunel (2007) - Sect.\ref{sec:Graupner-Brunel-2007}.

Based on the experimental result that the mechanism for depotentiation is not
the same as LTD (Zhuo et al. 1999), it is expected that {\bf tristable chemical
switch} is required to appropriately describe LTP and LTD (Pi and Lisman 2008).

The model combine (1) compartmental model of postsynaptic neuron
(Sect.\ref{sec:Shouval-2002}), (2) calcium dynamics with a tristable
kinase/phosphastase network (Pi and Lisman, 2008).
This is the first model to pair both STDP-induced postsynaptic $\Ca$ dynamics +
tristable biochemical switch, that can yield LTP state, LTD state, and basal
synaptic state.

ASSUMPTION:
\begin{itemize}
  \item instantaneous binding of glutamate to NMDAR (using step function)
  while glutmate unbinding is modeled via two step function
  
  \item 
\end{itemize}

\subsection{Mathematical model}

The postsynaptic side is modeled as a single equipotential compartment with both
NMDAR and VDCC present as source of $\Ca$ influx.
To account for buffering, a decay term is added as a source driving $[\Ca]$ back
to basal level (based on Sect.\ref{sec:Shouval-2002}).


\subsection{Data analysis}


The model show good qualitative agreement with data from
Sect.\ref{sec:Bi-Poo_1998} for both pulse pair protocol and blockage of L-type
VDCC protocol.


\section{Graupner - Brunel (2012) - bistable}
\label{sec:Graupner-Brunel-2012}

A phenomenological model proposed by Graupner-Brunel to model the synaptic
efficacy (Sect.\ref{sec:synaptic-efficacy}) as a function of $\Ca$ dynamics.

The model implements two different $\Ca$-triggered pathways that can mediate
synaptic strengths
\begin{itemize}
  \item LTP: via protein kinase cascades
  \item LTD: via protein phosphatase cascades of G-protein cascades
\end{itemize}

\subsection{Hypothesis}

The model hypothesize that 
\begin{enumerate}
  \item pre-synaptic signal trigger a smaller release of $\Ca$ on postsynaptic
  side (as it is assumed only NMDAR open)
  
  \item the post-synaptic signal (e.g. bAP) triggers a bigger release of $\Ca$
  on postsynaptic side (as it is assumed VDCC open)
  
  \item ignore NMDAR nonlinearity
  \item ignore finite rise-times, i.e. assume instantaneous reaching the peak
  \item the same decay time constant for calcium via NMDAR and VDCC
  
  \item there are two thresholds of $[\Ca]_i$: one ($\theta_p$) leading to
  potentiation, and one ($\theta_d$) leading to depression
  
They assumed $ \theta_d < \theta_p$
\begin{verbatim}
Ca2+ level

  ...(above) --> LTP
theta_p

   ...(between)  --> LTD

theta_d
  ...(below) --> no change
  
\end{verbatim}

  \item the two quantitaties: $\gamma_p, \gamma_d$ measures the rate of synaptic
  increase (or decrease)  when potentiation threshold (or depression threshold)
  exceeds, respectively.
  
  
\end{enumerate}
%the synaptic efficacy increases


\subsection{Mathematical model}

A model of a single synapse submitted to trains of pre- and postsynaptic APs
is developed. 

\subsection{- synaptic efficacy}

The state of a synapse is represented via the \textcolor{red}{synaptic efficacy
variable} $\rho(t)$, described via first-order ODE

\begin{equation}
\tau\frac{d\rho}{dt} = \underbrace{-\rho (1-\rho)(\rho_* -
\rho)}_{\text{first}} + 
\underbrace{\gamma_p \times (1-\rho) \times \Phi
[c(t)-\theta_p]}_{\text{second}} - \gamma_d \times \rho \times \Phi [c(t) -
\theta_d] + \text{Noise}(t)
\end{equation}

  NOTE: the time constant $\tau$ of synaptic efficacy change in the order of
  seconds to minutes; the quantities $\gamma_p, \gamma_d$ measures the rates of
  synaptic increase/decrease when potentiation/depression thresholds are
  exceeded, respectively.

\begin{itemize}
  
  \item the first term on right-hand side: the dynamic of synaptic efficacy in
  the absence of pre- and post-synaptic activity, which is modelled as a cubic
  function (Sect.\ref{sec:cubic-function}).
  
  The cubic functional form is chosen so that it has 2 stable states at rest,
  representing low efficacy ($\rho =0$, DOWN state) and high efficacy
  ($\rho=1$, UP state). The boundary of the basin of attraction between the two
  states is $\rho_* = 0.5$.
  
  \item the second term on the right-hand side: $\Ca$-dependent signaling
  cascades leading to synaptic potentiation 

\begin{mdframed}
The simple form of the function $\Phi$ on the second and third terms on the
right-hand side uses Heaviside function $\Phi[c-\theta]$ which get either
 \begin{equation}
 \Phi[c-\theta] = \left\{ \begin{array}{c}
 		0  \;\; \text{ if } c<\theta \\
 		1  \;\; \text{ if } c\ge\theta \\
                \end{array}\right.
 \end{equation}
\end{mdframed}  

  \item the third term on the right-hand side: $\Ca$-dependent signaling
  cascades leading to synaptic depression


  
  \item the last term on the right-hand side: model the activity-dependent
  fluctuation stemming from stochastic neurotransmitter release, stochastic
  channel openning and diffusion.
  
\begin{mdframed}
  The noise is assumed to present whenever the calcium level below the threshold
  for potentiation and/or depression.
  \begin{equation}
  \text{Noise}(t) = \sigma \sqrt{\tau}\Phi[c(t) - \min(\theta_d,
  \theta_p)]\eta(t)
  \end{equation}
  with $\sigma$ is the amplitude of noise, $\eta(t)$ is the Gaussian white noise
  process with unit variance density
\end{mdframed}

\end{itemize}

\subsection{- calcium dynamics}

Calcium release is assumed as an instantaneous release reaching the
peak (at the time the signal come), and then decay with a given time constant.
There are two types of signal that can trigger the release
\begin{itemize}
  \item pre-synaptic signal: which trigger the release after a delay D (ms)
  
  \item post-synaptic signal: which trigger the release immediately
\end{itemize} 

$\Ca$ is removed from the spine via some passive process with a given time
constant $\tau_\ca$.

\def\Cpre{{\text{Cpre}}}
\def\Cpost{{\text{Cpost}}}

The postsynaptic $\Ca$ dynamics
 \begin{equation}
 \frac{dc}{dt} = \underbrace{- \frac{c}{\tau_\ca}}_{\text{removal}} +
   \underbrace{\Cpre \sum_i \delta(t - t_i - D)}_{\text{NMDAR current}} +
   \underbrace{\Cpost\sum_j \delta(t-tj)}_{\text{VDCC current}}
 \end{equation} 
 with $\Cpre$ is the peak of the $\Ca$ elevation under pre-synaptic signal; and
$\Cpost$ is the peak of $\Ca$ elevation under post-synaptic signal.


 \section{Trick}
 
The time $[\Ca]_i$ above both threshold determines the time course for LTP to
maintain. This depends on 2 factors from the model setting
\begin{itemize}
  \item the peak of $[\Ca]_i$ which of course higher when time of release via
  NMDAR and via VDCC are closed to each other
  
  \item the decay setting $\tau_\ca$ which is 150 second = 2.5 minutes, which is
  the larger the longer the time LTP to be maintained.
  
  \item the value of $\gamma_p$ and $\gamma_d$: 
  the bigger $\gamma_p$ (and calcium transient), the more probability the
  synaptic efficacy $\rho$ can jump from DOWN to UP state (here, it is a random
  process due to the incorporation of Noise(t)).
  
  
\end{itemize}
