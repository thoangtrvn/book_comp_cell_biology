\chapter{DARPA's SyNAPSE project: IBM's TrueNorth neural net chip}
\label{chap:SYNAPSE}

The SyNAPSE project is DARPA's program that aims to develop a
neuromorphic system (Sect.\ref{chap:NeuroMorphic-Computing}), i.e. a multi-chip
on a single board, using IBM's TrueNorth chip model.

{\bf SyNAPSE = Systems of Neuromorphic Adaptive Plastic Scalable Electronics.}

The ultimate vision of the DARPA SyNAPSE program is to build a cognitive
computing architecture with $10^{10}$ neurson and $10^{14}$ synapses. This
approximates the number of neurons and synapses that are estimated to be present
in the human brain.

\section{TrueNorth chip}
\label{sec:TrueNorth-chip}

TrueNorth core (digital 'neurosynaptic' core)
\begin{itemize}
  \item as in 2011: 1 kHz clock $\sim$ 1ms biological time-step. An internal 1
  MHz clock was used for other processing.

\begin{enumerate}
  \item a core is a system with 256 leaky integrate-and-fire 'neurons'
  arranged in 16x16 array.
  
In total: a core has 256 axons, 256 dendrites feeding to 256 neurons.

The leaky IaF dynamics is characterized  by  configurable parameters sufficient
to produce a rich repertoire of dynamic and functional behavior.
A neuron on any TrueNorth core can connect  to  an  axon  on  any  TrueNorth 
core  in  the  network; when the neuron fires it sends a spike message to the
axon.
    
  \item a single leaky IaF 'neuron' is connected with another neuron via 4
  programmable synapse (simple binary synapses), i.e. there are total 4x256x256
  = 262,144 synapses per core.
\end{enumerate}

  \item as a non-Von Neuman architecture, the computation and memory units of
  this chip are tightly integrated, i.e. memory are associated with individual
  cores.
  
NOTE: Computation ("neurons"), memory ("synapses"), communication ("axons",
"dendrites") are mathematically abstracted away from biological detail towards
engineering goals of maximizing function (utility, applications), minimizing
cost (power, area, delay), and minimizing design complexity of hardware
implementation.
  

\end{itemize}

In 2012: IBM started working on inter-core communication to build a large
on-chip network of these cores (a communication  network  to  other  TrueNorth 
cores  in  the  system).  A TrueNorth chip is an interconnected and
communicating  network  of  extremely large numbers  of neurosynaptic  cores.
\begin{itemize}
  \item As of 2013: a TrueNorth chip has 4096 TrueNorth cores
  
  \item  each  core  integrates computation  (neurons),  memory  (synapses), 
  and  intra-core communication  (axons)
  
  \item each  core  is  event-driven  (as  opposed  to  clock-driven),
  reconfigurable, and consumes ultra-low power.

TrueNorth chip operates at lower temperatures and less power because it operates
only when it needs, rather than all the time.

  \item cores operate  in  parallel  and  send  unidirectional  messages  to 
  one another;  that  is,  neurons  on  a  source  core  send  spikes  to
axons on a target core.

{\it A neuron on any TrueNorth core can connect  to  an  axon  on  any 
TrueNorth core  in  the  network; when the neuron fires it sends a spike message to the
axon.  TrueNorth  cores  are synced in a semi-synchronous fashion, i.e. each
core has its own internal faster clock (1MHz), and it receives  a slow clock
tick at 1000 Hz to discretize neuron dynamics in 1-millisecond time steps.

When a TrueNorth core receives a tick from the slow clock, it cycles through
each of its axons.

For  each  axon,  if  the  axon  has  a  spike  ready  for  delivery  at the
current time step in its buffer, each synaptic value on the horizontal  synaptic
 line  connected  to  the  axon  is  delivered to  its  corresponding 
post-synaptic  neuron  in  the  TrueNorth core. If the synapse value for a
particular axon-neuron pair is non-zero,  then  the neuron  increments its 
membrane potential by  a  (possibly  stochastic)  weight  corresponding  to  the
 axon type.  After  all  axons  are  processed,  each  neuron  applies  a
configurable,  possibly  stochastic  leak,  and  a  neuron  whose membrane 
potential  exceeds  its  threshold  fires  a  spike.  Each spike  is  then 
delivered  via  the  communication  network  to its  corresponding  target 
axon.  An  axon  that  receives  a  spike schedules  the  spike  for  delivery 
at  a  future  time  step  in  its buffer.  The  entire  cycle  repeats  when 
the  TrueNorth  core receives the next tick from the slow clock.}
\url{http://www.modha.org/blog/SC12/SC2012_Compass.pdf}

\end{itemize}  

This has 10 billion 'neurons' - more than 10 percent of a human brain. It ran
about 1,500 times more slowly than an actual brain. Further, it required several
megawatts of power, compared with just 20 watts of power used by the biological
brain.



\section{Other cricisms}

Yann Le Cun has critized about this project ``
{\it TrueNorth implements networks of integrate-and-fire spiking neurons. This
type of neural net that has never been shown to yield accuracy anywhere close to
state of the art on any task of interest (like, say recognizing objects from the
ImageNet dataset)}''.
\footnote{\url{https://www.facebook.com/yann.lecun/posts/10152184295832143}}

Yann's main cricisim was that ``Spiking neurons have binary outputs (like
neurons in the brain). The advantage of spiking neurons is that you don't need
multipliers (since the neuron states are binary). But to get good results on a
task like ImageNet you need about 8 bit of precision on the neuron states. To
get this kind of precision with spiking neurons requires to wait multiple cycles
so the spikes "average out". This slows down the overall computation.''


\section{Application}

One of the big thing for a new architecture is to be able to demonstrate
the performance gain from real-world applications using such new systems.

\subsection{Compass}

Compass, also developed by IBM, is software that simulates the TrueNorth
architecture. \footnote{\url{http://www.modha.org/blog/SC12/SC2012_Compass.pdf}}

Compass is a multi-threaded, massively-parallel functional simulator.
Compass is also a parallel compiler that can map a network of long-distance
neural pathways in the macaque monkey brain to TrueNorth.


Compass has been used to demonstrate numerous applications of the TrueNorth
architecture, such as optic flow, attention mechanisms, image and audio
classification, multi-modal image audio classification, character recognition,
robotic navigation, and spatio-temporal feature extraction.  

\url{http://www.artificialbrains.com/darpa-synapse-program#truenorth-compass}


\subsection{Cafee?} 

The effort to test if the machine learning tool - Cafee - can be utilized
efficiently on TrueNorth chip or not was also initiated.



