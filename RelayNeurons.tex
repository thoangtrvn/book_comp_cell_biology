\chapter{Relay Neurons (Interneurons)}
\label{chap:RelayNeurons}
\label{chap:interneurons}

Thalamic relay neurons are capable of exhibiting multiple
distinct firing modes depending on the stimulus
holding level.

When sufficiently depolarized from rest, a relay neuron shows continuous spiking
(CS) activity, with with $\Na$-dependent action potentials at high frequencies,
$>	100$ Hz. 

This response to excitation, the {\bf 'relay mode'}, corresponds to the
transmission of sensory information through the thalamus to the cortex in the
awake state (Steriade et al. 1990).

{\bf Low-threshold spike}:
When the membrane is hyperpolarized, a transient calcium current slowly
deinactivates (aka $I_h$ current - Sect.\ref{sec:Ih-current}) and, on release
from this voltage level, mediates a slow depolarizing wave, the low-threshold
spike (LTS). This rebound excitation frequently exceeds the voltage threshold
for the generation of sodium action potentials so that a burst of spikes rides
the crest of the LTS.

Bursting is seen naturally during certain phases of sleep, when neuromodulators
lead to hyperpolarization in the thalamus, and during absence seizures.
Other than on release, bursting behavior may occur repetitively (1-10 Hz) for
maintained hyperpolarizing stimuli in an appropriate range.


\section{Hindmarsh-Rush model (bursting) 1984-1989 - thalamic relay neuron}
\label{sec:Hindmarsh-Rush_model}
%\section{Rose-Hindmarsh (1989) - thalamic neuron}
\label{sec:Rose-Hindmarsh-1989}

Hindmarsh-Rush model is the first single cell model to reproduce spike-bursting
behavior of nerve cell, for thalamic relay neuron
(Sect.\ref{sec:thalamic-relay-neuron}) \citep{hindmarsh1984}.
%http://www.math.montana.edu/pernarow/neural_field/HindMarsh_Rose_Model/Hindmarsh_Rose_Model.pdf


They developed a model with 4 ionic currents; before the available of
voltage-clamp data to simulate the response under an applied current $I_\app$
\begin{enumerate}
  \item Na-current - Sect.\ref{sec:Ina_Rose-Hindmarsh-1989}
  \item potassium A-current
  \item T-type $\Ca$ current
  \item leak current
\end{enumerate}

The model, however, is dimensionless, with membrane is represented by $x(t)$,
the transport of sodium is represented by $y(t)$ - spiking variable; and the
transport of other ions is made throught the slow channels represented by
$z(t)$ - bursting variable. \textcolor{red}{IMPORTANT}: The third state variable
$z(t)$  allows a great variety of dynamic behaviors of the membrane potential,
including unpredictable behavior, which is referred to as chaotic dynamics.

\begin{equation}
\begin{split}
\frac{dx}{dt} &= y + \Phi(x) - z + I_\app \\
\frac{dy}{dt} &= \Psi(x) - y \\
\frac{dz}{dt} &= r \times [ s \times (x - x_R) - z]
\end{split}
\end{equation}
with
\begin{equation}
\begin{split}
\Phi(x) = -a x^3 + b x^2 \\
\Psi(x) = c - d x^2
\end{split}
\end{equation}
Usually the parameter I, which means the current that enters the neuron, is
taken as a control parameter. 
\begin{itemize}
  \item The parameter r is something of the order of $10^{-3}$
  \item When a, b, c, d are fixed the values given are a = 1, b = 3, c
= 1, and d = 5. 

  \item Frequently, the parameters held fixed are s = 4 and $x_R$ = -8/5. 
\end{itemize}


The model suggested A-type $\K$ current played an important role in the fast
spike-generating dynamics, with bursting in particular
(Sect.\ref{sec:A-type-K+current}).
However, in a more recent model, Rush-Rinzel (1994) showed that
the inclusion of a strong A-current in their model can annihilate bursting
behavior (Sect.\ref{sec:Rush-Rinzel-1994}). 



\section{Wang et al. (1991) - thalamic relay neuron}
\label{sec:Wang-1991-thalamic-relay-neuron}

\citep{wang1991} used whole-cell voltage-clamp data to build  transient,
low-threshold T-type $\Ca$ model. Wang et al. (1991) showed that the transient
(T-type) calcium current and a passive leakage current are sufficient to
generate the LTS response (Sect.\ref{sec:thalamic-relay-neuron}).
They did not include currents for generating fast action potentials, assuming
these could later be added to mimic bursting activity.

\begin{enumerate}
  \item leak current
  \item T-type $\Ca$ current - Sect.\ref{sec:LCC_Wang1991}
\end{enumerate}





\section{McCormick-Huguenard (1992) - thalamocortical relay neurons}
\label{sec:relay-neuron-McCormick-Huguenard-1992}
\label{sec:McCormick-Huguenard-1992}
\label{sec:Huguenard-Mccormick-1992}

This model is pretty much the same as the model published as Huguenard-McCormick
(1992) \cite{huguenard1992}; except with the addition of sodium conductance.

They found that the transient, A-type $\K$ current, along with other potassium
current, is dominant in the interval between action potentials, affecting both
repolarization and spike frequency. 
 
The mathematical model of rodent thalamocortical relay cells
(Sect.\ref{sec:thalamocortical-neuron}) was proposed with 4 ionic currents
\begin{enumerate}
  \item low-threshold T-type $\Ca$ current $I_\CaT$
  
  \item hyperpolarization-activated cation current $I_h$:
  Sect.\ref{sec:Ih-Huguenard-McCormick-1992}.
  
  \item transient, depolarization-activated $\K$ current $I_A$:
  
  \item slowly inactivating, depolarization-activated $\K$ current $I_{\k2}$
\end{enumerate}



\url{http://www.ncbi.nlm.nih.gov/pubmed/1331356}

%\section{Wang et al., 1991 - thalamic neuron}
\section{Rush - Rinzel (1994) - thalamic relay neuron}
%\section{Rush-Rinzel (1994) - thalamic neuron}
\label{sec:Rush-Rinzel-1994}

\citep{rush1994} extended Wang et al. (1991) -
Sect.\ref{sec:Wang-1991-thalamic-relay-neuron} with adding Hodgkin-Huxley model
for sodium and potassium currents derived from Hindmarsh-Rush
(Sect.\ref{sec:Hindmarsh-Rush_model}).
These currents and the T-type calcium current, are active in different voltage ranges
\begin{enumerate}
  \item T-type Ca2+ current (I$_T$): Sect.\ref{sec:T-type-Ca2+-Rush-Rinzel-1994} 

  \item Na-current - Sect.\ref{sec:Ina_Rush-Rinzel1994}

Shifting the currents for sodium action potentials to a higher-voltage regime
allowed for continuous spiking activity at a depolarized level without
compromising the I$_T$-mediated excitability in the low-voltage regime.
   
  \item delayed rectifier $\K$ current: $g_K$ ($I_K$) -
  Sect.\ref{sec:KDR-Rush-Rinzel-1994}
  
  \item The leak current is composed of two components: INaL, and IKL (sodium
  and potassium leak)

% TODO: sleep-wake modeling
The conductance IKL for the latter is treated as neuromodulator-sensitive to
represent various degrees of the sleepwake status; its resting value is 0.088 mS/cm$^2$.
\end{enumerate}

The current that generate action potential (AP)
\def\AP{{\text{AP}}}
\begin{equation}
\begin{split}
I_\AP = g_\na m_{\infty}^3 (0.85 - n) \times (V - E_\na) + \\
       g_\k n^4 \times (V - E_\k)
\end{split}
\end{equation}

NOTE: The model is minimal to reproduce LTS-mediated bursting (based on T-type
$\Ca$ current) and can produce the continuous spiking and bursting activity
modes in response to steady stimulus.

Only the gating $h$ is needed to be adjusted to approximate
body temperature 35$^\circ$C using a Q10 of 2.3 ($\Phi$ = 3).

The model was tuned to achieve total AP current dynamics when testing with
$I_\app = 14$ pA/cm$^2$.

This shows similar behavior to a previous model by Hindmarsh-Rush
(Sect.\ref{sec:Hindmarsh-Rush_model}) yet with opposite role of A-type current.
Here, the A-type $\K$ current can annihilate the bursting behavior.
The authors proposed that a potassium A-current shifts the threshold for sodium
spikes (by model the inactivating gating variable $h$ as a function of potassium
activation variable $n$, i.e. $h=0.85-n$), reducing the number of fast sodium
spikes in an LTS burst.


There are only 3 ODEs: Voltage, gating $n$ for $\K$ current and gating $h$ of
T-type $\Ca$ current. 


\subsection{A-type current effect}

They also studied the effect of potassium A-current (using the voltage-clamp
data for $I_A$ of Huguenard et al. 1991) to modulate and refine the basic
voltage patterns of continuous spiking and LTS activity, e.g. to reduce spiking
in both rebound and periodic burst activity.
The whole system now has a total 4 ODEs.

There are two formulas for A-type currents
\begin{enumerate}
  \item  activation is considered instantaneous
\begin{equation}
I_A = g_K \times a_\infty^4(\Vm) \times b \times (\Vm - E_{\rev,K})
\end{equation}

and
\begin{equation}
\frac{db}{dt} = \frac{b_\infty(\Vm)  - b}{\tau_b(\Vm)}
\end{equation}

  \item  For single rebound LTS study, a different formula for A-type current
  was used, with the new ODE for $b$ gating variable


  
\end{enumerate}

The role of A-type was previously studied in Huguenard-McCormick (1992) -
Sect.\ref{sec:Huguenard-Mccormick-1992}; yet the role of A-type in LTS has not
been done. For LTS generation, the transient A-current, as an outward current
imposed on the T-current, can decrease the rate of rise of the LTS and decrease
its peak amplitude.

Here, they studied the effect using fast/slow analysis
(Sect.\ref{sec:fastslow-analysis}).
\begin{itemize} 
  \item  Where the fast sodium spikes are situated in phase space now depends on
the A-current.

As a consequence, the number of sodium spikes in an LTS burst is modulated; in
fact, it is possible for no spikes to occur. This contrasts with the Rose and
Hindmarsh model (1989b) in which the generation of bursting seems to rely on an
A-current

  \item 
\end{itemize}





% The authors extended the previous work (Sect.\ref{sec:Rose-Hindmarsh-1989})
% and proposed a simpler model for thalamic activity.






