\chapter{Models of Functional Brain Dynamics}

Build a model of functional brain dynamics derived from brain structural
constraints for understanding and predicting fMRI resting state activity from
healthy controls’ and post concussion syndrome (PCS) patients’ DTI data.


Use optimization over novel transformations of DTI collected from healthy
controls and PCS patients to match the subject-specific default mode network
(DMN) from fMRI data

Once optimized and validated, we will compare the optimization’s predicted
functional modes to PET-Tau scans from suspected CTE patients
(Sect.\ref{sec:PET-tau-scan}). The goal is to determine if pathological
functional modes in patient DTI predict CTE.


STEP 1: first work with the population dMRI data (aka diffusion tensor imaging (DTI) data) from
the human connectome project (HCP).
\begin{verbatim}

sMRI data [nifti format]      ---[using Freesurfer]----> 
                             cortical surface + 
                             region mapping


dMRI data [DSI fotmat] (averaged-for-population (first), or individual (better)) 
           ----[MRtrix pipeline, DSI Studio]-->   cortical tractogram networks (tracts)
\end{verbatim}

dMRI data (formerly called DTI, which is indeed just a technique to analyze dMRI
data) can help to predict where in cortex degradation occurs during the course
of disease.



STEP 2: initially focus on using fMRI resting state data from healthy controls
to validate the optimization (see below for optimization explain).


Optimize the extraction DMN functional model from DTI/tractogram
\begin{verbatim}
 cortical tractogram networks +
 cortical surface +
 region mapping
   ----[Laplacian]---> 
         ----[eigen-value-decomposition]----> Harmonics (i.e. a vector)
              -[using fMRI data]--->  DMN (aka fMRI correlation)
              -[using PET data]---->  PET correlation
\end{verbatim}
Harmonics of low values represent slowing changes quantities; high values = fast change quantities.

The resulting optimization techniques will form the basis for target validation
against resting state fMRI from various stages of PCS.

fMRI data from healthy control subjects in the resting-state (rs-fMRI) as
validation targets for optimizing corticocortical tractogram networks derived
from DTI.


GOAL: tractograms from different subjects have high correlation to each other
when optimized for maximal correspondence between one functional mode and the
corresponding patient DMN.


STEP 3: Finally, we will interpret PET-Tau data based on the model (see last
line in the above diagram).
Uses previous optimizations of PCS patient DTI tractograms to analyze PET-Tau
scans (HOW the optimizations can be used for a different data type).

After concussion, you can
\begin{enumerate}
  \item recover quickly
  \item post-concussion syndromes: suicide suceptibility, light sensitive, 
  \item (if repeated cucussions): lead to death
  
  The brain swelling (at 2nd concussion, on next Monday, after the first one on
  Friday), leading to sudden death.
  
\end{enumerate}


How PET data is used
\begin{verbatim}
SUVR Image (using reference: cerebellum grey)
   ---->     Spatial normalization: fMRI co-registered with PET
      -->  smoothed
         ----> ...
\end{verbatim}


\begin{enumerate}
  \item how many cortical surface regions
  
  \item do PET has mapping to the above regions?
  
  \item quality of tracts extracted?
  
  \item 
\end{enumerate}

\section{CTE}
\label{sec:CTE}

Chronic Traumatic Encephalopathy (CTE) is a degenerative brain disease found in
athletes, military veterans, and others with a history of repetitive brain
trauma.
% https://www.gq.com/story/the-concussion-diaries-high-school-football-cte


\section{Default Mode Network (DMN): is it task-negative network?}
\label{sec:Default-Mode-Network}

Though the DMN was originally noticed to be deactivated in certain goal-oriented
tasks and is sometimes referred to as the task-negative network, it can also
appear in task-positive recordings (Sect.\ref{sec:DMN-task-positive-network}).

The DMN has been shown to be negatively correlated with other networks in the
brain such as attention networks.

Evidence has pointed to disruptions in the DMN of people with Alzheimer's and
autism spectrum disorder.


\subsection{What is DMN?}

Hans Berger, the inventor of the electroencephalogram, was the first to propose the idea that the brain is constantly busy. 
\begin{enumerate}
  \item  in 1929 he showed that the electrical oscillations detected by his device do not cease even when the subject is at rest. 

% Raichle, Marcus (March 2010). "The Brain's Dark Energy". Scientific American: 44–49.

  \item In the 1950s, to the surprise of the researchers, Louis Sokoloff and his
  colleagues noticed metabolism in the brain stayed the same when a person went
  from a resting state to performing effortful math problems suggesting active
  metabolism in the brain must also be happening during rest
  

  \item  In the 1970s, Ingvar and colleagues observed blood flow in the front
  part of the brain became the highest when a person is at rest.
  
%Buckner, R. L.; Andrews-Hanna, J. R.; Schacter, D. L. (2008). "The Brain's Default Network: Anatomy, Function, and Relevance to Disease". Annals of the New York Academy of Sciences. 1124 (1): 1–38. doi:10.1196/annals.1440.011. PMID 18400922.
  
Around the same time, intrinsic oscillatory behavior in vertebrate neurons was
observed in cerebellar Purkinje cells, inferior olivary nucleus and thalamus (Llinas, 2014).

%Llinas, R. R. (2014). "Intrinsic electrical properties of mammalian neurons and CNS function: a historical perspective". Front Cell Neurosci. 8: 320. doi:10.3389/fncel.2014.00320. PMC 4219458 Freely accessible. PMID 25408634.

  \item In 1988s, the finding of the relative independence of blood flow and
  oxygen consumption during changes in brain activity which provided the
  physiological basis of fMRI (Peterson et al., 1988).
  
%Petersen, SE; Fox PT; Posner MI; Mintun M; Raichle ME (1988). "Positron emission tomographic studies of the cortical anatomy of single-word processing". Nature. 331 (6157): 585–589. doi:10.1038/331585a0. PMID 3277066.

  \item Raichie and colleagues found that the brain's energy consumption is
  increased by less than 5\% of its baseline energy consumption while performing
  a focused mental task
  
%Raichle, M. E.; MacLeod, A. M.; Snyder, A. Z.; Powers, W. J.; Gusnard, D. A.; Shulman, G. L. (2001-01-16). "A default mode of brain function". Proceedings of the National Academy of Sciences of the United States of America. 98 (2): 676–682. doi:10.1073/pnas.98.2.676. ISSN 0027-8424. PMC 14647 Freely accessible. PMID 11209064.


  

%Kiviniemi, Vesa J.; Kantola, Juha-Heikki; Jauhiainen, Jukka; Hyvärinen, Aapo; Tervonen, Osmo (2003). "Independent component analysis of nondeterministic fMRI signal sources". NeuroImage. 19: 253–260. doi:10.1016/S1053-8119(03)00097-1. PMID 12814576.

\end{enumerate}

A baseline or control state is fundamental to the understanding of most complex
systems. Defining a baseline state in the human brain, arguably our most complex
system, poses a particular challenge.
\begin{enumerate}
  
    \item Raichle suggested that they can find this baseline state for the brain
    activity, and coined the term "default mode" in 2001 to describe resting
    state brain function (Raichle et al., PNAS, 2001)
    
    This is defined in terms of the brain oxygen extraction fraction or OEF.
    
% Raichle et al. A default mode of brain function, PNAS (2001)

%Raichle, Marcus E.; Snyder, Abraham Z. (2007). "A default mode of brain function: A brief history of an evolving idea". NeuroImage. 37 (4): 1083–90. doi:10.1016/j.neuroimage.2007.02.041. PMID 17719799.

In the beginning to mid 2000s, researches labeled the default mode network as the task negative network. 

%Fox, Michael D.; Snyder, Abraham Z.; Vincent, Justin L.; Corbetta, Maurizio; Van Essen, David C.; Raichle, Marcus E. (2005-07-05). "The human brain is intrinsically organized into dynamic, anticorrelated functional networks". Proceedings of the National Academy of Sciences of the United States of America. 102 (27): 9673–9678. doi:10.1073/pnas.0504136102. ISSN 0027-8424. PMC 1157105 Freely accessible. PMID 15976020.

  
\end{enumerate}

\subsection{Task-induced deactivation of brain region}

Researchers have also frequently encountered task-induced decreases in regional
brain activity even when the control state consists of lying quietly with eyes
closed or passively viewing a stimulus.

Interestingly, using PET scans, in 1990s, researchers began to notice that when a person
is involved in perception, language, and attention tasks the same brain areas
become less active compared to passive rest, and labeled these areas as becoming
“deactivated”.

The fact that many brain areas decreases appear to be largely task independent,
varying little in their location across a wide range of tasks. This may be the result of 'lacking' oxygen or energy supply,
as the result of these energy substrates are being drawn in to be used by the brain regions that become


\subsection{How DMN is represented?}

A baseline state of the normal adult human brain in terms of the {\bf brain oxygen
extraction fraction} or OEF. The OEF is defined as the ratio of oxygen used by
the brain to oxygen delivered by flowing blood and is remarkably uniform in the
awake but resting state (e.g., lying quietly with eyes closed).

OEF is extracted regionally throughout the brain, based on quantitative
metabolic and circulatory measurements from positron-emission tomography (Sect.\ref{sec:PET}).


The default mode network is most commonly defined with resting state data by
putting a seed in the posterior cingulate cortex and examining which other brain
areas most correlate with this area.





\subsection{Why DMN?}

The default mode network has been hypothesized to be relevant to disorders
including Alzheimer's disease, autism, schizophrenia, depression, chronic pain,
posttraumatic stress disorder and others.

If the default mode network is altered, this can change the way one perceives
events and their social and moral reasoning, thus making a person more
susceptible to major depressive-like symptoms.


%Buckner, R. L.; Andrews-Hanna, J. R.; Schacter, D. L. (2008). "The Brain's Default Network: Anatomy, Function, and Relevance to Disease". Annals of the New York Academy of Sciences. 1124 (1): 1–38. doi:10.1196/annals.1440.011. PMID 18400922.

%Akiki, Teddy J.; Averill, Christopher L.; Wrocklage, Kristen M.; Scott, J. Cobb; Averill, Lynnette A.; Schweinsburg, Brian; Alexander-Bloch, Aaron; Martini, Brenda; Southwick, Steven M.; Krystal, John H.; Abdallah, Chadi G. (2018). "Default mode network abnormalities in posttraumatic stress disorder: A novel network-restricted topology approach". NeuroImage. 176: 489–498. doi:10.1016/j.neuroimage.2018.05.005. ISSN 1053-8119.


People with Alzheimer's disease show a reduction in glucose (energy use) within
the areas of the default mode network. These reductions start off as slight
decreases in mild patients and continue to large reductions in severe patients.
Disruptions in the DMN begin even before individuals show signs of Alzheimer's disease

DMN is thought to be disrupted in individuals with autism spectrum disorder.
Studies have shown worse connections between areas of the DMN in individuals
with autism, especially between the mPFC (involved in thinking about the self
and others) and the PCC (the central core of the DMN). 
The more severe the autism, the less connected these areas are to each other.
It is, however, not clear if this is a cause or a result of autism.

%Washington, Stuart D.; Gordon, Evan M.; Brar, Jasmit; Warburton, Samantha; Sawyer, Alice T.; Wolfe, Amanda; Mease-Ference, Erin R.; Girton, Laura; Hailu, Ayichew (2014-04-01). "Dysmaturation of the default mode network in autism". Human Brain Mapping. 35 (4): 1284–1296. doi:10.1002/hbm.22252. ISSN 1097-0193. PMC 3651798 Freely accessible. PMID 23334984.

%Yerys, Benjamin E.; Gordon, Evan M.; Abrams, Danielle N.; Satterthwaite, Theodore D.; Weinblatt, Rachel; Jankowski, Kathryn F.; Strang, John; Kenworthy, Lauren; Gaillard, William D. (2015-01-01). "Default mode network segregation and social deficits in autism spectrum disorder: Evidence from non-medicated children". NeuroImage: Clinical. 9: 223–232. doi:10.1016/j.nicl.2015.07.018. PMC 4573091 Freely accessible. PMID 26484047.

Lower connectivity between brain regions was found across the default network in
people who have experienced long term trauma, such as childhood abuse or
neglect, and is associated with dysfunctional attachment patterns.


Among people experiencing posttraumatic stress disorder, lower activation was
found in the posterior cingulate gyrus compared to controls, and severe PTSD was
characterized by lower connectivity within the DMN.


Hyperconnectivity of the default network has been linked to rumination in first-episode depression,
and chronic pain.

%Zhu, X; Wang, X; Xiao, J; Liao, J; Zhong, M; Wang, W; Yao, S (2012). "Evidence of a dissociation pattern in resting-state default mode network connectivity in first-episode, treatment-naive major depression patients". Biological Psychiatry. 71 (7): 611–7. doi:10.1016/j.biopsych.2011.10.035. PMID 22177602.






\subsection{task-positive network}
\label{sec:DMN-task-positive-network}

The DMN (Sect.\ref{sec:Default-Mode-Network}) can be active in other
goal-oriented tasks such as social working memory or autobiographical tasks




\section{DTI (Diffusion Tensor Imaging)}
\label{sec:DTI}

