\chapter{AMPAR Model}
\label{chap:AMPAR-models}

\def\decay{{\text{decay}}}
\def\rise{{\text{rise}}}
\def\AMPA{{\text{AMPA}}}
\def\post{{\text{post}}}


AMPAR (Sect.\ref{sec:AMPAR}) is a non-NMDA glutamate receptor
(Sect.\ref{sec:glutamate_receptor}) that is specifically activated by AMPA. Upon
binding of AMPA, such molecules can produce a desensitizing response of AMPAR in
mouse hippocampal neurons (Sect.\ref{sec:desensitizing-response}). 

Another member of non-NMDA receptor is kainate-receptor
(Sect.\ref{sec:kainate_receptor}); though AMPAR and kainate-receptor have
different relative affinity to AMPA and kainate.

\section{Introduction}
\label{sec:AMPAR-kinetics}

The  AMPA  synapses  can  be  very  fast.  For  example  in  some  auditory 
nuclei,  they  have  sub-millisecond  rise  and  decay  times.  In  typical 
cortical  cells,  the  rise  time is about 0.4 ms to 0.8ms. 

AMPA receptors onto inhibitory interneurons are about twice as fast in rise and
fall times.

\subsection{AMPAR conductance}
\label{sec:AMPAR-conductance}


AMPA receptors (Sect.\ref{sec:AMPA/Kainate-receptor}) mediate EPSCs in most CNS
neurons, so the single-channel properties of these receptors can affect
information processing in neurons by influencing the amplitude and kinetics of
synaptic currents.

\begin{enumerate}
  
  \item Both AMPA and NMDA receptors, however, have an equilibrium potential near 0 mV
  
  \item The principal ions gated by AMPARs are sodium and potassium (i.e.
  GluR2-expressed AMPAR), distinguishing AMPARs from NMDA receptors (the other
  main ionotropic glutamate receptors in the brain), which also permit calcium influx.
  
  \item   apparent unitary conductances of <1 pS in patches from both migrating
  and mature granule cells in acute cerebellar slices (Smith et al., 2000)
  
NOTE: Granule cells begin to express AMPA receptors before they arrive in the
internal granule cell layer and receive synaptic input. 

They found individual AMPAR exhibits as many as 4 distinguishable conductance
levels, i.e. suggesting the granule cells express a heterogeneous
population of AMPA receptors, with average is unitary conductances
estimated previously for synaptic AMPA receptors
%   Heterogeneous Conductance Levels of Native AMPA Receptors
% T. Caitlin Smith,1 Lu-Yang Wang,2 and James R. Howe1,2

NOTE: The AMPAR is a tetramer with 4 subunits, i.e. 4 ligand binding sites.
The receptor opens when two sites are occupied, and increases its current as
more binding sites are occupied. This may explain the multi-conductance level
observed.

  \item estimates for Schaffer collateral-commissural fiber synapses onto
  hippocampal CA1 pyramidal cells range from 1.5 to 22.3 pS (Benke et al., 1998).
\end{enumerate}

AMPA and NMDA receptors are evenly distributed in the dendritic membrane before
synaptogenesis with an estimated density of 3 receptors/$\mu^2$
\begin{itemize}
  
  \item Following synaptogenesis, in hippocampal neuron, functional AMPA and
  NMDA receptors are clustered at synapses with a density estimated to be on the
  order of $10^4$ receptors/$\mum^2$, which corresponds to about 400
  receptors/synapse (Cottrell et al., 2000).
  
 It means about 400 pS  
\end{itemize}


\subsection{Experimental studies}

To study the kinetics of AMPAR, photolysis of caged glutamate is used
(Sect.\ref{sec:two-photon-uncaging}).

\begin{enumerate}
  \item current response after 0.3$\pm$ 0.1 ms after flashing the light to
  uncage glutamate.
  
  The inward is composed of 2 components: one through AMPAR (which is blocked
  by 5$\muM$ CNQX) and one through NMDAR (which is blocked by 50$\muM$ APV)
  
  \item amplitude of inward current depends upon caged glutamate concentration
  and light energy.
  
  \item peak current achieved at 2-9ms from the onset of uncaging
  
  \item Once open, the channel may undergo rapid desensitization, stopping the
  current. The channel  close quickly (1ms).
\end{enumerate}

A recent in vivo study with 2P uncaging showed structure-functional relationship
in pyramidal cell of adult mice neocortex \citep{noguchi2011}.
\begin{itemize}
  \item  functional AMPAR expression was stable andproportional
to spine volume

  \item glutamate-induced Ca2+ transients were inversely proportional to spine
  volume
\end{itemize}
%No current was generated when the spot was moved 5-10 mm away from a cell



\section{Two-state}

%1msec pulse of 1mM 

The gating of AMPAR is modeled using 
\begin{equation}
\ce{ C <=>[\alpha.\text{[NT]}][\beta] O}
\end{equation}
with [NT] is concentration of neuro-transmitters, which can also be represented
as the pre-$\Vm$ as an alternate option.

Suppose $r$ is the fraction of AMPAR in the current synapse in the open-state,
then
\begin{equation}
\frac{dr}{dt} = \alpha \times [NT] \times (1-r) - r \times \beta
\end{equation}
with the function for [NT] is given Sect.\ref{sec:neurotransmitter-time-course}.

and AMPAR current is
\begin{equation}
I_\AMPA = \bar{g_\AMPA} \times r \times (\Vm_\post - E_{\rev,\AMPA})
\end{equation}

The time course of $r(t)$ is important to model the current that appears on the
postsynaptic side. 

\section{Destexhe, Mainen, Sejnowski (1994)}


\section{Hausser, Roth (1997)}


\section{Koike, Ozawa (2000)}
