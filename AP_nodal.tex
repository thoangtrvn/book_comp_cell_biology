
\chapter{Models AP in nodal cells}
\label{chap:models-ap-nodal}

Automatic excitation is one of the characteristic feature of pacemaker cells,
e.g. SA nodes. This spontaneous beating activity is triggered by a slow
diastolic repolarization \citep{draper1951, trautwein1952, west1955}. As $\K$
current is the one that maintain the resting potential, based on the models
from Purkinje fibers, it was first hypothesized that a spontaneous gradual decay
in this current causes the depolarization. However, the reconstructed K current
from SA node is much smaller than $i_{K2}$ in Purkinje fiber. Also, the
automatic excitation still occurs even when blocking time-dependent K current
using $\Ba$ ions \citep{Yanagihara1980}. Thus, it was hypothesized that the slow
inward current in the SA node control the automatic excitation. 

The SA node is not homogeneous. In rabbit, the SA node is of size about 8 mm
$\times$ 10 mm \citep{Boyett1999}. In the vertical direction it is bounded by
the superior and inferior venae cavae, and in the horizontal direction it is
bounded by the crista terminalis (a thick bundle of atrial muscle) and the
interatrial septum \citep{zhang2000}. The AP is initiated at a small part of the
SA node, the so-called leading pacemaker site which is referred to as the center
of the SA node. Normally, the leading pacemaker site is approximately midway
between the two venae cavae and 1-2 mm from the crista terminalis
\citep{Bleeker1980}.
From the center of the SA node, the Ap propagates to the periphery of the SA
node and then onto the atrial muscle of the crista terminalis. The peripheral of
the SA node is often called the {\it perinode} or transitional tissue. Even
though the major function of the peripheral of the SA node is to conduct the AP
from the leading pacemaker site to the atrial cell, the cells in the peripheral
of the SA node also show pacemaker activity.

There are differences between the peripheral region and the ring bundle
(diastolic membrane potential near -75 mV) vs. central region (diastolic
membrane potential only in the range -55 to -65 mV) \citep{noble1984msa}. The
parameters in the former model is adjusted to give the less negative diastolic
potential in the former one by using one of the following strategies
\begin{enumerate}
  \item reduce $i_\k$
  \item increase background inward current $g_{b,\na}$
  \item change the parameters determining $i_{\na\ca}$.
\end{enumerate}

Even though Purkinje fiber also has automaticity, the striking difference is
that Purkinje fiber is extremely sensitive to [K]o, but it's fairly insensitive
for SA node \citep{noble1984msa}. To explain for this, [K]o strongly
affect $i_{K1}$ which is relatively unimportant in SA node, but is very
significant in Purkinje fiber. 

\citep{Wilders1991} demonstrated that the early models
(Sect.\ref{sec:SA_node-Yanagihara1980},
Sect.\ref{sec:SA_node_Bristow1982}, Sect.\ref{sec:SA_node_Noble1984}) have
many drawbacks.

\section{Ion concentrations}

\begin{enumerate}
  \item NOTE: [K]i = 140 mM, [Na]i = 7.5 mM, [K]o = 3 mM, [Ca]o = 2 mM
  \citep{noble1984msa}
\end{enumerate}

\section{Yanagihara-Noma-Irisawa (rabbit - 1980)}
\label{sec:SA_node-Yanagihara1980}

\citep{Yanagihara1980} developed the model with spontaneous AP that includes 4
ionic currents using Hodgkin-Huxley-based equations
(Sect.\ref{sec:HH_equations}): $i_\na$ (Na current), $i_s$ (slow-inward
current), $i_\k$ (K current), $i_h$ (delayed-inward hyperpolarization-activated
current), and $i_1$ (time-independent leak current). \textcolor{red}{This is
simple as there is no exchanger or ionic pumps}.

The membrane circuit is formulated based on a unit membrane ($\Cm = 1 \muF$)
\begin{equation}
\Cm \frac{dV_m}{dt} = i_m - (i_s + i_\na + i_k + i_h + i_1)
\end{equation} 
with $i_m$ is the total current passing through the unit membrane. 

NOTE: Unit rate constant (1/(msec)) and voltage (mV).

\subsection{Ionic currents}

$i_\na$ (sodium current)
\begin{equation}
i_\na = m^3 . h . \bar{i}_\na
\end{equation}


\subsection{Numerical solution}

The ODEs were solved using Runge-Kutta 4-th order of approximation on digital
computer Nicolet (model NIC-80). The time step $dt=1$ (msec) for $i_\na$.

\subsection{Data input}

The initial resting membrane potential is the steady-state value -60 mV. The
values of the parameters given after 2 second of simulation is used as the
initial inputs.


\subsection{Data analysis}

The model suggested a large contribution from $i_s $ and $i_1$, yet they are
difficult to quantify experimentally. Thus, it was not enable to confirm the
model's result.

\section{Bristow-Clark (1982)}
\label{sec:SA_node_Bristow1982}

\citep{bristow1982mmp}

\section{Noble-Noble model (SA node - 1984)}
\label{sec:SA_node_Noble1984}

~\citep{noble1984msa} modified the equations of the model developed for
Purkinjie fiber (Sect.\ref{sec:difr-noble-purk}). 
\begin{enumerate}
  \item $g_\to = 0$: it has not been found in SA node
  \item $g_{b,\na} = 0.07 \muS$: so that maximum diastolic potential
  between -70 mV and -50 mV during pacemaker activity
  \item $g_{b, \ca} = 0.01 \muS$:  so that the free calcium in the range
  0.5$\muM$.
  \item $g_{K1} = 10 \muS$: which is greatly reduced compared to the value used
  in Purkinje fiber 900 $\muS$ \citep{difrancesco1985mcea}.
\end{enumerate}

Compared to the previous model (Sect.\ref{sec:SA_node-Yanagihara1980}), not only
ionic currents, but also Na/K and Na/Ca exchanger are added. Nomenclature:
\begin{itemize}
  \item 'total second inward current': current generated by both $V_m$-gated
  $\Ca$ channel  and $\Ca$-dependent inward currents (e.g. NCX).
  \item 'gated Ca channel' : $V_m$-gated $\Ca$ channel only
\end{itemize}

Total membrane capacitance $\Cm = 0.006 \muF$. The specific membrane capacitance
is $\Csc = 1 \muF/\cm^2$.





\section{Hagiwara-Irisawa-Kameyama model (1988)}
\label{sec:hagiw-iris-kamey}

Hagiwara et al. \citep{hagiwara1988ctt} use the same approach as HH
model. In particular, they also use a fast voltage-dependent gating
variable $d_L$ and a slow voltage-dependent gating inactivating
variable $f_L$, now adding a new variable $\gamma_{Ca,L}$ which was
sensitive to {\it extracellular} calcium concentration.

\begin{equation}
  \label{eq:387}
  \gamma_{Ca,L} = \frac{\gamma_{Ca,L,max}}{1+(K_m/[\ce{Ca^2+}]_o)}
\end{equation}
with $\gamma_{Ca,L}$ is the maximum current that flows.


\section{Dokos et al. (19993}
\label{sec:SA_node_Dokos1994}

\citep{Dokos1993}


\section{Demir et al. (rabbit - 1994)}
\label{sec:SA_node_Demir1994}

\citep{Demir1994}
\begin{enumerate}
  \item introduced $i_\na$: based on experimental data from Colatsky (rabbit
  Purkinje fiber) with Hodgkin-Huxley based formula ($m^3h$)
\end{enumerate}


\section{Zhang et al. (rabbit - 2000)}
\label{sec:SA_node_Zhang2000}


\citep{zhang2000} changed
\begin{enumerate}
  \item $i_\na$: the data for SA node cells showed that the recovery from
  inactivation can be fit better using two exponentials. So, there are three
  gating variables $m, h, j$.
\end{enumerate}


\subsection{Tissue model}

A one-dimensional (1D) cable was developed as a string of tissue of length (L)
= 12.6 mm. Here, the length of the SA node is $L_s$ = 3 mm (the same as the
distance from the center of the SA node to the atrial muscle in the rabbit), and
the string of atrial tissue is of length 9.6 mm. 


%%% Local Variables: 
%%% mode: latex
%%% TeX-master: "mainfile"
%%% End: 
