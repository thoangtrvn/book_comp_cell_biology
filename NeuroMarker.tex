\chapter{Neuro Marker in diseases}
\label{chap:neuro-marker}




\section{Neuro Markers}

\subsection{Glu}

Glu is a ubiquitous molecule used in cellular metabolism and is the principal
excitatory neurotransmitter (Sect.\ref{sec:Glutamate}).

\subsection{NAA: N-Acetylaspartic acid}
\label{sec:NAA}

NAA (N-Acetylaspartic acid) - a derivative of aspartic acid - is an indicator of
neuronal integrity, with decreases suggesting neuronal dysfunction. 

NAA is the second-most-concentrated molecule in the brain after the amino acid
glutamate. It is synthesized in the mitochondria from the amino acid aspartic
acid and acetyl-coenzyme (Sect.\ref{sec:CoA}). It has also been demonstrated
that high NAA level in hippocampus is related to better working memory
performance in humans. NAA may function as a neurotransmitter in the brain by
acting on metabotropic glutamate receptors (Sect.\ref{sec:mGluR}).

\subsection{tCr (total Creatine)}
\label{sec:tCr}

The signal from tCr, with contributions from creatine and phosphocreatine,
represents the high-energy biochemical reserves of neurons and glia. 

\subsection{Cho (Choline-containing compounds)}
\label{sec:Cho}

The signal from Cho, including contributions from free choline,
glycerophosphorylcholine and phosphorylcholine, is a marker for cell
membrane synthesis and turnover. 

\section{Immune System}
\label{sec:immune-system}

\begin{enumerate}
  \item  Neutrophils are the body's first line of defense and the main cell protecting us from bacterial infections 
  (Sect.\ref{sec:neutrophils})
\end{enumerate}

\subsection{Neutrophils}
\label{sec:neutrophils}

Neutrophils are the most abundant type of granulocytes and the most abundant
type of white blood cells in most mammals. They form an essential part of the
innate immune system.

While their protective function is very positive, neutrophils also have
inflammation-producing properties (Sect.\ref{sec:inflammation-neuro}) that cause
problems in heart disease and a host of autoimmune diseases, for example lupus.
This makes understanding how to manipulate these cells extremely important in
disrupting disease.

Neutrophils move around the body through the blood stream to fight infections,
but in order to do this they must travel through the blood vessel walls of sites
of inflammation, infection, or injury.
When there is a site of inflammation in the body, the blood flow increases because the blood transports immune cells to the site to promote healing.

\textcolor{red}{\bf How neutrophils attach to the blood vessel wall?}:
A study published in Nature (2012) by Dr. Klaus Ley
\footnote{\url{https://www.medicalnewstoday.com/articles/247459.php}}
showed that when blood flow is extremely fast, neutrophils attach on to the
blood vessel wall using sling-like membrane tethers. They observed using
a novel imaging technique they developed in 2010 in order to see and photograph
the neutrophil adhesion process.
The cells separate their cytoskeleton from the cellular membrane, wrapping the sling around themselves like a lasso and then digging their hooks into the blood vessel wall



\textcolor{red}{\bf How neutrophils, once attached to the blood vessel wall, migrate out of the blood vessel?}:

 
 
\section{Inflammation in diseases}
\label{sec:inflammation-in-diseases}

Direct evidence for an innate inflammatory response in Alzheimer's disease (AD)
was described nearly 20 years ago (reviewed in Akiyama, 1994) and subsequent
studies have documented inflammatory components in Parkinson's disease (PD),
amyotrophic lateral sclerosis (ALS), multiple sclerosis (MS -
Sect.\ref{sec:MS-multiple-sclerosis}), and a growing number of other nervous
system pathologies.



\begin{mdframed}

Inflammation happens when the immune system fights against something that may
turn out to be harmful..

Inflammation is the normal response of your body's immune system
(Sect.\ref{sec:immune-system}) to injuries and harmful things that enter your
body. Immune cells quickly react to the damaged area to fix the problem. During
the process, you may feel symptoms like pain, warmth, swelling and redness.

Inflammation can occur at different body's organs/parts, e.g. respiratory tracts, in brain.

Although inflammation is a normal part of the healing process, in certain diseases it is undesired. 

\end{mdframed}

What are the roles of the innate and adaptive immune systems in diverse forms of
neurodegenerative disease?


Although inflammation may not typically represent an initiating factor in
neurodegenerative disease, there is emerging evidence in animal models that
sustained inflammatory responses involving microglia and astrocytes contribute
to disease progression.

 Inflammatory responses that establish feed-forward loops may overwhelm normal
 resolution mechanisms. Although some inflammatory stimuli induce beneficial
 effects (e.g., phagocytosis of debris and apoptotic cells), and inflammation is
 linked to tissue repair processes, uncontrolled inflammation may result in
 production of neurotoxic factors that amplify underlying disease states.

\textcolor{red}{A  major unresolved question is whether inhibition of these responses 
will be a safe and effective means of reversing or slowing the course of disease}

\subsection{Mitochondrial inflammation}
\label{sec:inflammation-mitochondria}

Defective mitochondria (Sect.\ref{sec:mitophagy}) might also release, besides
reactive oxygen species (ROS), components that are not normally present in the
cytoplasm, such as mitochondrial DNA.
Indeed, the intrusion of mitochondrial DNA into the cytoplasm can trigger
inflammation mediated by the protein STING.

This raises the question of whether protection from inflammation, rather than
from oxidative damage, might be the key role of mitophagy in the context of
Parkinson’s disease.



\subsection{Microglial inflammation}
\label{sec:inflammation-microglial}

Microglia (Sect.\ref{sec:microglial-cell}) are the major resident immune cells
in the brain, where they constantly survey the microenvironment and produce
factors that influence surrounding astrocytes (another type of glial cell with
support functions) and neurons.

Under physiological conditions, microglia exhibit a deactivated phenotype that
is associated with the production of anti-inflammatory and neurotrophic factors
(Streit, 2002). Microglia switch to an activated phenotype in response to
pathogen invasion or tissue damage and thereby promote an inflammatory response
that serves to further engage the immune system and initiate tissue repair. In
most cases, this response is self-limiting, resolving once infection has been
eradicated or the tissue damage has been repaired.

A persistent stimulus may result from environmental factors or the formation of
endogenous factors (e.g., protein aggregates) that are perceived by the immune
system as “stranger” or “danger” signals.




\subsection{Neuroinflammation}
\label{sec:inflammation-neuro}



Neuroinflammation is associated with elevated levels of myo-inositol (mI -
Sect.\ref{sec:myo-inositol}), choline-containing compounds (Cho) and total
creatine (tCr) - Zahr et al. (2014).



\section{Neuronal injury}

neuronal injury (dysfunction or loss) is associated with low levels of N-acetyl
aspartate (NAA) and glutamate (Glu)

