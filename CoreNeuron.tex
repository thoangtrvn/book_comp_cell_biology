\chapter{Blue Brain project: CoreNeuron - STEPS simulator}
\label{chap:CoreNeuron}

Blue Brain project aimed to provide a simulation-based integration of
heterogeneous data into a unified model in 2005. The different components and
their interactions are modeled according to theire presence in a biological
system rather than based on some hypothesis, and the resulting emergent
properties are studied.

They developed a program named {\bf CoreNeuron} - extended based on {\bf NEURON}
software (Sect.\ref{sec:NEURON}) based on optimized data structure (reduce
memory footprint), maximize all hardware threads, and vectorization using
auto-vectorization techniques.

In 2013, BBP at EPFL joined Human Brain Project (humanbrainproject.eu).
Recently, there is an effort to couple STEPS simulator and CoreNeuron to build
the first efficiently multiscale computational neuroscience model

\section{CoreNeuron}
\label{sec:CoreNeuron}

CoreNeuron is developed based on NEURON, which has been demonstrated to scale
up to 64K core counts on BlueGene/P. They simulated neocortical microcircuit
(about 200K morphologically detailed neurons) was done using Blue Brain IV
supercomputer, called JUQUEEN of 28 racks of BlueGene/Q (Feb, 2015), and using
ALCF (Argonne Leadership Computing Facility) of 48 racks of BlueGene/Q (May,
2015).


CoreNeuron extracted the core functionality of NEURON in about 15,000 lines of
C/C++ code, and then parallelized at 3 levels
\begin{enumerate}
  \item one MPI process per node:
    that handles a set of neuron groups.
  
  Neurons are organized into groups so that each group has equivalent
  computational cost, and then equi-distributed to each node using {\bf Least
  Processing Time} algorithm.
  
  \item OpenMP threads:
  
  Each neuron group is assigned to a particular OpenMP thread to ensure data
  locality, giving thread load imbalance under 2\%.
  
  \item Vectorization: All channels of the same type of all neurons in the same
  group is solved at the same time using vectorization; before moving to the
  next type.
  
  This maximize cache efficiency and vectorization.
  
\end{enumerate}

{\bf A single thread per node is responsible for}
initiating synchronization of neurons - through exchange of spike events.
The synchronization is done using MPI {\bf Allgather} and {\bf AllgatherV}, i.e. 
every neuron group receives spiking information from all other neuron groups.
\begin{itemize}
  \item PROS: avoid explicit storage of memory to store information about
  network topology
  
  
  \item CONS: extra work to filder out which group really connect to other group
  to get the spike events
  
  So it relies on efficient implementation of the MPI collective Allgather on
  supercomputer.
\end{itemize} 

\subsection{In memory compression: BLOSC}

BLOSC open source package.

\section{Distributed STEPS simulator}
\label{sec:STEPS-simulator}

STEPS simulator is used to model subcellular scale of neuronal network
electrical activity.
STEPS has a stochastic simulator of reaction-diffusion system which uses
Gillespie stochastic formalism to model population of chemical species across
reactions and diffusion events.

The first MPI-based STEP simulato was showed to scale up to a few hundreds core.
The next step is to build the distributed version of STEPS simulator that is
expected to scale up to thousands of cores on IBM BlueGene/Q.

\section{CoreNeuron + STEPS}

USER CASE 1: A part of the dendrite per neuron is modeled using STEPS, while
the rest of the neuron (channel, other synapses, etc.) is modeled using CoreNeuron. The coupling
takes place at the level of CoreNeuron compartments.


USER CASE 2: A single full neuron is modeled using STEPS, embedded in a circuit
where all other neurons are represented using CoreNeuron cellular
representation. The coupling takes place at the time of spike exchange between
the neurons of the considered circuit.


The simulation of inhomogeneous biological system is often dominated by the cost
of modeling the diffusion process, and these processes can be modeled
independently by an inexact algorithm in a procedure called {\bf operator
splitting.}

Reaction process at different subvolume can be treated independently, i.e. can
be solved in parallel, and is being coupled by an inexact diffusion model above.



