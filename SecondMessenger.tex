%%
%% SecondMessenger.tex
%% Login : <hoang-trong@hoang-trong-laptop>
%% Started on  Sun May 31 11:21:39 2009 Hoang-Trong Minh Tuan
%% $Id$
%% 
%% Copyright (C) 2009 Hoang-Trong Minh Tuan

% \chapter{Second Messenger System}
\label{chap:second-mess-syst}

Signal transduction is just the manner by which cells transmit chemical
signals that initiate a plethora of biochemical and cellular events.

{\bf Why cells communicate}?: There are 3 basic types of intercellular
communications
\begin{itemize}
  \item coordinate of movement
  \item coordinate of metabolism
  \item coordinate of growth
\end{itemize}


However, most signal molecules are too large and too polar to pass through the
membrane, and no appropriate transport systems are present. Thus, the
information that signal molecules are present must be transmitted across the
cell membrane without the molecules themselves entering the cell.
This requires membrane-associated receptors (Chap.\ref{chap:neuron-receptors}).

The extracellular signal molecules are called {\bf first messengers} and
functions as ligands to these membrane-associated receptors. Upon first
messenger binding, it induces the conformation changes of these receptors - 
tertiary or quaternary structure of the receptor, including the intracellular domain.

However, these structural changes are not sufficient to yield an appropriate
response, because they are restricted to a small number of receptor molecules in
the cell membrane. For such signals, in the form of ligand-binding to some
membrane-associated receptors, to induce further change to the intracellular of
the target cell, there is a need for {\bf second messengers}.



% DNA is considered as
% the first messenger agent as it encodes genetic information.

Second messengers are molecules whose functions are to relay information from
the receptor-ligand complex above (i.e. transmit extracellular stimulus or
signals into the cell or amplify it).
It helps regulates internal cellular events, e.g. cell growth, apoptosis, cell
migration, endocytosis, and cell differentiation, etc.

% The signaling pathways for these second messengers are called second
% messenger systems. Specifically, under a stimulus, a diffusible
% signalling molecule is rapidly produced/secreted from the biomembrane,
% which can then go on to activate {\it effector} protein within the
% cell to exert a cellular response (e.g. exocytosis, contraction,
% metabolism, transcription...). 

All secondary messenger molecules can be divided into three types:
\begin{itemize}

   \item {\it Hydrophobic molecules}: those not mixing with water, so can
   diffuse through the membrane.
   
   They are: PIP3 (Sect.\ref{sec:PIP3}),  diacylglycerol (Sect.\ref{sec:DAG}),
   phosphatidylinositol (Sect.\ref{sec:phosphatidylinositol})...

   \item {\it Hydrophilic molecules}: those are water soluble  
   
   They are: cAMP (Sect.\ref{sec:cAMP}), cGMP (Sect.\ref{sec:cGMP}), IP3
   (Sect.\ref{sec:IP3}), \ce{Ca^2+}, \ce{K+}... locate within the cytosol.

    \item {\it Gases}: can diffuse both through cytosol and across cellular
  membranes.
    
    They are: nitric oxide (\ce{NO}), carbon monoxide
  (\ce{CO})... 
  
\end{itemize}

In this part of the book, we will cover important ions (hydrophilic
molecules) that serve as second messengers in excitable cells,
i.e. neurons and myocytes. 
\begin{enumerate}
\item cAMP in Chap.~\ref{chap:camp}
\item \ce{Ca^2+} in Chap.~\ref{chap:calcium-signalling}
\item $\beta$-adrenergic in Chap.~\ref{chap:beta-adren-sign}. 
\end{enumerate}

\begin{mdframed}
Signal transduction by passive diffusion (i.e. no direct input of ATP energy) of
signaling proteins. 

BUT, it does not mean they follow uncordinated random walks, yet it's like a
elegantly choreographed ballet. This diffusional dance is controled by 'time'.
Cells utilize several biophysical mechanisms to facilitate this flow of
information.

They call it 'cheap trick' which has 3
\begin{enumerate}
  \item Local concentration effect (aka reduction of dimensionality):
     
%   only a small fraction of a cell need be covered by receptors to effectively
%capture a ligand

  \item two domains are better than one
  
  \item electrostatic sequestration 
\end{enumerate}
\citep{McLaughlin2002}
\end{mdframed}


%%% Local Variables: 
%%% mode: latex
%%% TeX-master: "thermo-stat"
%%% End: 
