\chapter{Brain and Energy}




\section{CSF}
\label{sec:CSF}


The ventricular systems (or ventricles) is the space inside the brain that
produces CSF (Sect.\ref{sec:ventricle_brain}).


CSF is a clear fluid that fills the ventricular system of the central nervous
system (CNS) (inner CSF) and surrounding the brain and spinal cord in the
cisternae and the subarachnoid space (outer CSF). Proteomic data from human
cerebrospinal fluid (CSF) and brain extracellular fluid (ECF) -
Sect.\ref{sec:ECF}, mostly obtained by cerebral microdialysis (Maurer et al.,
2010).

In the adult human, the total CSF volume is about 135 mL, of which 0.35 mL is
replaced every minute (Lentner, 1977).
Thus, the complete CSF volume is exchanged 3-4 times each day (Maurer, 2008).

CSF acts as a buffer, providing basic mechanical and immunological protection to
the brain inside the skull.

Cells of the choroid plexus and ependymal cells that line the ventricular walls
secrete CSF, and the exchange of molecules with blood plasma and nearby brain
tissue contributes to CSF's final composition.

the protein concentration of CSF is about 400 times less than in human blood
serum (Sect.\ref{sec:protein-components-CSF}).

\section{ECF}
\label{sec:ECF}

The extracellular fluid (ECF) of the brain is of central
importance to the supply of nutrients for growth for the circulation
of hormones.


Extracellular fluid (ECF) is the non-cellular components of the brains.
Besides the cellular components of the brain (i.e., neurons, astrocytes,
oligodendrocytes, and microglia, but also endothelium, ependyma, and tanicytes),
the extracellular space constitutes about 20\% of the total brain volume
(Ungerstedt, 1991).

The extracellular space in mainly filled with brain ECF, which is composed of
ions, neurotransmitters, amino acids, peptides and proteins, gases (mainly O2,
CO2, NO, CO), metabolites, and waste products.

The ECF mediates an extensive chemical trafficking among the brain cells, mostly
to secure oxygen and glucose supply from the blood to the brain.

Main chemical components are the monoamines dopamine, 5-hydroxytryptamine
(serotonine), dihydroxyphenylacetic acid (DOPAC), homovanillic acid (HVA),
5-hydroxyindole acetic acid (5-HIAA); ascorbic acid, uric acid; amino acids
(aspartate, glutamate, arginine, glycine, taurine, alanine, GABA, phenylalanine,
methionine, valine, isoleucine, and leucine); purines (adenosine, inosine,
hypoxanthine, and guanosine); and substance P.



\section{Blood-Brain barrier}


Drugs such as antibody are delivered to the organs via the blood vessel.
However, in the brain, only small molecules are allowed to diffused across the
blood vessel, which prevent potential drugs to do its job inside the brain.
The blood-brain barrier (BBB) is a highly selective membrane system that
prevents unwanted passage of molecules into the brain extracellular fluid.

Roche developed the so-called {blood-brain shuttle} which can help to
transport such antibody across the blood vessel into the brain.
The brain shuttle is indeed a special molecule, that can bind to the drug, and
then both will diffuse across the blood vessel.
\url{https://www.youtube.com/watch?v=2JiUPbDtGuQ}

