\chapter{Astrocytes}
\label{sec:astrocyte-modeling}

%http://www.ncbi.nlm.nih.gov/pmc/articles/PMC4738265/

Unlike neurons, astrocytes are not electrically excitable and do not fire or
propagate action potentials (APs) along their processes (Smith, 1992). However,
astrocytes are thought to {\it display excitability in the form of intracellular
calcium concentration increases} (Sect.\ref{sec:astrocyte-calcium-elevation})
that have been postulated to have a responsive, instructive and/or regulatory
role within neuronal networks.

Mature hippocampal astrocytes exhibit a membrane K+ conductance characterized by
a linear current-to-voltage (I-V) relationship (passive conductance), and a
highly negative membrane potential (Vm).


Neonatal astrocytes exhibit a more negative resting membrane potential (Vm), -85
mV, than mature astrocytes, -80 mV and a variably rectifying whole-cell current
profile due to complex. While passive behavior and low membrane resistance (Rm)
represent intrinsic properties of membrane ion channels (Du et al., 2015), the
molecular identity of K+ channels underlying this unique membrane conductance
remains an issue to be fully resolved.

Potential candidates:

\begin{itemize}
  \item  The inwardly rectifying K+ channel Kir4.1
(Sect.\ref{sec:Kir4.1}) 

The functional analysis indicated that 48\% of passive conductance
is mediated by Kir4.1 channels (Ma et al., 2014).

  \item [wrong] two-pore domain K+ channels (K2P) TWIK-1(K2P 1.1 -
Sect.\ref{sec:TWIK-1-channel}) and TREK-1(K2P 2.1 -
Sect.\ref{sec:TREK-channel}): 

TWIK-1 gene knockout affected the passive conductance minimally, as the major of
TWIK-1 are in intracellular compartments; and the behavior of this channel as a
non-selective cation channel in the membrane (Wang et al., 2013, 2015).

The basic electrophysiological properties of mature hippocampal astrocytes were
not altered in either TREK-1 single or TWIK-1/TREK-1 double gene knockout mice
(Du et al., 2016)
%http://www.ncbi.nlm.nih.gov/pmc/articles/PMC4738265/
%Genetic Deletion of TREK-1 or TWIK-1/TREK-1 Potassium Channels does not Alter
% the Basic Electrophysiological Properties of Mature Hippocampal Astrocytes In Situ

A quasi-physiological Vm of -69 mV was retained when inwardly rectifying Kir4.1
was inhibited by 100 $\mu$M Ba2+ in both wild type and TWIK-1/TREK-1 double gene
knockout astrocytes, indicating expression of additional leak K+ channels yet
unknown.
%http://molecularbrain.biomedcentral.com/articles/10.1186/s13041-016-0213-7

  \item ??? [unknown]
  
We need to explore alternative K+ channels responsible for the still
mysterious passive conductance in astrocytes.
\end{itemize}


\section{Calcium elevation}
\label{sec:astrocyte-calcium-elevation}

$\Ca$ is detected using GCaMP (Shigetomi, Khakh, 2010) to monitor frequent and
highly localised nearmembrane calcium microdomains that were completely missed
with cytosolic calcium indicators such as GCaMP2 (Shigetomi et al., 2010).

Astrocyte calcium elevations are known to occur
\begin{itemize}
  \item   in vivo (Hirase et al., 2004; Wang et al., 2006; Dombeck et al.,
  2007; Gobel et al., 2007; Bekar et al., 2008; Schummers et al., 2008) and 
  
  \item in astrocytes from human brain slices (Oberheim et al., 2009)

   \item  calcium signalling in astrocyte processes is not correlated with that
   measured in the soma (Nett et al., 2002; Shigetomi et al., 2010) and another
   study has provided strong evidence for calcium signalling via astrocyte
   processes in the control of synaptic function (Gordon et al., 2009)   
  
  \item astrocyte calcium transients occur spontaneously and can be increased by
  neuronal AP firing and neurotransmitter release (Fiacco et al., 2009).
  
  Brain calcium-dependent astrocyte-to-neuron signalling is still debated with
  evidence for and against it 
  (Parpura et al., 1994; Pasti et al., 1997; Fellin et al., 2004; Fiacco et al.,
   2007; Lee et al., 2007; Petravicz et al., 2008; Gordon et al., 2009; Agulhon
   et al., 2010; Gourine et al., 2010; Henneberger et al., 2010)  
   
\end{itemize}


