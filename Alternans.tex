 
\chapter{Alternans}
\label{chap:alternans}


\section{Introductions}
\label{sec:introductions}

% Intracellular calcium transient alternans (CTA) is ...

A brief introduction of alternans is given in
Sect.~\ref{sec:card-cell-altern}. Markers for detecting increased risk of
sudden cardiac death (SCD) due to calcium transient alternans (CTA) are
 {\bf T-wave alternans} (TWA) and {\bf pulsus alternans}. It has been  studied
 for years; however, the underlying mechanism remains  controversial and thus
 the origins is not yet fully understood. CTA  was thought to be caused
 primarily by the alternations of the duration  of cardiac AP (a
 review:~\citep{myles2008}).  However, the \ce{Ca^2+}  handling has
 instabilities on its own that can result in CTA  independently of APD
 alternans~\citep{chudin1999icd} which promotes  {\bf ventricular fibrillation}
(VF)~\citep{adam1984,smith1988,nearing1991}.  We will use
computational model to study how randomly occurring \ce{Ca^2+} sparks
interact collectively to result in whole-cell \ce{Ca^2} alternans.

Early models hypothesized the mechanism of \ce{Ca^2+} alternans is
based on the CICR mechanism, i.e.
\textcolor{red}{the models replicates the steep relation showing the
  dependence of} \ce{Ca^2+} release to the SR \ce{Ca^2+} load before
  release.
Here, the luminal \ce{Ca^2+} overload is the proposed mechanism of
CTA~\citep{diaz2004src}. In other words, luminal \ce{Ca^2+} regulates
the sensitivities of RyRs by the interactions of the auxiliary
proteins (triadin-1/junctin, T/J) with the luminal \ce{Ca^2+} buffer
CASQN~\citep{terentyev2003cdf,Gyorke2004,knollmann2006cdc}
(Sect.\ref{sec:RyR_luminal_Calcium}).
\begin{itemize}
\item CSQN and T/J complex regulates luminal \ce{Ca^2+} release
\item In transgenic mouse, the over(under)-expressed CSQN show
  longer(shorter) refractory periods. 
\end{itemize}
At low luminal concentration, CSQN binds to T/J and this complex
inhibit the opening rate of RyRs. The underlying mechanism is that in
high luminal \ce{Ca^2+} concentration, CSQN is unbound from T/J and
then mostly bound to luminal \ce{Ca^2+}.  This makes RyR sensitivity
to \ce{Ca^2+} in the luminal is higher, i.e. its transition from close
to open state is higher.

Based on nanoscopic of imaging, \citep{restrepo2008cmm} created a
model to test two hypothesises:
\begin{itemize}
\item refractoriness can also cause the alternans: the refractoriness
  of RyRs associated with the delay in unbinding of CSQN from T/J, can
  also cause CTA; independently whether or not SR load is
  alternating. 
\item the \ce{Ca^2+} concentration dependence of CSQN
  binding/unbinding kinetics can explain the non-linear steepness of
  the SR release-load relationship at high load.
\end{itemize}

\citep{rovetti2010sis} induced three critical properties to lead to
\ce{Ca^2+} alternans are:
\begin{enumerate}
\item randomness (of \ce{Ca^2+} spark)
\item refractoriness (of
 RyRs in calcium-release unit (CRU) after
  \ce{Ca^2+} sparks)
\item recruitment (\ce{Ca^2+} sparks in one CRU may trigger \ce{Ca^2+}
  sparks in neighboring CRUs).
\end{enumerate}

\ce{Ca^2+} sparks can be triggered by one of the three ways
\begin{enumerate}
\item the opening of one or more DHPRs, during an AP through the CICR process
\item the spontaneous opening of RyRs, especially when SR \ce{Ca^2+}
  load is high, e.g. due to \ce{Ca^2+} leak
\item \ce{Ca^2+} diffusion from nearby CRU that have just released
  \ce{Ca^2+} from SR via CICR process. 
\end{enumerate}
The first two mechanisms are well-documented experimentally. However,
the third one, {\bf spark-induced sparks},q is often overlooked during
normal ECC. Recent experimental data showed that the transition from
\ce{Ca^2+} spark to \ce{Ca^2+} waves occur in normal ECC. This turns
spark-induced sparks to be important; and
\textcolor{red}{the sequential sparks often occur along the Z-line}.

\section{T-wave alternans}
\label{sec:t-wave-alternans}

T-wave alternans (TWA) is an electrocardiolographic (ECG) finding of
alternating T-wave morphology (amplitude and shape). This was first
discovered a hundred years ago~\citep{herring1909}.  With the new ECG
technology to detect microvolt (change in) T-wave amplitude (MTWA),
these small changes has been associated with the onset of
{\bf ventricular arrhythmias}, leading to sudden cardiac death. In
addition, several clinical trials have revealed that T-wave alternans
is a marker of susceptibility to ventricular arrhythmias in
humans~\citep{kavesh1998,ikeda2000,klingenheben2000}, with a greater
predictive accuracy than other factors.

\begin{framed}
  If the linking between microvolt T-wave alternans and ventricular
  arrhythmias can be confirmed, a noninvasively method then can be
  used to identify patients who may suffers sudden cardiac
  deaths~\citep{rosenbaum2001}.
\end{framed}

\begin{verbatim}
Pastore, J. M., S. D. Girouard, K. R. Laurita, F. G. Akar, and
D. S. Rosenbaum. 1999. Mechanism linking T-wave alternans to the
genesis of cardiac fibrillation. Circulation. 99:1385-1394.

Estes, N. A. M., G. Michaud, D. P. Zipes, N. ElSherif, F. J. Venditti,
D. S. Rosenbaum, P. Albrecht, P. J. Wang, and
R. J. Cohen. 1997. Electrical alternans during rest and exercise as
predictors of vulnera- bility to ventricular
arrhythmias. Am. J. Cardiol. 80:1314-1318.

Rosenbaum, D. S., L. E. Jackson, J. M. Smith, H. Garan, J. N. Ruskin,
and R. J. Cohen. 1994. Electrical alternans and vulnerability to
ventricular arrhythmias. N. Engl. J. Med. 330:235-241.

Pruvot, E. J., and D. S. Rosenbaum. 2003. T-wave alternans for risk
stratification and prevention of sudden cardiac
death. Curr. Cardiol. Rep. 5:350-357.
\end{verbatim}

\section{Pulsus alternans}
\label{sec:pulsus-alternans}




%%% Local Variables: 
%%% mode: latex
%%% TeX-master: "mainfile"
%%% End: 
