\chapter{Granule cells}
\label{chap:granule-cell-models}

Granule cells is discussed in Sect.\ref{sec:granule_cells}.
In this chapter, we focus on models for granule cells.

Synaptic currents induced by mossy-fiber stimulation in rat
granule cells are mediated by ar-amino-3-hydroxy-5-methyl-4-
isoxazolepropionoc acid (AMPA) as well as by NMDA receptor.


\section{cerebellar granule neurons (CGNs)}
\label{sec:cerebellar-granule-neuron}
\label{sec:CGN}

The cerebellum (Sect.\ref{sec:cerebellum}) plays an important role in motor
control, motor skill acquisition, memory and learning among other brain
functions. In rodents, cerebellar development continues after birth,
characterized by the maturation of granule neurons.

Cerebellar granule neurons (CGNs) are the most abundant neuronal type in the
central nervous system, and they provide an excellent model  for investigating
molecular, cellular, and physiological mechanisms underlying neuronal
development as well as neural circuitry linked to behavior (Selvakumar,
Kilpatrick, 2013).
%https://link.springer.com/protocol/10.1007%2F978-1-62703-444-9_5

CNGs can fire action potentials (APs) at high frequencies during sustained
depolarization.
\begin{itemize}
  \item background $\K$ current in CNG correlate with strong expression of 5
  $\KtwoP$ genes: TWIK-1, TREK-2c, THIK-2, TASK-1, and TASK-3
  
  It is likely that TASK-1/TASK-3 heterodimers are a prominent combination
in adult CGNs (Aller et al., 2005).  At physiological pH, TASK-3- containing
channels (homodimers or heterodimers) will increase potassium permeability to a
greater extent than TASK-1 homodimeric channels. 

Previous evidences: genetic deletion of TASK-1 (i.e. TASK-1 KO neurons) does not
alter the leak conductance, the RMP, or AP firing in adult CGNs (Aller et al.,
2005).

Now: TASK-3 KO neurons show a significant depolarization of the RMP in CNGs
(Brickley et al., 2007).


\end{itemize}


\section{Gabbiani-Midtgaard-Knopfel (1994)}
\label{sec:granule-cell-Gabbiani-1994}

\citep{gabbiani1994} developed a compartmental model for turtle cerebellar
granule cells (13 compartments: soma + 4 dendrites).

They used the model to study synaptic input via mossy fiber to granule cell.

\subsection{Assumption}

The strength of the synaptic input was adjusted at the most distal compartment
of the dendrite so that the synchronous activation of 2 mossy fiber synapses
is enough to trigger the AP in the soma.



\subsection{Mathematical models}

6 ionic currents for soma
\begin{enumerate}
  \item Nat: fast activation and fast inactivation
  
  \item Ca(HVA):  high-Vm activated $\Ca$ current
  
  \item K(DR): delayed rectifier $\K$ current (Sect.\ref{sec:delay-rectifier})
  
  \item Na/K exchanger: activated at hyperpolarized Vm
  
  \item K(Ca): $\Ca$-dependent and Vm-dependent $\K$ current
\end{enumerate}
whose kinetics were derived from rat and turtle granule cells.

NCX model is discussed in Sect.\ref{sec:NCX-Gabbiani-1994}.

\subsection{Ca2+ dynamics in soma}

$\Ca$ in the soma was modeled with radial diffusion, controlled by NCX, buffers.



