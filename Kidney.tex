\chapter{Kidney}
\label{chap:kidney}

Kidney is a non-gland organs yet it also secretes hormones (see
Chap.\ref{chap:Glands}).

\section{Adrenal gland: adrenal cortex + adrenal medulla}
\label{sec:adrenal-gland}
\label{sec:adrenal-medulla}
\label{sec:adrenal-cortex}

{\bf Adrenal gland}: The adrenals are two small glands that look like mushroom
caps. One adrenal gland sits above each kidney. These glands are 1-2 inches long
and weigh only 1.5-2.5 grams.
They make several hormones that are needed for well being and normal body
functioning.
  
Adrenal glands produce hormones in response to signals from the hypothalamus
(Sect.\ref{sec:hypothalamus}) and the pituitary gland
(Sect.\ref{sec:pituitary-gland}) in the brain.

Each adrenal gland is made up of a large outer zone - the {\it adrenal cortex},
and a small inner zone - the {\it adrenal medulla}.

\begin{itemize}

    \item adrenal cortex: produces three different kinds of hormones:
    glucocorticoids [gloo-koh-KAWR-ti-koids], mineralocorticoids
    [min-er-uh-loh-KAWR-ti-koids], and androgens.
  
    \item adrenal medulla: produces dopamine (Sect.\ref{sec:dopamine}),
    epinephrine, and norepinephrine.
    
\end{itemize}
\footnote{\url{http://www.empoweryourhealth.org/issue-2/The-Adrenal-Glands-Small-but-Mighty}}.

In the adrenal medulla the enzyme that catalyzes the transformation of
norepinephrine to epinephrine is formed only in the presence of high local
concentrations of glucocorticoids from the adjacent adrenal cortex; chromaffin
cells in tissues outside the adrenal medulla are incapable of synthesizing
epinephrine.
