\chapter{Neuron Morphology}
\label{chap:neuron-morphology}


Cell morphology: cell body size, number of neurites, number of branching
points, total length, thickness, area covered.

The three-dimensional shape of a neuron plays a major role in determining its
connectivity, integration of synaptic inputs and cellular firing properties, and
also changes dynamically with its activity and the state of the organism.
Analyzing the three-dimensional shape of neurons in an unbiased way is critical
to understanding how neurons function and developing applications to model
neural circuitry.

\begin{itemize}
  \item Input format: See Sect.\ref{sec:NeuroLand} for tools to convert between
  formats
  
  \item Intermediate format (for visualization purpose)
  
  Tool: Vaa3D (Sect.\ref{sec:Vaa3D})
  
  \item Output format:

\end{itemize}
\begin{verbatim}
Input format: image data (raw 3D image stacks, each contains
    a single neuron or disconnected multiple neurons)
    3D *.TIFF (single color, (X,Y,Z) voxel storage)
    
Intermediate: 8-bit 
    V3DPBD (compressed Vaa3D raw file format)
       used in Janelia's FlyWorkstation and 
       Allen Institute's In Vitro Single Cell Characterization pipeline,
    --> can be visualized using Vaa3D
    
Output format: .SWC file
\end{verbatim}
Sect.\ref{sec:SWC-format}.

There are tools that help to create morphology data from raw stacked images
\citep{myatt2012}.
\begin{enumerate}
  \item manual:
  \begin{itemize}
    \item free:
    
    \item commercial: Camera lucida (2D or 3D)
  \end{itemize}
  
  \item semi-automatic: users need  to define the basic morphology, such as
  identifying the tree root and terminations, but branch paths are traced by the computer
  \begin{itemize}
    \item free: NeuroMantic (free academic use),  NeuronJ (2D reconstruction), 
    
    \item commercial: Imaris (3D reconstruction)
    
  \end{itemize}
   
  \item automatic: 
  \begin{itemize}
    \item free: NeuronStudio
    \item commercial: Imaris, AutoNeuron add-on for Neurolucida 
  \end{itemize}
\end{enumerate}

Here, we describe different efforts to define and advance state-of-the-art of
single neuron reconstruction: an essential unsolved challenge in brain science.
\begin{enumerate}
  \item NeuroMorph.org: from GMU
  
  \item BigNeuron: from Allen Brain Science
\end{enumerate}




\section{Introduction}

{\it Polarized protein trafficking} is a crucial determinant of
neuronal morphogenesis and synaptic function which in
turn govern connectivity and information processing.
At the single neuron level, this assymmetry starts with 
neurites grow at different rates \citep{ramirez2011}.
\begin{enumerate}
  \item the neurite with faster growth rate develop into the axons
  
  \item the remaining neurites develop into a complex and diverse dendritic
  arbor
\end{enumerate}



\subsection{Morphology recording}

\begin{enumerate}

  \item Electron microscopy offers the highest precision of cellular morphology, but due
to the method's practical constraints, is only applicable to fixed tissue and is
restricted to subvolumes of neurons. 
  
  \item  Confocal or two-photon microscopy offer excellent
possibilities for monitoring the growth in situ and in vivo of
fluorescent-stained neuronal structures at high resolution in three dimensions,
while the neuron is in its natural environment
\end{enumerate}

\subsection{Morphology reconstruction}

Volume reconstructions, which provide surface and volume
measures, are a commonly available method for the automatic
reconstruction of neural morphology from confocal image
stacks, but information about branching number, diameter, and
length must still be determined.

For tools, both manual or automatic, see
Sect.\ref{sec:tracing-software-neuron-morphology}

\section{Experiment}


To visualize dendritic and axonal arbors neurons are typically labeled with
markers (e.g. biocytin (neurobiotin 0.125\%) or GFP) and imaged using
brightfield or fluorescence microscopy.
\url{http://diademchallenge.org/challenge.html}

\begin{enumerate}
  \item animal selection
  
Example: p21-25 Drd1a eGFP Swiss Webster mice
  
  \item slice preparation
  
Example: 300$\mum$ thickness acute corticalstriatal slice 
  
  \item fill a neuron with marker
  
Example: a neuron is filled with biocytin  through a patch pipette and tagged
with Alexa 488.
Biocytin (Sigma) at 5 mg/ml was dissolved into the patch-clamp pipette solution,
and cells were filled during at least 20 min of recording (performed at 32$^\circ$C).
Subsequently, slices were fixed overnight in 2\% paraformaldehyde at 4$^\circ$C.
Biocytinfilled cells were visualized using the avidin- biotin- horseradish
peroxidase reaction (ABC Elite peroxidase kit; Vector Laboratories) according to
the instructions of the manufacturer.
  
  \item the image is captured:
  
Example: using infrared differential interference contrast video microscopy.
  
  \item a tracing tool is used to analyze the captured image
  
Example: Neurolucidia (Sect.\ref{sec:Neurolucidia}) is used to manually traced
the neuron. The trace data include the length and diameter of the dendrites

\url{http://diademchallenge.org/metric.html}

  \item converted to a morphological data format: Sect.\ref{sec:NeuroLand}
 \begin{itemize}
   \item .SWC file
   
   \item NEURON hoc file: using NLMorphologyViewer and NLMorphologyConverter
   software. Here, all spines are ignored.
   
 \end{itemize} 
  
\end{enumerate}

\section{Morphology viewer}
\label{sec:morphology-viewers}

\begin{enumerate}
  \item NLMorphologyViewer - Sect.\ref{sec:NLMorphologyViewer}

  \item SharkViewer - Sect.\ref{sec:SharkViewer}
  
  \item cvApp - Sect.\ref{sec:CVapp}
  
  \item Cvapp-neuroMorpho - Sect.\ref{sec:Cvapp-neuroMorpho}
  
  \item L-viewer - Sect.\ref{sec:L-viewer}
  
  \item Dendro1 - Sect.\ref{sec:Dendro1}
\end{enumerate}

\subsection{SharkViewer}
\label{sec:SharkViewer}

A javascript-based tool to view 3D of single neuron in SWC file.

\url{https://github.com/JaneliaSciComp/SharkViewer}

\subsection{Cvapp}
\label{sec:CVapp}

CVAPP was originally written by Cannon et al. (1998) in Java that can display
.SWC file, NL and NL3 format.


\subsection{Cvapp-neuroMorpho.org}
\label{sec:Cvapp-neuroMorpho}

This is the trimmed version of Cvapp (Sect.\ref{sec:CVapp}) that enables
displaying and convert the .SWC file to some other formats, call
Cvapp-NeuroMorpho.org



\subsection{L-viewer}
\label{sec:L-viewer}

L-viewer is the viewer associated with L-neuron project
(Sect.\ref{sec:L-neuron}).


\subsection{Dendro1}
\label{sec:Dendro1}

Dendro1 converts morphological files in their dendrograms.
It is written in Java and runs in DOS. 

Input: SWC, Output: SWC.

Dendro1 reduces each branch to a single segment while conserving its total
length, beginning and ending diameter values. The result will be a 2D
representation of each tree in the neuron, i.e. all angle information is
removed.



\section{Morphology file format}

\subsection{ASC file format (Neurolucidia)}
\label{sec:ASC-file-format}

This is text-file format for representing morphology from Neurolucidia.
Neurolucidia also has its proprietary binary format (.DAT). To convert to a
different format, use NLMorphologyConverter -
Sect.\ref{sec:NLMorphologyConverter}.

\subsection{NineML file format}
\label{sec:NineML-file-format}

NineML (Raikov et al., 2011)

\subsection{NeuroML file format}
\label{sec:NeuroML-file-format}

NeuroML is an XML based model description language (Gleeson et al., 2010).
The focus of NeuroML is on models which are based on the biophysical and
anatomical properties of real neurons.

The initial idea started in 2001 at Edinburg. Neosim (2003)  was developed
based on this goal, that can load a range of plugin components  to handle different aspects of
a simulation problem.

A software library - NeuroML Development Kit (NDK) - was developed by Howell and
Cannon to simplify the process of serializing models in XML.
Developers of plug-ins for Neosim were free to invent their own structures and
serialize them via the NDK, in the hope that some consensus would emerge around
the most useful ones. As no other users, except those from Edinburg, the
project ended in 2005.

A different effort to develop a language for describing neuronal morphologies -
called {\bf MorphML} - that would include all of the necessary components to
serve as a common data format with the added advantages of XML.

The effort for NeurophML was then merged with the effort in neuroConstruct
(Sect.\ref{sec:neuroConstruct}) to develop {\bf NeuroML}. The schema was divided
into levels (e.g. MorphML, ChannelML, and NetworkML) to allow different
applications to support different part of the language

\subsection{-- Level 1: MorphML}

Level 1 focuses on the anatomical aspects of cells.


\subsection{-- ChannelML}


\subsection{-- NeuroML}

\subsection{-- Nodes, Segments, Sections}




\subsection{SWC file format}
\label{sec:SWC-format}

Each row
\begin{verbatim}
row-index, branch-type-index, X, Y, Z, radius, parent-row
\end{verbatim}
Columns 1, 2 and 7 are always integer. Columns 3,4,5, and 6 represent whatever
units were used in the reconstructions process (e.g. pixels, micometers, etc.) and can have decimal points.
Row 1 has a parent = -1, which means that this row does not have a parent and is
thus the root of the reconstruction.


with 
\begin{verbatim}
branch-type-index 
    
    0 = undefined
    1 = soma
    2 = axon
    3 = dendrite (basal)
    4 = apical dendrite
    5 = fork point
    6 = end point
    7 = custom
X,Y,Z
    Cartesian coordinate of the given node
radius
    the radius of the given node    
parent-row
    if -1, then this is the first point (the soma)
    
\end{verbatim}
\url{http://diademchallenge.org/faq.html#faq02}

\url{http://research.mssm.edu/cnic/swc.html}

\section{Soma surface area and volume}

Soma can be a single point or more than one point. When the soma is encoded as
one line in the SWC, it is interpreted as a "sphere". When it is encoded by more
than 1 line, it could be a set of tapering cylinders (as in some pyramidal
cells) or even a 2D projected contour ("circumference").


If the soma is represented as a single point in SWC file format, then 
the soma is treated as a sphere, with volume and surface area calculated
following the ones for the sphere.

If the soma is represented based on morphometric data with multiple points
forming a polygon, calculating the surface area and volume in 3D is more
complicated, and depending on the formula being used.
\begin{enumerate}
  \item if they represent a set of tapering cylinders: 
  
  
  \item if they represent 2D projected contour (circumference): 
\end{enumerate} 

Another way that people some time use is that operator interrupts a sequence of
somatic measurements with measurements of axons and dendrites that are attached
to the soma. Total soma area would then be the sum of the areas of each of the
soma sections.

\begin{mdframed}

In NEURON software, everything, including the soma, is treated as a
single/multiple-compartment section. So, the software itself cannot tell which
part is a soma or not-soma. 

With a single-point soma, NEURON's response by find the point connect
to the soma, e.g. the axon on the second line, and reassign it to the soma
so that it defines a cylinder.

If a section with 1 $\mu$m diam and 1000 $\mu$m length that you call "soma",
then NEURON can tell that the "soma" has an area of 1000*PI
square microns. To calculate the surface area in NEURON, users are
recommended to write code as given
\begin{verbatim}
func secarea() { local x, sum
  sum = 0
  for (x,0) sum += area(x)
  return sum
}

secname A = secarea()
\end{verbatim}

Import3D feature of NEURON enable importing a morphology into the system for
building the sections. In the case of pyramidal cell morphologies, shape plots
will appear most natural if the root branches of apical trees are attached to soma(1) and the root
branches of basal trees are attached to soma(0); a few trees of either class
might arise from soma(0.5) without damaging the appearance. This strategy favors
making soma(0) the reference point for distance measurements.

\end{mdframed}

Ref
\begin{itemize}
  \item \url{https://www.neuron.yale.edu/phpBB/viewtopic.php?f=13&t=2161}
  
  \item \url{http://www.neuron.yale.edu/phpbb/viewtopic.php?f=8&t=858}
  
\end{itemize}

\section{Data-source}

\subsection{NeuroMorph.org}
\label{sec:NeuroMorpho.org}

\url{http://neuromorpho.org/neuroMorpho/bycell.jsp}


\subsection{BigNeuron}
\label{sec:BigNeuron}

\url{http://alleninstitute.org/bigneuron/about/}


\section{Tracing}
\label{sec:tracing-software-neuron-morphology}

GOAL: to detect, visualize and measure branched structures in 2D and 3D images.
\begin{itemize}
  \item  tracing blood vessels, roots
  
  \item Neurite tracing : detecting and measuring dendritic or axonic outgrowths
  of a neuron in cell culture or in living or fixed tissue. The term neurite is
  discussed in Sect.\ref{sec:neurite}.

Neurite tracing can be done manually using common image analysis software (e. g.
ImageJ, Adobe Photoshop), but this is very labor intensive.

Fully automated programs are available (e.g. Neurolucida (Sect.\ref{sec:Neurolucidia})
or HCA-Vision (Sect.\ref{sec:HCA-vision})) and are much less labor intensive,
but are often cost prohibitive.

An alternative is to use the semi-automated tracing program NeuronJ
(Sect.\ref{sec:NeuronJ}).

\end{itemize}


\subsection{HCA-vision (tracing)}
\label{sec:HCA-vision}

HCA-vision is a commercial software for neurite tracing
(Sect.\ref{sec:tracing-software-neuron-morphology}).

%\section{Wis-Neuromath}

% \section{Imaris}

% \section{Amira}


\subsection{ImageJ}
\label{sec:ImageJ}


ImageJ plugins
\begin{itemize}
  \item NeuronJ - Sect.\ref{sec:NeuronJ}: 
  
  \item NeuriteTracer - Sect.\ref{sec:NeuriteTracer}: 
  
  \item Simple Neurite Tracer: manual, support 3D
  
No: batch processing

Output: number of nuclei, length of dendrites
\end{itemize}

\subsection{NeuronJ (tracing)}
\label{sec:NeuronJ}

NeuronJ is another tool for neurite tracing (Sect.\ref{sec:neurite})
\begin{itemize}
  \item  manual neuron tracing
  
  \item No: 3D support, batch processing
  
  \item Suitable: not too complex structures
  
  \item Output: length of dendrites
  
\end{itemize}


\subsection{NeuriteTracer}
\label{sec:NeuriteTracer}

NeuriteTracer is designed to trace neurite (Sect.\ref{sec:neurite}) starting
from a given soma location
\begin{itemize}
  \item  automatic, batch processing, require nuclei staining for detection nuclei first
before tracing.
  
  \item  No: 3D support.
  
  \item Output: number of cells, length of dendrites.
  
\end{itemize}

In contrast to NeuriteQuant, NeuriteTracer requires images of separated nuclei
for quantification of average neurite length, and is thus less reliable at high
densities of non-neuronal cells, such as in cultures of differentiating P19
cells.

\subsection{Wis Neuromath}
\label{sec:Wis-Neuromath}

automatic, batch processing.

No: 3D support

Features:
\begin{itemize}
  \item cell morphology
  \item neurite length analysis (in tissue)
  \item ganglion explant analysis (similar to Sholl analysis)
\end{itemize}

\subsection{Imaris (tracing, spine analysis)}
\label{sec:Imaris}

Commercial tool, automatic and manual processing.

Features:
\begin{enumerate}

  \item  3D reconstruction of dendrites and spines: 3D-MIP rendering (filament
  tracer tool embedded)

  \item output: detailed spine analysis, dendrite length, volumes, thickness of
  dendrites
\end{enumerate}

No: batch processing

\subsection{Amira}
\label{sec:Amira}

Commercial tool, automatic and manual processing.

Features:
\begin{enumerate}

  \item  3D reconstruction of dendrites and spines

  \item output: dendrite length, volumes, thickness of
  dendrites
\end{enumerate}

No: batch processing

\subsection{-- Evers et al. (2004)}

Evers et al. at Germany \citep{evers2004} have developed a tool set for
automatic geometric reconstruction of neuronal architecture from stacks of
confocal images.


\subsection{NeuriteIQ}
\label{sec:NeuriteIQ}

NeuriteIQ. This tool has similar features to NeuriteTracer, but is reported to
be more accurate.

\subsection{Wu et al. (2010)}
\label{sec:Wu-et-al.2010}

The analysis algorithm of Wu et al. is optimized to detect neurites
(Sect.\ref{sec:neurite}) with high accuracy, but it also requires more
computational power (approximately four-fold slower than NeuriteQuant -
Sect.\ref{sec:NeuriteQuant}), which might be disadvantageous for large-scale
analysis, such as in high-content screening campaigns \citep{wu2010}.

The program is based upon MATLAB.


\subsection{NeuriteQuant (2011)}
\label{sec:NeuriteQuant}

This tool enables fully automated morphological analysis of large-scale image
data from neuronal cultures or brain sections that display a high degree of
complexity and overlap of neuronal outgrowths.

The majority of the NeuriteQuant tool is implemented as an ImageJ macro
(Sect.\ref{sec:ImageJ}), and can be easily manipulated using a simple text
editor. Additional functionality that could not be implemented as an ImageJ
macro was added in the form of ImageJ plugins using the programming language
Java. The source code for these custom-made plugins is also included in the
NeuriteQuant package.

The program follows a similar approach of Wu et al.
(Sect.\ref{sec:Wu-et-al.2010}). NeuriteQuant trades off accuracy for speed as
compared to the method of Wu et al.


\subsection{Vaa3D}
\label{sec:Vaa3D}

Vaa3D performs 3D, 4D and 5D reconstruction and rendering of very large image
data sets, especially those generated using various modern microscopy methods, and
associated 3D surface objects.

Vaa3D has a rich set of functions and plugins for neuron quantification, and is
compatible with well-established neuron analysis tools such as L-Measure.
Vaa3D is suitable for manual, semi-automatic, and completely automated digital
tracing. This software has been used in several large neuroscience initiatives
and a number of applications in other domains.

\subsection{-- plugin writer}

Vaa3D has a simple and powerful Open Source plugin interface that allows
extending the functionalities of the software. Specifically, a developer can
take advantage of a tool called Vaa3D Plugin Creator (which is itself a Vaa3D
plugin) to generate source code templates for porting algorithms or other
functions into Vaa3D simply and effectivel

\subsection{Neurolucidia (manual)}
\label{sec:Neurolucidia}

Neurolucida is a powerful tool for creating and analyzing realistic, meaningful,
and quantifiable neuron reconstructions from microscope images
(Sect.\ref{sec:tracing-software-neuron-morphology}). Perform detailed
morphometric analysis of neurons, such as quantifying:

\begin{itemize}
  \item  the number of dendrites, axons, nodes, synapses, and spines
  \item the length, width, and volume of dendrites and axons
  \item  the area and volume of the soma
  \item  the complexity and extension of neurons  
\end{itemize}
\url{http://www.mbfbioscience.com/neurolucida}

Manual reconstruction using Neurolucidia:
\url{http://research.mssm.edu/cnic/repository_manual.html}

\subsection{NeuroStudio (automatic)}
\label{sec:NeuroStudio}

Automatic reconstruction using NeuroStudio:
\url{http://research.mssm.edu/cnic/repository_automated.html}

\subsection{neuroConstruct}
\label{sec:neuroConstruct}

neuroConstruct can import morphology files in GENESIS, NEURON, Neurolucida, SWC
and MorphML format for inclusion in single cell or network models, or more
abstract cells can also be built manually.

neuroConstruct utilized an internal simulator-independent representation for
morphologies, channel and networks; and it can generate neuronal simulations for
the NEURON and GENESIS simulators. Nowaday, neuroConstructs use NeuroML file
format (Sect.\ref{sec:NeuroML-file-format})

\url{http://www.neuroconstruct.org/}

\section{Generate virtual neuron}
\label{sec:growth-neuron}

\begin{enumerate}
  \item neuroGEN - Sect.\ref{sec:neuroGEN}
  
  \item L-neuron - Sect.\ref{sec:L-neuron}
\end{enumerate}

\subsection{neuroGEN (NTS)}
\label{sec:neuroGEN}

NTS system has developed neuroGEN - an algorithm to growth neuron
(Sect.\ref{sec:growth-neuron}) based on the concept of electric force-field
between atoms in molecular dynamics simulation.


\subsection{L-neuron}
\label{sec:L-neuron}

L-neuron is the project to create a virtual neuron  that are anatomically
indistinguishable from their real counterparts.

Note that different sets of statistical distributions correspond to different
morphological families. By varying these values, the same algorithm can describe
neurons as diverse as pyramidal, granule, Purkinje, or stellate cells.
\url{http://krasnow1.gmu.edu/cn3/L-Neuron/index.htm}

L-Neuron can output its virtual structures in a variety of formats, including
virtual reality, graphical files, and the standard neuroanatomical coordinates
compatible with neurophysiological simulators such as GENESIS and Neuron.

L-Neuron has implemented many algorithms
\begin{enumerate}
  \item Lyndenmayer rewrite rules
  
  \item Hillman algorithm and its derivatives - Sect.\ref{sec:Hillman-PK-algorithm}
  
  \item Burke and colleagues
  
  \item L-neuron algorithm

L-neuron adds stochastically the bifurcation angles and dendritic orientation
along the appropriately fragmented branches.

\end{enumerate}

\subsection{Hillman/PK algorithm}
\label{sec:Hillman-PK-algorithm}

A soma has a number of dendritic tree. The number of trees per neuron is the
first basic parameter.

Each dendritic tree is generated independently. 

For each tree: The simulation starts with sampling an initial stem diameter,
taper rate, and branch length. Dendritic tree then grows is an iterative process
depending on the branch diamter.
\begin{itemize}
  \item bifurcate
  \item terminate
\end{itemize}
depending whether its ending diamter (which is calculated from the branch's
initial diamter and the taper rate) is greater than a sampled diamter threshold.


{\bf If bifurcate to occurs}, then two daughter branches  are created, whose
initial diameters are calculated based on 
\begin{enumerate}
  \item dauther diameter ratio \verb!dr!: this is sampled:
  
  \item ending diameter of the parent branch: this is sampled
\end{enumerate}

Using formula (a variant of Rall's equation)
\def\PK{{\text{PK}}}
\begin{equation}
\PK \times d_p^{1.5} = d_1^{1.5} + d_2^{1.5}
\end{equation}
PK is a sampled numerical parameter (usually between 1 to 2.0).
then
\def\dr{{\text{dr}}}
\begin{equation}
\begin{split}
d_1 = d_p \sqrt[1.5]{\frac{\PK}{1+ \dr^{1.5}}} \\
d_2 = \dr . d_1
\end{split}
\end{equation}

The process continue with each new daugher: the new taper rate and branch length
are sampled for that daughter branch.

{\bf If terminate} (i.e. the ending diameter of the branch is less than the
sampled threshold), the branch grows by an additional 'terminal' length and
stops.

SUMMARY: For each branch; we need to sample new value for branch length, taper
rate, diameter threshold, daughter diameter ratio and PK (or terminal length).



\section{NeuroLand}
\label{sec:NeuroLand}

\url{http://neuronland.org/NL.html}

Neuronland provides free software tools for the experimental neuroscience and
neuron mathematical modeling communities.

Currently, two applications are available, both focused on facilitating the
interchange of neuron morphology data between the diverse software packages used
within these fields of research, as well as the preservation of such data.
\begin{enumerate}
  \item NLMorphologyConverter
  
  \item NLMorphologyViewer
\end{enumerate}

\subsection{NLMorphologyConverter}
\label{sec:NLMorphologyConverter}

It can convert from one 3D neuron morphology format to another.
Currently 21 formats (50+ variations) are supported, including Neurolucida, SWC,
MorphML, NeuronHOC, Genesis, NeuroZoom, Eutectics. 


\subsection{NLMorphologyViewer}

It is a simple user interface built on top of the technology developed for the
NLMorphologyConverter


\section{Other tools}


\url{http://research.mssm.edu/cnic/tools.html}

