\chapter{Steps to build a model}

There are different scale/level of modeling
\begin{itemize}
  \item single-cell level
  \begin{enumerate}
    \item point neuron, 
    
    \item multi-compartmental single-branch neuron, or 
    
    \item multi-compartmental branched neuron.
  \end{enumerate}  
  in a well-mixed system, or spatially system in 1D, 2D or 3D.

  \item network-level: what is the details of the representation for a single
  element in the network, e.g. 
  \begin{enumerate}
    
    \item intracellular reaction network
  
    \item voltage dynamics that involve ion channels
  
    \item voltage dynamics and intracellular ion concentration dynamics 
  \end{enumerate}
  
\end{itemize}

A major problem with models of intracellular reaction networks is the overall
lack of comprehensive quantitative data, be it kinetic constants of molecular
interactions, or spatially resolved concentration distributions of the molecular
species involved.

Typically, it initially involves considerable effort to fine-tune the model to
fit certain experimental results, before being able to use simulations to
explore the predicted behavior of the system.

There are different softwares we can use (Part.\ref{part:Modeling-Software}).

\section{Workflow}

There are 4 main steps to perform in order to build a realistic model neuron,
and to connect it to others in biologically realistic network.
\begin{enumerate}
  \item create morphological reconstruction of the cell 
  
REMEMBER: To model a multi-compartmental cell model on a computer, we need to
numerically solve the following equation for each compartment
\begin{equation}
\Cm \frac{d\Vm}{dt} = \frac{E_m - \Vm}{R_m} - \sum_k g_k \times (\Vm - E_k) 
     +  \frac{\Vm(left) - \Vm - }{R'_a} + \frac{\Vm(right) - \Vm}{R_a} +
     I_\inj
\end{equation}  
with 
\begin{itemize}
  \item  $R_m$ is membrane resistance (which can be used to infer $g_\leak$ -
  Sect.\ref{sec:leak-current}) - Sect.\ref{sec:membrane-resistance}.
  
  \item the equilibrium potential $E_m$ (or $E_\leak$) is typically closed
  to the resting membrane potential (or V$_\text{rest}$)
  
  In some cases, it is given a slightly different value, Eleak, in order to
  reduce the net channel current to zero when $\Vm = E_\rest$

  \item with $\Vm(left), \Vm(right)$ are the membrane potential value at the two
  adjacent compartments; and $R'_a$, $R''_a$ are the axial resistance between
  the current compartment to each of the two adjacent ones.
\end{itemize}



  \item build a suitably realistic passive compartmental model, without the
  ion channels (i.e. those with variable conductances)
  
We we need to calculate $\Cm, R_m, R_a$ for each compartments. NOTE
\begin{itemize}
  \item $R_m$ is measured experimentally 
  
  \item $C_m$
  
  \item $R_a$
\end{itemize}  
  The hyperpolarizing current is applied, i.e. to disable ionic channels
  (typically agonists are also used to create completely block), and the $\Vm$
  changes can be used to estimate passive membrane properties.
  
There are three things that can be measured experimentally and are related to
the parameters that we need for a model
\begin{enumerate}
  \item space constant or length constant: show the attenuation of $\Vm$ over
  distance from the injection site.
  \item input resistance (measured at soma) - Sect.\ref{sec:input-resistance}
  \item membrane time constant  $\tau_m$
  
\begin{equation}
\tau_m = R_m C_m = R_M C_M
\end{equation}
This is a function of the cell surface area; but not the dimension of the
section/compartment.

The value is importance for determining the time that it takes for ionic
currents to produce changes in the membrane potential, and can be used to
estimate $R_M$. 

We can estimate from $\Vm$ recording at soma, if we assume single-compartment
then fit the change of $\Vm$ as an exponential fit
\begin{equation}
\Vm(t) = \Vm(0) \times \exp\left( \frac{-t}{\tau_m} \right)
\end{equation}
NOTE: We can apply the log operator and find the coefficient $\tau_m$ from the
straight line.
\end{enumerate}

  
  \item add appropriate variable conductances to compartments that correspond to
  the part of the cell 
\end{enumerate}
\url{http://www.genesis-sim.org/cnslecs/cns2a.html}
\url{http://www.genesis-sim.org/GENESIS/iBoG/iBoGpdf/chapt2.pdf}

A good common workflow
\begin{enumerate}
  
  \item  first develop and analyze a compartmental model (one or many
  compartments): zero-dimensional model where a well-mixed compartment
  approximation is being used - essentially assuming infinitely fast diffusion
  rates

This is an ODE-only mathematical only system.


  \item more complicated model
  
The single cell modelling workflow comprises various steps
\begin{enumerate}
  \item curate morphologies
  
  \item  extraction   and   definition   of   electrical   features,   
  
 To automatically extract these data from time-series data (both in vitro and
 in silico) to compare them, \verb!eFEL! library can be used
 (Sect.\ref{sec:eFEL})
  
  \item search   for   optimal parameters,
  
NOTE:    when building neuron models for other brain  regions it  is sometimes
necessary to define novel electrical features (e.g. complex dendritic features)
or to constrain the parameters with additional knowledge (e.g. make        
certain parameters dependent on each other)

The optimizer framework in Python called {\bf BluePyOpt}
(Sect.\ref{sec:BluePyOpt}) that utilize NEURON simulator, and parameter
extraction eFEL library, based on some parameter search method (can be chosen).


  \item model  selection  and  inspection
\end{enumerate}

\end{enumerate}

\begin{enumerate}
  \item Perform a sweep protocol of series of voltage steps, find the I-V curve
  (Sect.\ref{sec:I-V-curve}) and try to fit with the experimental data
  
This I-V curve can be measured under whole-cell condition; or the condition in
which certain channel is enabled; but all the others are blocked to study the
kinetics for that single channel only
\begin{itemize}
  \item NCX - Sect.\ref{sec:I-V-curve-NCX}
  
\end{itemize}
  
  \item 
\end{enumerate}


The modeling study of excitable cells is conducted based on the electricical
representation of the transmembrane protein to model the current flow $I$.
The Ohm's law for an electric circuit: $I=V/R$ is often written in the form
$I=g(V_m-E_\rev)$ in cellular electrophysiology (Sect.\ref{sec:ohms-law}) under
the assumption that the ionic current, at an instantaneous of time, is linearly
dependent upon the membrane potential. 

The above equation is then extended for an equivalent circuit of the region of
plasma membrane of isopotential, first presented by Hodgkin-Huxley in their
landmark model (Sect.\ref{sec:leaky_integrate-and-fire_model}):
\begin{equation}
\Csc \frac{dV_m}{dt} = \sum_\ion g_\ion (V_m - E_\ion) + I_\app
\end{equation}
Of course, we will learn that there are certain ionic channels whose ionic
current (e.g. Calcium channels) is better modeled as a non-linear function of
membrane potential, using GHK equation (Sect.\ref{sec:GHK_voltage}).

When the isopotential assumption is relaxed, the formula that incorporate the
voltage change over the length of the neuron (soma, dendrite, axon) needs to use
the formula (Sect.\ref{sec:Hodgkin-Huxley-1952-model-of-propagation})
\begin{equation}
\Csc \frac{dV_m}{dt} = \frac{d}{4R_a}\nabla^2 V_m + \sum_\ion g_\ion (V_m - E_\ion) + I_\app
\end{equation}

\section{Modeling approaches}

\subsection{deterministic/stochastic 3D/2D/1D/0D}

\begin{enumerate}
  \item deterministic
  
  \item stochastic
\end{enumerate}
which can be 3D, 2D, 1D, or 0D (well-mixed (non-diffusional) compartmental
models).

\subsection{-- Gibson-Bruck Next Reaction method}
\label{sec:Gibson-Bruck-Next-Reaction-method}

Gibson-Bruck Next Reaction method [13], an optimized version of one of the
Gillespie algorithms, which interprets reactions as Poisson processes.

% Gibson M, Bruck J. Efficient exact stochastic simulation of chemical systems with many species and
% channels. J Phys Chem A 2000;104:1876-1889

 
\subsection{mechanical force}

Modeling with mechanical force to accounting for diffusion in crowded spaces.


\subsection{Rule-based modeling}
\label{sec:Rule-based-modeling}

Rule-based modeling is invaluable when the number of possible species and
reactions in a model become too large to allow convenient manual specification.

Models accounting for hundreds of species and reactions are no longer rare [17].
This is a typical situation when protein complexes and phosphoforms are included
into a model, e.g. a receptor with 10 tyrosine phosphorylation sites can exist
in $2^{10}=1024$ different phosphoforms.
Many or all of these phosphoforms have to be accounted for individually to
simulate and predict the time course for receptor-mediated signaling (Blinov et
al., 2006).  Manually specifying a list of species and reactions becomes
error-prone and slow.
Furthermore, when potential protein complexes are being included, the size of
reaction network can easily increase by a few orders of magnitude, e.g. In a
system where A, B, C interact with each other needs to include AB, AC, BC, ABC.
The cost of many commonly used methods for simulating network dynamics is a
function of network size.

A solution for this combinatorial complexity challenge is provided by a
rule-based approach, in which a model is specified in the form of bio-molecular
interaction rules (Sect.\ref{sec:BioNetGen}).


The system-level dynamics of many molecular interactions, particularly
protein-protein interactions, can be conveniently represented using reaction
rules, which can be specified using model-specification languages, such as the
BioNetGen language (BNGL) - Sect.\ref{sec:BNGL}.


Several methods for simulating rule-based models have been developed that avoid
the expensive step of network generation. The cost of these " network-free "
simulation methods is independent of the number of reactions implied by rules


\begin{enumerate}
  \item BioNetGen: define model-specification language (BNGL) -
  Sect.\ref{sec:BioNetGen}
  
  \item NFSim: stochastic 
  
  \item RuleMonkey: network-free software for stochastic simulation (similar to
  Gillespie' s method) of
  rule-based models - Sect.\ref{sec:RuleMonkey}
  
  \item DYNSTOC, another BNGL-compliant tool for network-free simulation
  
  \item VCell - Sect.\ref{sec:VirtualCell-Rule-based}
\end{enumerate}


\section{Cell to cell variations}


One of the challenging problem when modeling a given cell type is the variation
between cells, e.g. morphology (somatic size, branching of dendrites in
neurons, cell sizes in cardiac myocytes), expression levels of proteins. 
Recent studies in mammalian cells and lower eukaryotes
showed a significant variation in the expression level of individual
proteins between genetically identical cells (Niepel et al., 2009).

Thus, it has long been recognized that differences from one cell to the next can
arise through variation in the extracellular environment (e.g. morphogen
gradients during embryogenesis) or from genomic alteration (e.g. genetic
heterogeneity within a tumor). Only recently has it become clear that
intracellular biochemical fluctuations can also have profound effects on
phenotype.

Even members of a genetically identical group of cells or organisms in identical
environments can exhibit variability in drug sensitivity, cellular response, and
phenotype. Underlying much of this variability is stochasticity in gene
expression, which can produce unique proteomes even in genetically identical
cells (Niepel et al., 2008).
\begin{enumerate}
  \item covariance of two proteins (CD8 and SHP-1)  in T-cell reduces the
  diversity of T-cel response (Feinerman et al., 2008 - Science)
\end{enumerate}


These variations - often exceeding a factor of 2 - likely reflect noise in
transcription, mRNA turnover, translation, and protein degradation.

\subsection{cellular mechanism}

The puzzling question is how signal transmission processes can tolerate large
variations in the expression level of the underlying signaling components.

As $\Ca$ signals regulate fundamental eukaryotic processes that range from
secretion, muscle contraction, and memory formation to activation and
differentiation of immune cells, it is hypothersized that additional control
principles must exist and involve Ca2+ signaling system to avoids failure and
copes with large cell-to-cell variations in protein expression.

To keep cells in a signaling-competent state, the Ca2+ signaling system employs
a system of pumps and channels to tightly balance Ca2+ fluxes across the
endoplasmic reticulum (ER) and plasma membrane (PM).

It also raises the questions of whether the pumps and channels are always
expressed at the correct numbers by using shared promoters or coupled protein
stability in a forward regulation strategy or, alternatively, whether cells
monitor the output of a pathway and use a feedback strategy to adjust the
concentrations of individual signaling components (Abell et al., 2009).





\subsection{modeling}
To consider this variation:
\begin{enumerate}
  
  \item The simplest models are conceptual or diagrammatic and attempt to
  account for the mean values of key variables (e.g. rates of physiological
  processes) and their dispersion about the mean

  \item More rigorous are methods based on stochastic differential equations (SDEs) or the
Langevin approach (Sect.\ref{sec:Langevin-approach})

\end{enumerate}

Finally, the chemical master equation, which describes the probability that
individual molecules will collide and react in a given time interval, can be
approximated numerically using the Stochastic Simulation Algorithm (SSA) or
Gillespie Algorithm (Sect.\ref{sec:Gillespie-method})


\section{Log-normal distribution}
\label{sec:Log-normal-distribution}


Log-normal distribution are long-tailed so that even a tight distribution (e.g.
CV=0.25) contains individuals (e.g. cells) in the 95-th percentile that have
(e.g. 2.5-fold) many times the abundance of a particular proteins than
individuals in the 5-th percentile. This spread increase dramatically with the
increase of CV. 

Many biological phenomenon follow log-normal distribution
\begin{enumerate}
  \item concentration distribution for many proteins across a clonial population
  of mammalian cells
\end{enumerate}