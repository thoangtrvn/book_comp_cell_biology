\chapter{Inferior Olive Neurons (IO neurons)}
\label{sec:Inferior-Olive-neurons}

The inferior olive neurons are those from inferior olivary nucleus
(Sect.\ref{sec:inferior-olive-nucleus}), that projects the signal onto Purkinje cell in
the cerebellum (Sect.\ref{sec:Purkinjie_nerves}).
Remember that each Purkinje cell receives thousands of parallel fibers input;
but only 1 climbing fiber input. However, such single input can interrupt the
rapid spiking behavior of the Purkinje cell as it generates a so-called complex
spike (CS) after which the PC falls silent for approximately 15 ms.
As such, the axon of such neurons - called the climbing fiber - spikes lead to
large all-or-none action potentials in cerebellar Purkinje cells, overriding any
other ongoing activity and silencing these cells for a brief period of time
afterwards.

The IO neurons can generate spontaneous action potential (AP)
and such activation of climbing fibers is generally believed to be related to
timing of motor commands and/or motor learning.

\section{Schweighofer et al. (1999)}
\label{sec:Schweighofer-1999}

% \textcolor{red}{The original paper has an error: They used A-type potassium
% formula ($I_A$) for delayed rectifier potassium current (KDR)}. This need to be
% fixed.

The two-compartment model represents the known distribution of ionic currents
across the cell membrane, as well as the dendritic location of the gap junctions
and synaptic inputs.
\begin{enumerate}
  \item somatic compartment:  low-threshold calcium current (\verb!ICa_l!),
an anomalous inward rectifier current (Ih), a sodium current (INa), and
a delayed rectifier potassium current (\verb!IK_dr!).
  
  \item dendritic compartment:
  high-threshold calcium current (\verb!ICa_h!), a calcium-dependent
potassium current (\verb!IK_Ca!), and a current flowing into other cells through
electrical coupling (Ic).


  \item Ih current - adopted from Sect.\ref{sec:Ih-Huguenard-McCormick-1992}
  %with a modified time constant $\tau_m$ (Sect.\ref{sec:Ih-Schweighofer-1999})
  

\end{enumerate}

The model used an artificial sodium current to prevent generation of sodium
spike bursts at the soma.

\section{Van Der Giessen - \ldots - De Zeeuw (2008)}
\label{sec:VanDerGiessen-DeZeeuw-2008}

\begin{enumerate}
  \item Ih current - adopted from Sect.\ref{sec:Ih-Huguenard-McCormick-1992}
  with some modification to make it actives at a more hyperpolarized potential
  (V$_{1/2} = -80$ mV; and slope is $k=4.0$ mV)
  
The Ih current in this model is moved to the dendritic compartment


\end{enumerate}

\section{DeGruijl - DeZeeuw (2012)}

\begin{itemize}
  \item compartment to model the axon hillock of the cell and
enable the model to generate axonal bursts of sodium spikes

  \item somatic compartment's sodium and potassium currents were reworked
\end{itemize}

\section{Kozloski et al. (2014)}


\citep{kozloski2014} modified Schweighofer's work
(Sect.\ref{sec:Schweighofer-1999}) with changes to its several channel
conductances (Na, K, Cah, Cal, KCa, and h) targeted to neuronal branches in
order to replicate olivary oscillation.


Previous work has demonstrated that spike generation and bursting properties of
olivary neurons depend on axonal lengths, with experimental data only have 0-300
$\mum$ in slice preparation. The author conducted a more comprehensive study
using data from real Olivocerebellar axons, and used a 3D mesh of rat cerebellum
to constrain axonal morphology and length.

The authors demonstrated that the intra-burst spike frequency and number depend
on these variations in axonal lengths from 1-5 mm.
