\chapter{GSL}
\label{chap:MGS-GSL}

MGS (Model-Graph Simulator) is composed of 2 parts
\begin{enumerate}
  \item MDL : the script that helps generating C++ classes that conform to MGS
  code standard (Chap.\ref{chap:MGS-MDL})

The generated codes include
(1) the skeleton of the C++ class that can be used inside the GSL script, (2) a
series of \verb!CG_*! classes that support the calling the above class by the
framework.

It's important to add the behavior to the C++ class, which requires us to
understand how to in Sect.\ref{sec:GSL-C++-how-to-get-access-to}.

  
  \item GSL : the script that helps to connect different C++ classes together in
  a graph-based simulator system. (Chap.\ref{chap:MGS-GSL})
\end{enumerate}


\section{GSL script: how to use C++ classes generated from MDL}
\label{sec:GSL-script}

What you need to do in a GSL script
\begin{enumerate}
  \item phases - Sect.\ref{sec:MGS-GSL-phases}
\end{enumerate}

\subsection{a simulation defined as different phases}
\label{sec:MGS-GSL-phases}

Here, we explain how a typical simulation can be defined
via the concept of phases.

Define different phases you want 
\begin{verbatim}
InitPhases = { initialize1, initialize2, initialize3, initialize4, initialize5, initialize6 };

RuntimePhases = {
runphase1, runphase2, runphase3,
}

FinalPhases = { finalize };
\end{verbatim}

\textcolor{red}{Then the simulation will goes through like this, in sequential
order}
\begin{verbatim}
call any methods as 'initialize1' in parallel
call any methods as 'initialize2' in parallel
...
call any methods as 'initialize6' in parallel

i = 0
DO
   call any methods as 'runphase1' in parallel
   call any methods as 'runphase2' in parallel
   call any methods as 'runphase3' in parallel
   
UNTIL i > NumIteration

call any methods as 'finalize' in parallel

\end{verbatim}

\subsection{add components to a simulation}

Suppose you have the C++ class with the designed behaviors you need. 
These classes (depending upon how it was declared in MDL) can be used in
GSL as a different type (NodeType, ConstantType, VariableType, \ldots).

GSL script is a descriptive language, so you don't explicitly specify how many
instances of a particular \verb!X!Type (with X can be Node, Constant, Variable)
to be created, i.e. the MGS framework will know and handle that.
As a class, it has (1) data members, and (2) methods.
Here, we discuss how
\begin{itemize}
  \item initialize the data members
  
  \item put the methods into different phases
\end{itemize}

\begin{itemize}
  \item declare what \verb!X!Type to use, and maps its method to the proper
  phase defined above

\begin{verbatim}
NodeType HodgkinHuxleyVoltage( <
      Ra= Ra_value,      // Gohm*um
      Na= Na_i_value,          // mM
      K= K_i_value,            // mM 
      E_leak= E_leak_value        // mV
           >) { initializeCompartmentData->initialize2 };
\end{verbatim}

  \item Create a Grid, inside we 
  
  \begin{itemize}
    \item put a particular \verb!X!Type node into a Layer:
    
    \item Define the connectivity between 
    \begin{itemize}
      \item  instances of \verb!X!Type node
    within one layer or 
      \item instances of \verb!X!Type node in one layer to 
      instances of \verb!Y!Type node in another layer
  \end{itemize}
  
\end{itemize}



\subsection{-- initialize shared data member}
\label{sec:GSL_initialize-type-shared-data-member}

The data, including shared data defined in MDL
(Sect.\ref{sec:MDL-shared-members})
\begin{verbatim}
Node HodgkinHuxleyVoltage Implements ...
{

  Shared {
     dyn_var_t Ra;
     dyn_var_t Na;
  }
}
\end{verbatim}
can be initialized when being used in GSL in the form of name-data pair list
at the time of declaring the NodeType and its kernel mapping to different
phases.

Example: just declare the class type and pass the in the form of 
\begin{verbatim}
InitPhases {initialize1, initialize2};
RuntimePhases {runtime1, runtime2};

NodeType  HodgkinHuxleyVoltage(
    < Ra=0.015,
      Na=70,
    >
    ){ kernel1 -> initialize1, kernel2 -> runtime2};
\end{verbatim}
REMEMBER: If 'kernel1' and 'runtime2' are of the same name, we don't need to do
mapping.

The generated C++ code is a C++ class with suffix \verb!SharedMembers!
(Sect.\ref{sec:MDL-generated-classes})


\subsection{-- initialize data member}
\label{sec:GSL_initialize-type-data-member}

Private data member for a NodeType can be initialized when being used in GSL 
by calling to \verb!InitNode! statements (Sect.\ref{sec:InitNodes}), after
declaring the give NodeType and its kernel mapping as given in
Sect.\ref{sec:GSL_initialize-type-shared-data-member}.

\begin{verbatim}
InitNodes (.[].Layer(Layer-Name-Here), ...);
\end{verbatim}


At the code level, the inputs are treated as a \verb!ParameterSet*!
(Sect.\ref{sec:ParameterSet}), and then casted to the right datatype
\begin{verbatim}
void CG_HodgkinHuxleyVoltage::initialize(ParameterSet* CG_initPset)
{
   CG_HodgkinHuxleyVoltagePSet* CG_pset = dynamic_cast<CG_HodgkinHuxleyVoltagePSet*>(CG_initPSet);
   dimensions = CG_pset->dimensions;
   branchData = CG_pset->branchData;
   //...
   Vcur = CG_pset->Vcur;
   Vnew = CG_pset->Vnew;
   
}
\end{verbatim}


\subsection{-- map methods to phases}
\label{sec:GSL_assign-MDL-component-methods-to-GSL-phases}
\label{sec:GSL-RuntimePhases}

When we declare the nodetype, we use \verb!->! operator to redirect to a given
phase name
\begin{verbatim}
InitPhases = { initialize1, initialize2, initialize3, initialize4, initialize5, initialize6 };
RuntimePhases = {updateNMDADepPlasticity, solveChannels, predictJunction,
#if MAX_COMPUTE_ORDER>6
      forwardSolve7,
#endif
}

NodeType BackwardSolvePoint0 { produceInitialState->initialize1,
        produceBackwardSolution->solve };

NodeType ChannelHayNat { computeE->initialize2, initialize->initialize3, update->solveChannels };
\end{verbatim}

\subsection{stop/pause the simulation}

First, you need to understand how a Trigger-able object work
(Sect.\ref{sec:Triggerable-MGS}). Define the criteria for 
  \begin{itemize}
    \item pausing the simulation (Sect.\ref{sec:GSL-when-to-pause})
    \item stopping the simulation (Sect.\ref{sec:GSL-when-to-stop})
  \end{itemize}
  \end{itemize}
  
\subsection{-- Pause on}
\label{sec:GSL-when-to-pause}

\begin{verbatim}
Pause on <instance-Trigger>
\end{verbatim}

Sect.\ref{sec:Trigger} describes how to create an instance of a particular
Trigger.

\subsection{-- Stop on}
\label{sec:GSL-when-to-stop}
\label{sec:GSL-Stop}

GSL
\begin{verbatim}
Stop on <some-trigger>;
\end{verbatim}

In GSL, the \verb!Trigger! keyword is used to define a Triggerable nodetype
(Sect.\ref{sec:trigger-define-in-GSL}). First we need to declare the prototype
of the \verb!Trigger!-type class (Sect.\ref{sec:Trigger}). 

\begin{verbatim}

  // a trigger for calling a RuntimePhase
// We use modulo operator (!%) to determine when it is triggered 
UnsignedTrigger recOn ("Iteration Trigger : !%" STR(RecordInterval),
                         ::Iteration, "!%", RecordInterval, 0, solveChannels);

  // a trigger for calling a FinalizePhase
UnsignedTrigger endTrig("Iteration Trigger to end or stop",
                         ::Iteration, "==", 10, 0, lastPhase);

Stop on endTrig;
\end{verbatim}

\subsection{SeparationConstraint}
\label{sec:GSL-SeparationConstraint}

\begin{verbatim}
SeparationConstraint (  separation_constraint_list ) ;

separation_constraint_list  is a comma-separated list of
              separation_constraint
              

separation_constraint : declarator | nodeset 

declarator = SOME_NAME

nodeset = gridset | gridset nodeset_extension |
          declarator DOT declarator |
          declarator_nodeset_extension
          
declarator_nodeset_extension : declarator nodeset_extension

nodeset_extension: node_type_set_specifier node_index_set_specifier

node_type_set_specifier: DOT LAYER ( node_type_set_specifier_clause )
\end{verbatim}

nodeset - Sect.\ref{sec:NodeSet-GSL}

\subsection{access to a set of nodes}
\label{sec:NodeSet-GSL}


A set of node instances can be access (at the GSL level) using
\begin{verbatim}
instance-composite '/' instance-grid[].Layers(layer-name)[]

instance-composite '/' instance-grid[].Layers(layer-name)[node-indices]
\end{verbatim} 
To access nodes at C++ code level, read Sect.\ref{sec:doNodeInit-TissueFunctor}.

To access node set in GSL: follow the following examples.

Example: take the nodeset in one layer from the current grid
\begin{verbatim}
Grid MYGRID{
  Layer(branches, ...) ; 
  
   //to access all node sets in that layer
  .[].Layer(branches)
}
\end{verbatim}
with \verb!branches! is the layer-name defined in one \verb!Layer! statement
(Sect.\ref{sec:Layer-as-Branch}).

Example: take the nodeset in one layer (in the current grid) with a given
criteria in region
\begin{verbatim}
// Suppose the grid is defined with 2D dimension
// Dimension(100, 100)

  // to access only nodes in that layer and fall within the given range 
.[25:75, 25:75].Layer(branches)
\end{verbatim}
This is the example being used in Game of Life
(Sect.\ref{sec:GameofLife-GSL-InitNodes}).

Example: we can define a name to represent that set of nodes (to be easily
reused in GSL)
\begin{verbatim}
NodeSet myNodes(.[25:75, 25:75].Layer(branches));

NodeSet all(.[].Layer(branches));
\end{verbatim}


Example: the set of nodes can comprise nodes from different Layers. 
\begin{verbatim}
NodeSet all(.[].Layer(branches), .[].Layer(layer2))
\end{verbatim}

Example: use Composite (Sect.\ref{sec:Composite})
\begin{verbatim}
Grid GridA{
};

Grid GridB{
};

Composite World{

  GridA grida;
  GridB gridb;
  
  NodeSet all(grida[].Layer(layer_inA))
};

Composite Universe{
   World world;
   GridA grida;
   
   NodeSet all(world/grida[].Layer(layer_inA), grida[].Layer(layer_inA));
}
\end{verbatim}

\subsection{define a trigger}
\label{sec:trigger-define-in-GSL}

A trigger is some event (i.e. function) that get called
at a criteria is met. Such criteria is checked at a certain runtime phase
\begin{verbatim}
// We use modulo operator (!%) to determine the operator
//  and the criterio  is checked whenever the 'solveChannels' 
//              phase is evoked
UnsignedTrigger recOn ("Iteration Trigger : !%" STR(RecordInterval),
                         ::Iteration, "!%", RecordInterval, 0, solveChannels);
                         
// the event which is evoking 'somaVoltageDisplay.dataCollection()' method
// is now associated with this trigger 'recOn'
//  .. we can add as many events as we want to a single trigger
//  .. they are stored in the std::vector<TriggerableCaller*>
                         
VoltageDisplay somaVoltageDisplay<fileName= PathFileCombine("somaV.dat")>;
somaVoltageDisplay.dataCollection(<>) on recOn;  

CalciumDisplay somaCalciumDisplay<fileName= PathFileCombine("somaCa.dat")>;
somaCalciumDisplay.dataCollection(<>) on recOn;

\end{verbatim}
Check TriggerableCaller - Sect.\ref{sec:TriggerableCaller}.

Here, 2 Triggerable nodetypes are declared
\begin{verbatim}
  // 
Trigger UnsignedTrigger(string description, Service svc, string operator, 
                    int criterion, int delay, string PhaseName);

Trigger CompositeTrigger(string description, Trigger triggerA, int critA, 
                         string operator, Trigger triggerB, int critB, 
                         int delay, string phaseName);

\end{verbatim}
It means that if 
\begin{itemize}
  \item the comparison of a given Service
\verb!svc! which returns a data of given type (e.g. unsigned int for UnsignedTrigger)
to a \verb!criterion! value using the operator defined via the string
\verb!operator! (e.g. ``==``, ``<=``, \ldots) returns TRUE
\end{itemize}
then some-event (which is the method given via the string \verb!PhaseName! to be
invoked) will be triggered, after a certain delay (in terms of number of
iterations) \verb!delay!.

See the example of using \verb!::Iterator! (Sect.\ref{sec:Iteration})

Once a Triggerable nodetype is declared, we now can define a object of that
nodetype. NOTE: We can combine multiple condition using \verb!||! (OR),
\verb!^^! (XOR), \verb!&&! (AND). Remember to pass information based upon the
order
\begin{enumerate}
  \item a string telling some descriptive information about the triggerable
object 

  \item a service (Sect.\ref{sec:Service} that returns a value to be used as the
  left-hand side operand of the below operator
  
NOTE: \verb!::! (two colons) means queriable path at the Simulation level.

  
Example: \verb!::Iteration! is a special service that returns the current number
of iteration (check SimulationPublisher.C file).
\begin{lstlisting}
Service * s = new GenericService<unsigned>(_ptrPublishable, &(_sim._iteration));
_services.push_back(s);

_name = 
\end{lstlisting}
This is mapped to the \verb!Simulation::getServiceName(void* data)! method
- Sect.\ref{sec:Simulation-class-GSL}.
  
  
  \item the string representing the operator, e.g. \verb!==!, \ldots
  
  \item the right-hand-side operand of the operator above
  
  \item the name of the RuntimePhase to be evoked before the simulation stops.
\end{enumerate}



\subsection{record data}
\label{sec:TriggerFunction-define-in-GSL}

Data can be recorded by defining the recording protocol as a Variable type,
and connect to the Node type. Then, we connect them together.

This Variable type must have a TriggeredFunction method
(Sect.\ref{sec:TriggeredFunction}), that only called at a certain time
instance based on a given condition; and once is called, it performs the writing
data to buffer/file. Then the important step is to tell the system, when to call
this method, which is achieved by mapping it to a given Trigger
(Sect.\ref{sec:trigger-define-in-GSL}).

Example: in MDL 
\begin{verbatim}
Variable VoltageDisplay
{

  TriggeredFunction dataCollection();
  
}
\end{verbatim}

Example:
\begin{verbatim}
#define _STR(x) #x
#define STR(x) _STR(x)
#define RecordInterval 10 // every 10 time-steps

// We use modulo operator (!%) to determine when it is triggered 
UnsignedTrigger recOn ("Iteration Trigger : !%" STR(RecordInterval),
                         ::Iteration, "!%", RecordInterval, 0, solveChannels);

 //declare variable type
VariableType  VoltageDisplay { initialize -> initialize1};

  // define the object
VoltageDisplay vmDisplay<fileName = "out/somaV.dat">;
  // define when the method is called
vmDisplay.dataCollection(<>) on recOn;
\end{verbatim}
We can define it to run at every given number iterations. 
Here, the criterion is checked after the RuntimePhase \verb!solveChannels! is
called; and if the criteria is met; then vmDisplay.dataCollection is called.


\subsection{create a service and connect to the simulation}
\label{sec:service-how-to-declare-in-GSL}

First we need to understand what is a service (Sect.\ref{sec:Service}).
A service is mainly used with a Trigger (Sect.\ref{sec:Trigger}).

A service is declared in GSL with or without a name. With a name, it can be
reused any many location. Example of using without a name that is
\verb!::Iteration! service. Here, \verb!Iteration! is a 'declarator', and
\verb!::! (double colon) refers to the scope of the declarator which is inside
the current Simulation object - Sect.\ref{sec:Simulation-MGS} (See above).

The GSL grammar for defining a named service

\begin{verbatim}
SERVICE declarator '(' service ')'

// with token SERVICE = 'Service'
// and declarator is any name we want to use
// example:
Service CurrentIteration = ( ::Iteration )
\end{verbatim}

which (at the C++ parser level) 
\begin{enumerate}
  \item  creates a \verb!C_declaration_service!
  (Sect.\ref{sec:C_declaration_service}).
  
  \item mapping to {\it service} 

The grammar on how to access to a \verb!service!
\begin{verbatim}
service := query_path_product 
         | declarator ',' string_literal_list

\end{verbatim}

In either way, a \verb!C_service! object is created (Sect.\ref{sec:C_service}),
with 
\begin{enumerate}
  \item  \textcolor{red}{case 1}
  
\begin{verbatim}
service := query_path_product

query_path_product --> is stored as --> C_query_path_product* _queryPathProduct
\end{verbatim}  
Sect.\ref{sec:C_query_path_product}, and it returns a service indirectly when
\verb!internalExecute()! of \verb!C_service! is evoked by the system.

The method \verb!internalExecute(LensContext* c)! generate the \verb!_service!
pointer
\begin{verbatim}
if (_queryPathProduct)
{
  _queryPathProduct(c);
  _service = _queryPathProduct->getService();
}
\end{verbatim}

Example: 
\begin{verbatim}

//NOTE: The use of '::' refers to the current Simulation object
// then ::somename call the service with name 'somename'
query_path_product := query_path DOUBLE_COLON declarator 
                   | DOUBLE_COLON declarator
//NOTE: declarator := IDENTIFIER 
//       which is interpreted as the service name

query_path  := query_list
             | repname   query_list
//NOTE: a list is elements separated by COLON (:)

query_list := COLON query
           | query_list : query
query := query_field_entry
      |  'string_literal_list'
      |  constant
      |  declarator
query_field_entry := ( name=data)
           
           
// NOTE: 
//  IDENTIFIER can be instance of a 'Composite' or 'Grid'
//  repname = representative name
repname := preamble IDENTIFIER
         | DOT IDENTIFIER  (e.g. .identifier)
         | IDENTIFIER
preamble := DOT SLASH
         | IDENTIFIER SLASH
         | preamble IDENTIFIER SLASH 


repname_query_list := 

query_list :=                    
\end{verbatim}

  
  \item \textcolor{red}{\bf case 2}: a service reference to a data member from
  another Graph-component, e.g. Variable.
   
\begin{verbatim}
service := declarator ',' string_literal_list


declarator --> is stored as --> C_declarator* _declarator
string_literal_list --> is stored as --> a service descriptor 
\end{verbatim}


Example: define a service name 'someName' which references to data member
'currentTime' in the object 'simInfo' which is of Variable type SimulationInfo
- Sect.\ref{sec:SimulationInfo}.

\begin{verbatim}
Variable SimulationInfo < recordIntervalInTime=20.0>; //ms
//or Variable SimulationInfo < recordIntervalInIterations=20>; //iteration count

Service someName ( simInfo, "currentTime");

// NOTE: We define Variable 'SimulationInfo',
//    with data member 'currentTime'
//    get updated at every iteration
//    and returns the current time of the simulation 
//     (based on the time-step) 
//     (not the computer-running time)
\end{verbatim}
 

\end{enumerate}

SUMMARY:
\begin{verbatim}
declarator --> is stored as --> C_declarator* _declarator
service    --> is stored as --> C_service*    _service
\end{verbatim} 
after that statement in GSL, everytime we use the named service 'declarator',
its \verb!internalExecute()! method is evoked.


NOTE: Sect.\ref{sec:NodeSet-GSL} discuss nodeSet
\begin{verbatim}
//NOTE: declarator := IDENTIFIER (represents either 
            a nodeSet, 
            an edgeSet, 
            a variableSet)
//NOTE: string_literal_list := a space-separated set of strings
//       whose combination (single string) is interpreted as 
         the name of service (or indeed the name of the data member)
\end{verbatim}


The method \verb!internalExecute(LensContext* c)! generate the \verb!_service!
pointer
\begin{verbatim}
{
  _declarator->execute(c);
  DataItem* di = const_case<DataItem*>
      (c->symTable.getEntry(_declarator->getName()))
  
  NodeSetDataItem* nsdi = dynamic_cast<NodeSetDataItem*>(di);
  ...
}
\end{verbatim} 


\end{enumerate}



A service can be connected to any Type using the functor
\verb!ServiceConnectorFunctor()! - Sect.\ref{sec:ServiceConnectorFunctor}.


\subsection{what to do inside Layers}

Sect.\ref{sec:layer-gsl-script}

\subsection{-- get the nodeset}
\label{sec:lenscontext-get-nodeset}

Inside the function that handle the Layer statement, we can get access to the
set of node using
\begin{verbatim}
LensContext *c;

NodeSet *nodeset = c->layerContext->nodeset;
\end{verbatim}
Sect.\ref{sec:NodeSet}


\subsection{what to do inside NodeInit}
\label{sec:doNodeInit-TissueFunctor}
\label{sec:TissueFunctor-doNodeInit}

REMEMBER the Syntax of InideNodes - Sect.\ref{sec:InitNodes}.

\begin{verbatim}
	InitNodes ( .[].Layer(branches), tissueFunctor("NodeInit", <
				compartmentalize = { "Vnew", 
				"Vcur", 
				"Aii", 
				"Aim", 
				"Aip", 
				"RHS", 
				},
				Vnew = {-85.00}//-64.1235}
				> ) );

\end{verbatim}


In the context of \verb!NodeInits! statement (in GSL), the function to
handle that task in the C++ code can access to the node-instances using

\begin{lstlisting}
void SomeFunctor::doNodeInit(LensContext * lc) 
{
 assert(_params.get());
 
  //1. get the 'set' of nodes to be initialized
  NodeSet* nodeset = lc->layerContext->nodeset;

  // REMEMBER: this set of nodes can come from different layers
  // we need to work with nodes in a particular at a time
  
  //2. get to the handle of the layer (e.g. gld)
  //get to the layer data
  std::vector<GridLayerDescriptor*> const& layers = nodeset->getLayers();
  assert(layers.size() == 1);  //we just want to limit only 1 layer is accepted
  std::vector<GridLayerDescriptor*>::const_iterator gld = layers.begin();
  
  //4. extract the nodes, in the form of a vector of nodes, for that given layer
  std::vector<NodeDescriptor*> nodes;
  nodes.clear(); //to be safe
  nodeset->getNodes(nodes, *gld);
 
 
  // 5. then we can traverse
  std::vector<NodeDescriptor*>::iterator node, nodeEnd;
  node = nodes.begin();
  nodeEnd = nodes.end();
  
  std::auto_ptr<ParameterSet> initPset;
}
\end{lstlisting}

What we want can do in C++
\begin{enumerate}
  \item get to the list of all nodes to be initialized (first argument of
  InitNodes statement)
  
  \item 
\end{enumerate}

\section{GSL (*.gsl) - LENS}
\label{sec:GSL}
\label{sec:nts-graph}

GSL is the new name for LENS framework.
The complete code structure in the form of folder organization is discussed in
Sect.\ref{sec:GSL-code-folder}. 

%\subsection{????Graph in Grid}

A graph is put inside the world in the form of a Grid. In other words, we have a
single Grid in a simulation.


\subsection{How to write script}
\label{sec:GSL-script-structure}


The *.gsl file is a script written using the context-dependent language defined
in \verb!./gsl/framework/parser/flex/speclang.l! file (Sect.\ref{sec:speclang.l}).

It contains
\begin{enumerate}
  \item Comments use C/C++ - style. 
  
  \item definition - Sect.\ref{sec:gsl-definition-section}

  the definition in a hierarchical manner to help defining 
  one or more repertoire (Sect.\ref{sec:GSL-repertoire})
  
  definition of functor (Sect.\ref{sec:GSL-script-functor}) or type
  
  \item declaration - Sect.\ref{sec:gsl-declaration-section}
  
  \item directive - Sect.\ref{sec:gsl-direftive-section}
  
  (optional) include codes defined in external file
  
We can use the C macro-preprocessor commands. Thus, it will be possible to
include other files, have conditional specifications, etc. All preprocessor
commands are preceded by a \verb!#!). 
  
\end{enumerate}


Complete keywords:
\begin{verbatim}
int
string
float
list
Complex

Type
RefPointGen

SamplingFctr1
SamplingFctr2
SubNodeSetFctr

Layout


NodeInitializer
InitNodes
EdgeInitializer     (obsolete)
InAttrInitializer


NDPairListFunctor
NDPairList
NDPair

Pset

Grid
Composite

Connector

ConnectionScript

GridCoord

RepName

NodeSet

ConstantType

VariableType

NodeType

NodeTypeSet

EdgeType

ToolType

Struct

Member
TypeDef

Initialize

Function

Functor

Dimension

NodeIndex

EdgeIndex

Default

NodeInit

Out

EdgeInit
EdgeSet

Port

system

Pause

Stop

InitPhases

RuntimePhases

LoadPhases

FinalPhases

SeparationConstraint

on

In

Matrix

Layer

Trigger

GranuleMapper


/     (SLASH)
.     (DOT)
...   (ELLIPSIS)

||      (or)
^^      (XOR)
&&      (and)
==      (test equivalent)
!=      (test not equivalent)

->
:
::
:::

-        (minus)
+

\end{verbatim}

\begin{enumerate}
  \item Stop - Sect.\ref{sec:GSL-Stop}
  \item SeparationConstraint - Sect.\ref{sec:GSL-SeparationConstraint}
\end{enumerate}

\subsection{(1) Definition: define a new type or a new functor}
\label{sec:gsl-definition-section}

\subsection{ * Type: Grid or Composite}

A type here can be a Grid or a Composite repertoire

\begin{verbatim}
type_definition: GRID declarator '{' grid_definition_body '}'  
	| GRID declarator '{' grid_definition_body '}' declarator 
    | COMPOSITE  declarator '{' composite_definition_body '}' 
    | COMPOSITE  declarator '{' composite_definition_body '}' declarator;
\end{verbatim}
Grid: Sect.\ref{sec:GSL-repertoire-grid}, and Composite:
Sect.\ref{sec:GSL-repertoire-composite}

\subsection{ * Functor}
\label{sec:GSL-script-functor}
\label{sec:functor-using-functor-in-GSL}

A functor can be defined in a few different ways
\begin{verbatim}
functor_definition: functor_category declarator ;
            | functor_category declarator '(' parameter_type_list ')' ';'
        	| complex_functor_definition   ';'
        	| connection_script_definition ';'
;
\end{verbatim}
With \verb!functor_category! must be one of the 11 names defined in
Sect.\ref{sec:functors-in-MGS-framework}.

\subsection{ * -- functor-categories}

Example: first case of definition
\begin{verbatim}
<Category>   Functor_Class_Name();
\end{verbatim}
with \verb!<Category>! can be one of the 11 different categories (or basic
functor class) - Sect.\ref{sec:functors-in-MGS-framework}), depending on how
\verb!Functor_Class_Name! class was derived which basic functor class.

Example: in std.gsl file (if the functor can accept more than one way to create instances, we need to use one or many \verb!Initialize! statememts)
\begin{verbatim}
NDPairListFunctor BindName
{
	Initialize(string  name, Functor value);
	Initialize(string name, float value);
};

Functor Print;
\end{verbatim}
Here, we signal that we will use BindName functor - Sect.\ref{sec:BindNameFunctor}, and Print functor (Sect.\ref{sec:PrintFunctor}).

Once a functor is declared, i.e. giving its category and its interface (i.e. how
to use it), we can use it as a type to create as many instances as we need, e.g.
\verb!IN! instance of BindName type, and \verb!print! of Print type.
\begin{verbatim}
BindName IN("weight", UniformDist(1.0, 2.0));

Print print();
\end{verbatim}
An instance of a functor, can be used as a function at different location, i.e. we can pass argument when calling that instance.

Once an instance to that functor is defined, we can use that at different locations in the GSL file
\begin{verbatim}
print("Hello world");

print("an error at this location");
\end{verbatim}


\subsection{ * -- connection-script}
\label{sec:connection-script}

Remember that we can connect two node sets, using directly a connection functor,
in that we need to use explicitly the different arguments (Sect.\ref{sec:connector-functor}).

However, we can also parameterize these arguments, in the form of a connection
script, so that we can reuse again this connection statement, each time with
possibly using a different argument's value.

\begin{verbatim}
connection_script_definition:  CONNECTIONSCRIPT declarator '(' parameter_type_list ')' '{' connection_script_definition_body '}' 
;


connection_script_definition_body:  connection_script_declaration 
        | connection_script_definition_body  connection_script_declaration 
        |

connection_script_declaration: declaration 
        | directive
        | RETURN  directive
\end{verbatim}


\subsection{BindName}
\label{sec:BindNameFunctor}

BindName functor is defined inside \verb!gsl/framework/functors/src/!, with sources
\begin{verbatim}
BindNameFunctor.C
BindNameFunctorFactor.C
BindNameFunctorType.C
\end{verbatim}
As part of the framework, so there is no need to write MDL for the Functor.

Example:
\begin{verbatim}
NDPairListFunctor BindName
{
	Initialize(string  name, Functor value);
	Initialize(string name, float value);
}
\end{verbatim}
This functor bind the name to a value, which can be a \verb!float! value or a value as 
returned from a functor.

\begin{verbatim}
 BindName branchInit(    "compartmentalize", { "Vnew",
           "Vcur",
           "Aii",
           "Aim",
           "Aip",
           "RHS",
           },
           //"Vnew" , FloatArrayMaker(UniformDist(-80.0, -40.0),1),
           //"Vnew" , FloatArrayMaker(UniformDist(Vrest_value, Vrest_value),1),
           "Vnew" , {Vrest_value},
           "Cm", V_CM,             // pF/um^2
           "gLeak",V_GLEAK         // nS/um^2, Hines used 0.0003 S/cm^2
      );

BindName junctionInit(  "Vnew" , FloatArrayMaker(UniformDist(-80.0, -40.0),1));
\end{verbatim}


% Sect.\ref{sec:Functor-MGS}
% describes how a 

%Functor refers to function object and is widely used in NTS. 
\subsection{Accessing a functor via C++ code}
\label{sec:functor-access-via-C++-code}

A functor is represented as \verb!FunctorDataItem! in GSL framework. Then to get
the functor
\begin{verbatim}
FunctorDataItem *fdi = ...;
Functor* functor = fdi->getFunctor();
\end{verbatim}

IMPORTANT: There are different types of Functor, each type is designed to be
used for a certain purpose only (sect.\ref{sec:Functor-MGS}).
% \begin{verbatim}
% NodeInitializerFunctor
% LayoutFunctor
% \end{verbatim}

Some built-in functors are defined in Sect.\ref{sec:functors-in-MGS-framework}. 
User can define new functor by writing C++ class as a derived class from one of
these class. User can define new functor in MDL, and the mdlparser will help
generate the C++ class (stored in framework/extension/functor) from that user
can add the body implementation.


\subsection{(2) Declaration: declare an identifier}
\label{sec:MGS-grammar-keywords}
\label{sec:gsl-declaration-section}

\begin{verbatim}
declaration: INT declarator '(' INT_CONSTANT ')'  
    | INT declarator '=' INT_CONSTANT         
    | FLOAT declarator '('constant ')'   
    | FLOAT declarator '=' constant
    | PSET '<' parameter_type_pair '>' declarator '(' nvpair_clause_list ')' 
    | PSET '<' parameter_type_pair '>' declarator '(' declarator DOT set_operation '('  nvpair_clause_list ')' ')'
    | NVPAIR '(' nvpair_clause ')'
	| NVPAIRLIST declarator '(' nvpair_clause_list ')'
    | LIST '<' list_parameter_type '>' declarator '(' '{' argument_list '}' ')'
    | NODESET nodeset_declarator '(' nodeset ')'
	| NODETYPESET declarator '(' node_type_set_specifier_clause ')'
    | RELNODESET relnodeset_declarator '(' relative_nodeset ')'
    | GRIDCOORD declarator '(' gridset ')'
    | REPNAME declarator '(' repname ')'
    | NODETYPE declarator
    | EDGETYPE declarator
    | INDEXSET declarator '(' index_set ')'
    | matrix_type_specifier matrix_init_declarator
    | typedef_declaration
    | functor_declaration
	| set_operation_declaration
    | STRIDE declarator '(' stride_list ')'
	| repertoire_declaration;


functor_declaration: declarator functor_declarator '(' argument_list ')'
	| FUNCTOR functor_declarator '(' parameter_type_list ')';

declarator: IDENTIFIER;
functor_declarator: IDENTIFIER;

\end{verbatim}

Example:
\begin{verbatim}
int x(5);
float y = 5.5;
Pset<  ???   >   z ( ??? );

\end{verbatim}


\subsection{(3) Directive: functor or function}
\label{sec:gsl-direftive-section}

\begin{verbatim}
functor_specifier:   functor_declarator '(' argument_list ')';
function_specifier:  function_name '(' argument_list ')';
\end{verbatim}

Example:
\begin{verbatim}

\end{verbatim}


\subsection{(4) Error}
\label{sec:gsl-error-section}

If non of the above (e.g. definition, directive, declaration) matches, then an error occurs.
In that case, it is put in this category
\begin{verbatim}
| error ';' {
  // same as below
}
| error {
       LensLexer *l = ((LensContext *) parm)->lexer;

       cerr << "Error at file:" << l->currentFileName << ", line:"
            << l->lineCount<< ", " << "unexpected token: "
            << l->getToken() << endl << endl;

       CONTEXT->setError();
}
;
\end{verbatim}
%What is the language grammar for this???

\subsection{Data types}
\subsection{ * Primitive data types}

\begin{verbatim}
int
string
float
list
\end{verbatim}

\subsection{* Pair: NVPair or NDPair ?}
\label{sec:NDPair}

number-value pair NVPair is no longer used.


number-data pair NDPair is used in GSL in the form of 
\begin{verbatim}
 some_name '=' {somevalue[, somevalue2 [,...]}
\end{verbatim}

For a single value, we don't need \verb!{ }! (curly braces).

E.g.:
\begin{verbatim}
nodekind="CompartmentVariables[Voltage]"
\end{verbatim}

\subsection{* Set: PSet, NodeSet, NodeTypeSet, RelNodeSet ?}
\label{sec:PSet}

We use these keywords to define a group of something, e.g. 
\begin{verbatim}
PSet
NodeSet

NodeTypeSet

RelNodeSet
\end{verbatim}

\begin{itemize}
  \item \verb!PSet! is a keyword that indicate that it expects a
  ParameterSet - Sect.\ref{sec:ParameterSet} for assigning values to a node
  (Sect.\ref{sec:InitNodes}).

\begin{verbatim}
PSet(<param-type-pair>) identifier (NVPair-clause-list);

PSet(<param-type-pair>) identifier (identifier.set_operation(NVPair-clause-list);

    | PSET '<' parameter_type_pair '>' declarator '(' nvpair_clause_list ')' 
    | PSET '<' parameter_type_pair '>' declarator '(' declarator DOT set_operation '('  nvpair_clause_list ')' ')'

parameter_type_pair := declarator ','  init_attr_type_node

init_attr_type_node := IN | OUT | NODEINIT
\end{verbatim}  
Sect.\ref{sec:MGS-grammar-keywords}


  \item \verb!NodeSet! give a name for a set of nodes, so that we can
  conveniently use them everywhere
  
SYNTAX:
\begin{verbatim}
'NodeSet'  the-name '(' definition-node-set  ')'
\end{verbatim}  

Example:
\begin{verbatim}
Layer(nodes, ...) 

NodeSet all(.[].Layers(nodes);
\end{verbatim}
It means \verb!all! name refers to all nodes in the layer name \verb!nodes! in
the current Grid (Sect.\ref{sec:Grid}) (first dot '.' refers to the current
grid, then [] refers to )

 
  \item \verb!NVPair!: What is this?

\begin{verbatim}
NVPair(NVPair-clause)
\end{verbatim}

%  \item 
\end{itemize}


\subsection{* List?}
\label{sec:list-GSL}

A list is \verb!list<!typename\verb!>!.

A list in GSL is a comma-saparated list of NDPair.

\begin{enumerate}
  \item \verb!NVPairList!: What is this?

\end{enumerate}

\subsection{* Matrix?}

What dimension?

\begin{verbatim}
MATRIX '<' type_specifier '>'  identifier;
\end{verbatim}
with MATRIX maps to \verb!Matrix! keyword.

Example:
\begin{verbatim}
Matrix < int >   x;
\end{verbatim}

\subsection{Default value}

In GSL, any values undefined get zero by default.


\subsection{Identifier}

An identifier in *.gsl script is any thing start with a letter (L) and then with
one of the following forms
\begin{verbatim}
{L}{A}*

{L}[_]{A}*

{L}{A}*[_]{A}*

{L}{A}*{U}*

{D}{A}*{L}{A}*
\end{verbatim}

NOTE:
\begin{verbatim}
/* Digits, letters and alphanums */
D       [0-9]
L       [a-zA-Z]
A       [a-zA-Z0-9]
U       [a-zA-Z0-9_]
\end{verbatim}

\subsection{String literal}

Any thing wrapped by double quotes
\verb!''some string''!.

\subsection{Constant (integer, float)}

Integer constant, with/without a minus sign and digits after it.
\begin{verbatim}
\-?{D}+
\end{verbatim}

Float constant, can be in regular decimal form or scientific form
\begin{verbatim}
[+\-]?{D}+\.{D}* 

[+\-]?{D}+\.?{D}*[eE][+\-]?{D}+
\end{verbatim}

\subsection{Comment}

Use C/C++ style, \verb!#!

\subsection{Function}

A function is defined using \verb!Function! keyword, and can return some value
using \verb!Return! keyword.

\subsection{NodeType vs. Variable}
\label{sec:NodeType-GSL}
\label{sec:Variable-GSL}

A Variable is like a NodeType which can also goes through different phases, but
does not live in a grid. A Variable is a loner, i.e. come in and connect to a
specific set of NodeTypes instances to do (1) data I/O, (2) electrical
stimulation.


The difference between a NodeType and a Variable is that a NodeType is part of
the graph in the beginning, and a Variable is used to connected to one or more
Nodes in order to perform certain action, e.g. current stimulus, recording data
for I/O purpose, during a certain time period.


\subsection{Operators}

\begin{verbatim}
||      (or)
^^      (XOR)
&&      (and)
==      (test equivalent)
!=      (test not equivalent)

\end{verbatim}


\subsection{Repetoire: Grid + Composite}
\label{sec:GSL-repertoire}

A backbone of a GSL (*.gsl) script is defining the repetoire
which can be a Grid (Sect.\ref{sec:GSL-repertoire-grid}) or, for more complex system
simulation, a Composite (Sect.\ref{sec:GSL-script-Composite}).

The concatenated repertoire names will be delimited with the \verb!/! token.
Unix-style \verb!'.'! addressing will also be supported to allow relative
addressing. All repertoire names must begin with the delimiter, \verb!'.'!.

A fully qualified repertoire name can, therefore, be specified in three ways:
\begin{verbatim}
1)	A concatenation of repertoire names, according to the hierarchy structure,
starting with the root repertoire. 

Example: /Network/VOR/4Calpha

2)	A concatenation of repertoire names, according to the hierarchy structure,
starting with a named repertoire name specification that is globally accessible. 

Example: RepName VOR(/Network/VOR);


3)	A concatenation of repertoire names, according to the hierarchy structure,
starting with the repertoire in the current scope. 

Example: VOR/4Calpha
\end{verbatim}

NOTE: \verb!RepName!

\subsection{* Grid}
\label{sec:GSL-repertoire-grid}

\begin{verbatim}
type_definition: GRID declarator '{' grid_definition_body '}'  
	| GRID declarator '{' grid_definition_body '}' declarator 
    | COMPOSITE  declarator '{' composite_definition_body '}' 
    | COMPOSITE  declarator '{' composite_definition_body '}' declarator;

grid_definition_body: dim_declaration ';'
    | dim_declaration ';' grid_translation_unit;
\end{verbatim}
with \verb!dim_declaration! (Sect.\ref{sec:GSL-grid-dimension}),
\verb!grid_translation_unit! (Sect.\ref{sec:GSL-grid-body})

\subsection{ ** Grid dimension}
\label{sec:GSL-grid-dimension}


A dimension: a comma-separated list of constant integer value
  (typically 3D is used)
  
\begin{verbatim}
dim_declaration: DIMENSION '(' int_constant_list ')';
int_constant_list: INT_CONSTANT 
     | int_constant_list ',' INT_CONSTANT;
\end{verbatim}
with \verb!INT_CONSTANT! is a constant integer value.

Example:
\begin{verbatim}
#define _X_ 16
#define _Y_ 16
#define _Z_ 16

Grid Tissue {
   Dimension (_X_, _Y_, _Z_);

   ... // main body
}
\end{verbatim}


\subsection{ ** Grid main body}
\label{sec:GSL-grid-body}

Grammar:
\begin{verbatim}
grid_translation_unit: grid_translation_declaration_list;

grid_translation_declaration_list: 
      grid_translation_declaration
	| grid_translation_declaration_list  grid_translation_declaration;
	
grid_translation_declaration: 
      declaration ';'
    | grid_function_specifier ';';
    
grid_function_specifier: function_specifier
    | grid_function_name;
    
grid_function_name: 
      INITNODES '(' argument_list ')'
    | LAYER '(' argument_list ')';
\end{verbatim}

A Grid is defined with
\begin{enumerate}
  \item initialize nodes in the grid - Sect.\ref{sec:InitNodes}
  

  \item define one or more layers mapping to the grid: imagine it's like layers
  concept in Photoshop or GIMP, but the layer here has the same dimension of the
  grid. - Sect.\ref{sec:nts-layer}

\end{enumerate}

\subsection{*** InitNodes}
\label{sec:InitNodes}

In *.gsl script, we can have as many InitNodes statement as we want, each
statement initialize the values for a set of nodes.
{\small
\begin{verbatim}
// SYNTAX: 2 stages (create instance + use it) - InitNodes and NodeInitializer keywords.
// stage 1
NodeInitializer functor-of-NodeInitializer-category();
functor-of-NodeInitializer-category functor_instance();
// stage 2
InitNodes ( node-set, functor_instance)
\end{verbatim}
}
The C++ code that handles this statement is
\verb!C_grid_function_name::initNodes()! - Sect.\ref{sec:C_grid_function_name}


\textcolor{red}{InitNodes} statement in GSL requires minimum 2 arguments
\begin{enumerate}
  \item NodeSet = a set of Nodes to be initialized
 
There are different way to define a nodeset (Sect.\ref{sec:NodeSet-GSL}). 
We can define the border for a given layer within which the nodes should be set.  
  
  \item Subsequent arguments are functor instance of category
  \verb!NODEINITIALIZER! (i.e. derived from \verb!NodeInitializerFunctor!
  class), minimum one functor (Sect.\ref{sec:NodeInitializer-functors})
  
  Each functor does initialization on each Node in the NodeSet
\end{enumerate}

If a node is modified by two different InitNodes statement, then the last write
is kept.
 
Example: SameFunctor - Sect.\ref{sec:Same-functor}
\begin{verbatim}
NodeInitializer Same(Pset);  // functor that just provides parameter set
                             // argument

Grid grid_name{
   InitNodes ( .[25:75, 25:75].Layer(nodes), Same( Pset<LifeNode, NodeInit> ( <value = 1> ) ) );
   InitNodes ( .[40:60, 40:60].Layer(nodes), Same( Pset<LifeNode, NodeInit> (
   <value = 0> ) ) );
}
\end{verbatim}  



\subsection{*** Layers}
\label{sec:nts-layer}
\label{sec:layer-gsl-script}


You can also use condition \verb!#if! \ldots \verb!#endif! to  conditionally
choose to add more or less layers in a grid, i.e grid is a vector of layers.
Check Sect.\ref{sec:layers} for the  statement below allows us to add a layer
to a grid (Sect.\ref{sec:Grid}), underlying C++ implementation.

Example 1:
\begin{verbatim}
Layer(MegaSynapticClefts, MegaSynapticCleft, UniformLayout(1), 
          <nodekind = "MegaSC">);
\end{verbatim}

Example 2:
{\small
\begin{verbatim}
#define MAX_COMPUTE_ORDER 0

Grid Tissue {
   Layer(branches, HodgkinHuxleyBranch, tissueFunctor("Layout", <nodekind="Branches">), 
         <nodekind="Branches">, 
         tissueGM);
   Layer(endPoints, EndPoint, tissueFunctor("Layout", <nodekind="EndPoints">), 
         <nodekind="EndPoints">, 
         tissueGM);
#if MAX_COMPUTE_ORDER>0
   Layer(fwdSolvePoints1, ForwardSolvePoint1, 
          tissueFunctor("Layout", <nodekind="ForwardSolvePoints">), 
          <nodekind="ForwardSolvePoints">, tissueGM);
   Layer(bwdSolvePoints0, BackwardSolvePoint0, 
          tissueFunctor("Layout", <nodekind="BackwardSolvePoints">), 
          <nodekind="BackwardSolvePoints">, tissueGM);
#endif
}
\end{verbatim}
}

The \verb!Layer( )! statement in GSL is handled by the function
\verb!C_grid_function_name::layers()! (Sect.\ref{sec:C_grid_function_name::layers}).

, as represented by GridLayerDescriptor
(Sect.\ref{sec:GridLayerDescriptor}) at the code level, requires 4 or 5
arguments
\begin{enumerate}
  \item 1st = the name of the layer, any name defined by the user and must be
  unique in the simulation as it is used to retrieve the layer.
  
  This name provides an abstract representation to a set of instances, e.g.
  LifeNode (as provided by the 2nd argument), from a given CompCategory, e.g.
  LifeNodeCompCategory. The creation of such instances on the grid, is
  determined by the 3rd argument (a layout functor).
  
  Also, this name can be used to retrieve the reference to such layer -
  Sect.\ref{sec:get-to-layer-C++}.
  
  \item 2nd = the name of the NodeType (the name comes after NodeType keyword),
  each layer contains nodes from only one NodeType.
  
  \item 3rd = the layout functor (Sect.\ref{sec:functor-layout}) that 
  determine the densities, i.e. how many node instances to be created, on each
  grid element. NOTICE the dimension of the grid is determined by the
  \verb!Dimension! clause. 
  
%  Then,  of the given NodeType to each Granule.
  
  NOTE: On a multi-process (i.e. MPI implementation), the partition of the grid 
  depends on the type of Granule Mapper being used on the fifth argument.
  % what coordinate in the grid
  
  \item 4rd = a NDPairList - Sect.\ref{sec:NDPairList}

NOTE: A functor can accept one or many input via the form of a list of
(name=data), known as {\bf NDPairList} (NameDataPair-List)

  \begin{verbatim}
< nodekind = "Branches">

< nodekind = "Branches", someName = 10>

< nodekind = "Compartmental[Voltage][1]">
\end{verbatim}

To extract the information from a given 'name', e.d. \verb!nodekind!, we need to
write such function, e.g. in NTS we use 
\begin{verbatim}
TissueFunctor::getNodekind(const NDPairList* ndpl, 
       std::vector< std::string >& nodekind);kj
\end{verbatim}
Remember that here we define TissueFunctor (Sect.\ref{sec:TissueFunctor.mdl})
with the second argument is NDPairList. 


  \item 5th argument (optional) a granulemapper (Sect.\ref{sec:GranuleMapper}) -
  that decide how to partition the node instances for that layer into different
  MPI processes.
  
  \textcolor{red}{The default is VolumeGranuleMapper} -
  Sect.\ref{sec:VolumeGranuleMapper}
  
Check \verb!C_grid_function_name.C!
\begin{verbatim}
GranuleMapper * granuleMapper;

// c = LensContext
GranuleMapperType* gt = c->sim->getGranuleMapperType("VolumeGranuleMapper");

std::auto_ptr<GranuleMapper> apgm;
gt->getGranuleMapper(gmargs, apgm);
granuleMapper = apgm.get();

granuleMapper->setIndex(c->sim->getGranuleMapperCount());
c->sim->addGranuleMapper(apgm);
\end{verbatim}  

  Example: a GridGranuleMapper instance - Sect.\ref{sec:GridGranuleMapper}
  
%  Example: a TissueGranuleMapper instance - Sect.\ref{sec:TissueGranuleMapper}
\end{enumerate}


% On a layer, we can have one or more models (Sect.\ref{sec:Node-MDL}) connected
%to each other. 
% 
GRAMMAR:
\begin{verbatim}
Layer ( NodeTypeSetSpecifierClause );


primary_expression: MEMBER  '(' layer_set ')';
node_type_set_specifier: DOT TYPE '(' node_type_set_specifier_clause ')';
node_type_set_specifier_clause: layer_set
        	| logical_OR_expression;
layer_set: layer_entry
        	| layer_set ',' layer_entry;
layer_entry: layer_name
        	| name_range;
name_range: layer_name MINUS  layer_name;
layer_name: IDENTIFIER;




layer_entry :=
layer_name := STRING;
\end{verbatim}

Example:
\begin{verbatim}
 //define a particular NodeType, with its initialization  
NodeType LifeNode(< habitable=2, tooSparse=1, tooCrowded=4 >);

/*1. layer-name (anything)
  2. particular-NodeType-class-that-has-been-initialized 
      [NOTE: one layer only contains one NodeType
  3. a 'LAYOUT' functor that returns the density of 
      NodeTypes instances to each volume in the grid
  4. <nodekind = "">
  5. tisseGM (optional)? 
*/ 
Layer(nodes, LifeNode, UniformLayout(1), < nodekind="Nodes" >);
\end{verbatim}

Example:
\begin{verbatim}
InitPhase  = { method_name_in_GSL_1, method_name_in_GSL_2 }

NodeType HodgkinHuxleyVoltage (< some input argument >); 
{ method_name_C++_HodgkinHuxleyVoltage_class -> method_name_in_GSL_1 }

Layer(branches, HodgkinHuxleyVoltage, 
      tissueFunctor("Layout", <nodekind="CompartmentVariables[Voltage]">),
      <nodekind="CompartmentVariables[Voltage]">, 
      tissueGM
     );

\end{verbatim}


\subsection{* Composite}
\label{sec:GSL-script-Composite}
\label{sec:GSL-repertoire-composite}


\subsection{QueryPathProduct}

%TODO: ask James what is QueryPathProduct




\section{Error handling}

\begin{verbatim}
std::string mes = "Some error message";

throwError(mes);
\end{verbatim}

\section{I/O }
\label{sec:NTS-IO-classes}
\label{sec:GSL-IO-classes}

To provide convenient I/O for a class using \verb!std::cout!, we implement
\begin{verbatim}
 // header file
#include <iostream>
std::ostream & operator<<(std::ostream&,GridSet const&);

class GridSet{
public:
  // 1. make sure we implement this
 virtual std::ostream& print(std::ostream &os, GridSet const &gs) const;


}

  // source file
  // 2. make sure we 
std::ostream & operator<<(std::ostream& os,GridSet const& gs)
{
   return gs.print(os,gs);
}

  // 3. and implement GridSet::print() 
std::ostream & GridSet::print(std::ostream & os, GridSet const& gs);
{
  // add output to 'os'
  os << something;
  
  os << someotherthing;

  return os;
}
\end{verbatim}

\section{RNG (Random Number Generator)}
\label{sec:RNG}

Reference: \url{http://www.stat.osu.edu/~herbei/GPU/RNG.pdf}


\subsection{RNG.h}
\label{sec:RNG.h}

This header contains the static object which is used for generating random
numbers. The methods to generate random numbers are in a different header file -
Sect.\ref{sec:rndm.h}.


\begin{verbatim}
./gsl/utils/std/include/RNG.h
\end{verbatim}
which currently uses
\begin{lstlisting}
typedef MRG32k3a RNG;
//typedef MersenneTwister RNG;


static RNG Rangen;
static RNG SharedRangen;
static RNG& getRangen() {
      return Rangen;
};
static RNG& getSharedRangen() {
      return SharedRangen;
}
\end{lstlisting}


\subsection{-- MerseneTwister}
\label{sec:MerseneTwister}

It provides 2 methods
\begin{itemize}
  \item \verb!irandom32! : return unsigned long
  
  \item \verb!drandom32! : return double
\end{itemize}

\subsection{Multiple Recursive Generator: MRG32k3a}
\label{sec:MRG32k3a}

\url{https://software.intel.com/en-us/node/590402}

\url{http://simul.iro.umontreal.ca/rng/MRG32k3a.c}


\subsection{rndm.h}
\label{sec:rndm.h}


These APIs accepts an object of type \verb!RNG! (Sect.\ref{sec:RNG}), and
return a random number following different distribution

\begin{enumerate}
  \item \verb!drandom()!: floating-point uniform
  
  \item \verb!lrandom()!: long uniform
  
  \item \verb!irandom()!: integer uniform
  
  \item \verb!gaussian()!: Gaussian with mean 0 and std = 1
  
  \item \verb!gaussian(mean,std)!: 
  
  \item \verb!expondev()!: exponential distribution
  
  \item \verb!expondev(g)!: normalized to $g$
  \item 
\end{enumerate}

\section{Environment variables}

\begin{verbatim}
LENSROOT
\end{verbatim}


\section{Macro testing}

\begin{verbatim}
#ifdef USING_BLUEGENE
 //code specific for BlueGene
#endif

#ifdef HAVE_MPI
 // MPI-code
#endif

#ifdef INFERIOR_OLIVE

#endif

#ifdef SYNAPSE_PARAMS_TOUCH_DETECT

#endif

#ifdef USING_CVC

#endif

#ifdef DEBUG_ASSERT

#endif

#ifdef DEBUG

#endif

#ifdef VELOCITY_DAMPING_FACTOR
  VELOCITY_DAMPING_FACTOR = 0.9 
#endif
\end{verbatim}


