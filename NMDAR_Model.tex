\chapter{NMDAR models}
\label{chap:NMDAR_models}

\def\syn{{\text{syn}}}
\def\decay{{\text{decay}}}
\def\rise{{\text{rise}}}
\def\nonNMDA{{\text{non-NMDA}}}
\def\NMDA{{\text{NMDAR}}}
\def\on{{\text{on}}}
\def\off{{\text{off}}}
\def\slope{{\text{slope}}}

NMDAR is discussed in Sect.\ref{sec:NMDAR}.
Functional mapping of neurotransmitter receptors requires rapid and localized
application of transmitter (Sect.\ref{sec:two-photon-uncaging}).

The current via NMDAR is often expressed in the form
\begin{equation}
I_\NMDA = g_\NMDA  \times F(V_\post,[\Mg]) \times r_\NMDA
  \times \left( V_\post - E_\rev \right)
\end{equation}
with $F(V_\post, [\Mg])$ is the $\Mg$-block function (e.g.
Sect.\ref{sec:NMDAR-Mg-block}), $r_\NMDA$ is the function reflecting the
activation caused by pre-synaptic neurotransmitter release, as in the
form of presynaptic potential depolarization (Sect.\ref{sec:synaptic-efficacy-post-synaptic-factors}).

The fact that $I_\NMDA$ is only large if both $r$ is activated and the
post-synaptic membrane potential $V_\post$ is depolarized simultaneously.
\url{http://www.johndmurray.org/materials/teaching/tutorial_synapse.pdf}


\section{Scheme: 1 agonist T of two binding sites}

Several kinetic schemes are based on the following state diagram, with a single
agonist T but two different binding sites following sequential binding strategy:

\begin{equation}
\begin{split}
\ce{ C_0 <=>[R_bT][R_u] C_1 <=>[R_bT][R_u] C_2 <=>[R_o][R_c] O} \\
\ce{ C_2 <=>[R_r][R_d] D}
\end{split}
\end{equation}

\begin{itemize}
  \item D = desensitized form of the receptor
  
  \item single unbinding rate $R_u$
\end{itemize}

The whole-cell recorded NMDAR current, in free $\Mg$ condition, suggested:
$R_b = 5$ (1/($\muM$.sec)), $R_u = 12.9$ (1/sec); $R_d = 8.4$ (1/sec), 
$R_o = 46.5$ (1/sec) and $R_c = 73.8$ (1/sec).

The ionic current
\begin{equation}
I_\NMDA = \bar{g_\NMDA} \times B(\Vm) \times f_{[O]} \times (\Vm - E_\NMDA)
\end{equation}
with $\bar{g_\NMDA}$ for single-channel suggested to be in the 
range 0.01-0.6 nS (based on the assumption that conductance of dendritic NMDAR
is between 3\% to 62\% of AMPAR (Zhang-Trussell, 1994; Spurston, JOnas, and
Sakmann, 1994)).


In cortical pyramidal neurons, NMDAR is found; and they are quite slow with rise
time of 20 ms  and  decay  times  of 25 to 120 ms; compared to AMPAR
(Sect.\ref{sec:AMPAR-kinetics}). Here, $f_{[O]}$ is also modeled as a difference
of exponentials (Sect.\ref{sec:difference-exponentials}).

%\section{$\Mg$ block}


\section{Mg-depenent block}
\label{sec:NMDAR-Mg-block}

\begin{enumerate}
  \item Jahr-Stevens (1990) - Sect.\ref{sec:NMDAR-Mg-block-Jahr-Stevens-1990}
  
  \item Ascher, Nowak (1988) - Sect.\ref{sec:NMDAR-Mg-block-Ascher-Nowak-1988}
  
  \item Major-Tank (2008) - Sect.\ref{sec:NMDAR-Mg-block-Major-Tank-2008}
\end{enumerate}

\citep{fino2010} showed that (in MSN at least) it needs 30ms of suprathreshold
depolarization of current injection for removing $\Mg$ block.
They tested with 5ms suprathreshold depolarization (Fig.3D,E) it is not enough
to generate LTP. To remove $\Mg$ block , we need either longer suprathreshold
depolarization or subthreshold depolarization long  enough (Fino et al.
2009b). In vivo, cortical or thalamic activity can induce a spike in MSN (Stern,
Jaeger, Wilson 1998).


\begin{equation}
I_{NMDAR} = B(V) \times I_{whatever formula}
\end{equation}

example:
\begin{equation}
I_{whatever formula} = \bar{g_z} (h - m) (V_m - E_z)
\end{equation}
with $z$ is the ion species permeated through NMDA (which can be $\Ca$ or $\K$).

% \begin{equation}
% B(\Vm) = \frac{1}{1 + \frac{[\Mg]_o}{3.57 \text{mM}} \times \exp(-\Vm \times
% 0.062 \text{mV}^{-1})}
% \end{equation}
% NOTE: $[\Mg]_o$ has to be in mM unit, and $\Vm$ has to be in mV.

\section{Ascher, Nowak (1988)}
\label{sec:NMDAR-Mg-block-Ascher-Nowak-1988}

The $\Mg$ block is modeled with forward and backward rate constants based on
Ascher, Nowak (1988) - Sect.\ref{sec:NMDAR-Mg-block}

The given rates were designed at 22$^\circ$C:

\def\ms{{{\text{ms}}}}
\def\mV{{{\text{mV}}}}
\begin{equation}
\begin{split}
\alpha &= A_\alpha \times e^{\frac{\Vm}{V_\text{spread,$\alpha$}}} \qquad \;
A_\alpha = 5.4 \ms^{-1}; V_\text{spread,$\alpha$} = 47 \mV  \\
\beta &= A_\beta [\Mg] \times e^{\frac{\Vm}{V_\text{spread,$\beta$}}} \qquad \;
A_\beta = 0.61 \ms^{-1}; V_\text{spread,$\beta$} = 17 \mV 
\end{split} 
\end{equation}

\section{Jahr-Stevens (1990)}
\label{sec:NMDAR_Jahr-Stevens-1990}

\citep{jahr1990b} first developed a single channel scheme, and then
\citep{jahr1990} used this model to predict the macroscopic current of NMDAR at
different $V_m$ and $[\Mg]_o$ concentrations.

\citep{jahr1990}  studied the $V_m-$dependence of NMDAR in the presence of
different concentration of $[\Mg]_o$ (from 1$\muM$ to 10 mM).
\begin{enumerate}
  \item The physiological $[\Mg]_o$ is near 1 mM.

  At this condition: NMDAR is not opening if post-synaptic $V_m$ is more
  negative than -80 mV; and open if $V_m = +20$ mV.


\end{enumerate}
At the single-channel level, the addition of external Mg alters single-channel
openings from long-lived events to many very short events grouped into bursts of
openings. These bursts apparently result from short interruptions of current
flow during periods when the channel is in the open configuration.
The frequency of these interruptions is directly related to Mg concentration.


At single-channel behavior, \citep{jahr1990b} showed that
a model of NMDAR of 3 or 4-state can accurately model the behavior
\begin{itemize}
  \item 1 open state
  \item 1 closed state
  \item 1 or 2 'blocked' state

  The first 'blocked' state is $\Mg$-dependent.
  The second 'blocked' state is $V_m$-dependent and $\Mg$-independent that is
  required to explain short closed state (especially at low $[\Mg]$).

\end{itemize}

Scheme 1: fails (1) (expected to, but did not work) to show open burst (state
transition between O and B) lengthen in proportion to Mg concentration, (2)
to predict $V_m$-dependent of macroscopic currents.

\begin{equation}
\ce{C <=>[\text{neurotransmitter-bound}] O <=>[\text{binding of Mg}][V_m
\text{ depolarize}] B}
\end{equation}

\textcolor{red}{\bf Scheme 2}: 4 states with two Blocking states B1
($V_m$-dependent), B2 ($\Mg$-dependent)
\begin{verbatim}
     a1           a2
B1------\ O -----------\ B2
 \  b1     \    b2      /
  \___      \         /
      \     A\       /B2
     B1\     _\    /
        \_____  C
\end{verbatim}
\begin{itemize}
  \item a1 = independent of $[\Mg]_o$

  \item a2 = linearly dependent of $[\Mg]_o$
\end{itemize}

\begin{eqnarray}
a_1 = \exp \left( -0.016 V_m - 2.91 \right)  \;\;\;\text{ms}^{-1} \\
a_2 = C \exp \left( -0.045 V_m - 6.97 \right) \;\;\;\muM^{-1}\text{ms}^{-1}\\
b_1 = \exp \left( 0.009 V_m + 1.22 \right) \;\;\;\text{ms}^{-1}\\
b_2 = \exp \left( 0.017 V_m + 0.96 \right) \;\;\;\text{ms}^{-1}\\
A   = \exp \left( -2.847 \right) \;\;\;\text{ms}^{-1}\\
B_1  = \exp \left( -0.693\right) \;\;\;\text{ms}^{-1}\\
B_2 = \exp \left( -3.101 \right) \;\;\;\text{ms}^{-1}
\end{eqnarray}
NOTE: For microscopic reversibility, the rate $\ce{C ->O}$ are updated
accordinly.


Input for parameter estimations
\begin{enumerate}
  \item mean open time

  \item mean interruption time

As $V_m$ becomes more positive, the rate of $a_1$ and $a_2$ for entering the
blocked states must become smaller, i.e. approaching zero.


  \item mean number of interruptions per burst: $n$

As $V_m$ becomes large and positive, then the number of interruption per burst
become smaller, i.e. $n$ approaches zero.

\end{enumerate}
The gating function $g(V_m)$ is the conductance contributed per channel burst
relative to the amount of conductance that a burst would give at very positive
$V_m$ (where very few interruptions occur). Then the number of opening in a
burst is (n+1), and the mean dwell time in the O state each time the NMDAR
enter is denoted as $t_O$. So, for a single burst, the total time NMDAR in the O
state is
\begin{equation}
(n+1) \times t_O
\end{equation}
So
\begin{equation}
g(V_m) = \frac{(n+1)\times t_O }{\text{ open time per burst for large} V_m}
\end{equation}
NOTE: Both $n(V_m)$ and $t_O(V_m)$ are functions of $V_m$.

When $V_m$ becomes very large and positive, i.e. $\ce{n -> 0}$, so the total
time in O state approaches $t_O(V_m)$, and the mean open time is
\begin{equation}
t_O = \frac{1}{\text{sum of out-going rates}} = \frac{1}{a_1 + a_2 + A}
\end{equation}


\subsection{Mg-block}
\label{sec:NMDAR-Mg-block-Jahr-Stevens-1990}

\begin{equation}
I_{NMDAR} = B(V) \times I_{whatever formula}
\end{equation}
with $B(V)$ is $\Mg$ block factor.

The blocking of $\Mg$ is extremely fast compared to other processes, and thus
can be modeled using instantaneous function of $\Vm$ as a sigmoid function, i.e.
$B(\Vm)$ increases from 0 to 1 when $\Vm$ depolarize.

NOTE: $[\Mg]_o$ has to be in mM unit, and $\Vm$ has to be in mV.
\begin{equation}
B(\Vm) = \frac{1}{1 + \exp\left( -0.062 \Vm \right) \frac{[\Mg]_o}{K_p}}
\end{equation}
with $[\Mg]_o$ is in the range 1-2 mM (physiological conditions); and
$K_p=3.57$ (mM). 


\subsection{technical issue}

As the transition C-O (from Closed to Open) is technically difficult to study,
they assumed that theses rates are not significantly influenced by $V_m$. So
for a fixed agonist concentration, the entire gating function depends only on
the fraction of time channels spend in the O rather than B states.

\section{Silver et al. (1992): mossy fiber-granule cell synapse}
\label{sec:mEPSC-mossy-fiber-granule-cell}

The mossy fiber-granule cell synaptic transmission (Sect.\ref{sec:mossy-fiber})
generates a mEPSC with 2 components: fast and slow. Spontaneous mEPSC occurs
with low frequency: 0.21$\pm 0.05$ Hz in unstimulated cells bathed in 0.5$\muM$
TTX (blocking Na+ current).

The authors \citep{silver1992} recorded cerebellar synapses and found that
\begin{itemize}
  \item time courses of miniature EPSC and evoked EPSC (Sect.\ref{sec:EPSC}) are
  the same
  
  \item mEPSC has an exceptionally fast non-MNDA component: rise time 200 $\mus$
  (with pre-filtered rise time < 100 $\mus$); and decay time constant
  $\tau=1.3\pm 0.1$ms.
  
This rise/decay times are in constrast to that of non-NMDAR component in
synapses in other brain regions of $9\pm 1$ msec; and decay time $\tau = 52\pm 5$ msec.
  
The fast non-NMDAR component has 180$\pm$30 pS conductance. 
After that is NMDAR current with $\sim$ 50pS  conductance.

  \item the fast non-NMDA component has a skewed amplitude distribution; and is
  blocked by non-NMDAR receptor antagonist CNQX and NBQX
  (Sect.\ref{sec:AMPAR-antagonist})
  
  \item the slow component is blocked by NMDAR antagonist APV (10-20$\muM$), or
  $\Mg$ (1mM) or 7-chlorokynurenate (5-10$\muM$)
  (Sect.\ref{sec:NMDAR-antagonist})
  
  \item the time-couse of slow AMDAR component is $\tau = 37\pm 10$ msec;
  
  
  \item the time-course in these synapses are different form synapse in other
  regions
  
\end{itemize}


The AMPA receptor-mediated synaptic currents were modeled by a dual exponential
function, one for the fast non-NMDAR component and one for slow NMDAR component
\begin{itemize}
  \item whole-cell recording: $\tau_\nonNMDA = 1.5$ms; $\tau_\NMDA = 52\pm 5$
  (ms); curren amplitude ratio: $A_\NMDA/A_\nonNMDA = 0.12 \pm0.01$
  
  \item perforated patch-clamp: $\tau_\nonNMDA = 2.0$ms; $\tau_\NMDA = 49\pm 5$
  (ms); curren amplitude ratio: $A_\NMDA/A_\nonNMDA = 0.15 \pm0.02$
\end{itemize}

Final result:
\begin{itemize}
  \item $\tau_\nonNMDA = 1.1 $ms; $\tau_\NMDA = 26$ ms;
  
  \item $A_\nonNMDA = 12.6$ (pA); $A_\NMDA = 2.7$ pA
  
  \item mEPSC has a bimodal distribution with 2 peaks: one at 12 pA (for
  non-NMDA component); and one at 22 pA (for both NMDAR + AMPAR components)
\end{itemize}

As NMDA current has exponentially rise time and a slower exponentially decay
time, using the formula in
Sect.\ref{sec:exponentially-rise-exponentially-decay}, we have: 
\begin{equation}
g_\NMDA = \bar{g_\NMDA} \times 1.273 \times (e^{-t/\tau_\decay} - 
   e^{-t/\tau_\rise}) \times g_\infty(\Vm, [\Mg]_o)
\end{equation}
with $\tau_\rise = 0.09$ (msec); $\tau_\decay = 1.5$ (msec).

Based on the measurement at synapse of mossy fiber to cerebellar granule cells, 
the synaptic peak conductance is 270 $\mu S$ after activation at the synapse.

\section{Koch-Segev (1998)}
\label{sec:NMDAR_Koch-Segev-1998}

The current via NMDAR is modeled with one gating variable $u$
\begin{equation}
I_\NMDAR = \bar{g}_\NMDAR	B(V_m)u(V_m - E_\NMDAR)
\end{equation}
with $B(V_m)$ is used to model the non-linearity from $\Mg$ block
\begin{equation}
B(V_m) = \frac{1}{1 + \exp\left( -0.062 V_m \right)\frac{[\Mg]_o}{3.57}}
\end{equation}

The gating variable $u$ is in first-order kinetic equation

\begin{equation}
\frac{du}{dt} = \alpha_u (1-u) - \beta_u u
\end{equation}
with $V_p$ is pre-synaptic voltage, and the rate constants ([sec]$^{-1}$)
\begin{equation}
\begin{split}
\alpha_u &= T_\Glu(V_p) \times c_u \\
\beta_u &= 6.6
\end{split}
\end{equation} 
with $c_u = 7.2\times 10^4$ (1/(M.s)) and 
\begin{equation*}
T_\Glu(V_p) = \frac{180}{1+\exp\left(-(V_p-2)/5\right)}
\end{equation*}

\section{Major - \ldots - Tank (2008)}
\label{sec:NMDAR-Major-Tank-2008}

The synapse input is modeled with double exponentials with
\begin{itemize}
  \item $\tau_\on$  = 0.2 msec
  
  \item $\tau_\off$ = 2 msec
\end{itemize}

NMDAR current was modeled with triple-exponential time course; and a sigmoidal
voltage dependency

\def\slope{{{\text{slope}}}}
\def\fast{{{\text{fast}}}}
\begin{equation}
\begin{split}
g_\NMDA &= \bar{g_\NMDA} \left[ -e ^{-t/\tau_\on} + f_\fast e^{-t/\tau_{\off1}}
+ (1-f_\fast) e^{-t/\tau_{\off2}} p
\right]  \\
f_\fast &= f_0 + f_\slope \Vm \\
f_0 &= 0.515 \\
f_\slope &= -0.003125 \qquad \text{1/mV}
\end{split}
\end{equation}

Reversal potential at 0mV; and peak NMDAR conductance
0.875nS.

\subsection{Mg-dependent block}
\label{sec:NMDAR-Mg-block-Major-Tank-2008}


The $\Mg$ block is modeled with forward and backward rate constants based on
Ascher, Nowak (1988) - Sect.\ref{sec:NMDAR-Mg-block-Ascher-Nowak-1988} with 
fixed value for $\Mg$ = 1.8 (mM); the given parameters was adjusted from
20$^\circ$C to 35$^\circ$C using Q10=3, i.e. rates were multiplied by 5.196.

So, the sigmoidal steady-state voltage dependency formula is
\begin{equation}
B(V) = \frac{1}{1+ \exp^{-\frac{V-V_{1/2}}{V_\text{spread}}}}
\end{equation}
with $V_{1/2} = -19.9$ (mV); and $V_\text{spreak}$ = 12.48 (mV)


\section{Shouval (2002)}
\label{sec:NMDAR-Shouval (2002)}

This is from \citep{shouval2002} (Sect.\ref{sec:Shouval-2002}).

NMDAR $\Ca$ current is assumed to have 2 components (fast $I_f$ + slow $I_s$:
$\tau_f = 50$ ms, $\tau_s = 200$ ms) following a single exponential form and
both are $V_m$-dependent as a function of $H(V_m)$ based on Jahr-Stevens
(Sect.\ref{sec:NMDAR_Jahr-Stevens-1990})


\begin{equation}
I_\NMDA(t) = P_o \bar{G}_\NMDA \left[   I_f \theta(t) e^{-t/\tau_f} +   I_s
\theta(t) e^{-t/\tau_s}\right] H(V_m)
\end{equation}
\begin{itemize}
  \item $P_o$ is the fraction of NMDAR from Closed state switching to Open state after
each presynaptic spike (to account for the saturation of NMDAR current. Here,
they chose $P_o$ as a constant value $P_o=0.5$).

  \item $\bar{G}_\NMDA = \frac{-1}{500} [\muM/(\text{ms.mV})]$ (negative sign
  because of influx current) (they also used $\frac{-1}{1350}$ value in one simulation).

  \item $I_f,I_s$ are the relative magnitude of the fast and slow component of
NMDAR $\Ca$ current (the authors assumed $I_f + I_s = 1$). 
The time $t=0$ is the time at which glutamate binding, so when $t<0$ there is no
glutamate binding and thus there is no current. This is expressed via the theta
function $\theta(t)$
\begin{equation}
\theta(t) = \left\{ \begin{array}{lc}
0 & \text{ if } t< 0 \\
1 & \text{ if } t\ge 0 
\end{array} \right.
\end{equation}

Assumption: two components have the same magnitude, only different decay time.

  \item $H(V_m)$ summarizes the $V_m$-dependence as the driving force via
  $(V_m-E_\rev)$ and incorporate the effect of $\Mg$-blocking via $B(V_m)$
\begin{equation}
H(V_m) = B(V_m) \times \left( V_m - E_\rev \right)
\end{equation}
with $E_\rev=130 $(mV) as reversal potential for $\Ca$.
\begin{equation}
B(V_m) = \frac{1}{1+\exp(-0.062 V_m)\frac{[\Mg]}{3.57}}
\end{equation}

NOTE: $[\Mg]$ is assumed to be fixed and set to 1.0.

\end{itemize}

\section{Jadi et al. (2012)}
\label{sec:NMDAR-Jadi-2012}

\begin{equation}
I_{NMDAR} = B(V) \times I_{whatever formula}
\end{equation}


with number of NMDAR is $N_\syn$; and maximum single channel conductance is
$\bar{g}_\NMDA$ with opening probability $p(t)$
\begin{equation}
I_{formula} = \bar{g}_\NMDA \times p(t) \times N_\syn
\end{equation}
with
\begin{equation}
p(t) = \frac{e^{-at} - e^{-bt}}{\left(a/b\right)^{\frac{-a}{a-b}} -
\left(a/b\right)^{\frac{-b}{a-b}}}
\end{equation}
with $a=0.02; b=0.3$.


and $\Mg$ block is (whose suppress most current flow at negative potential) -
based on Major et al. (2008) - Sect.\ref{sec:NMDAR-Major-Tank-2008}
\begin{equation}
B(V) = \frac{1}{1 + \exp^{-(\Vm + 7)/12.5}}
\end{equation}

\section{Rui-Kozloski (2013)}
\label{sec:NMDAR-Rui-Kozloski-2013}

They used the formula from Sect.\ref{sec:NMDAR_Koch-Segev-1998}, yet they
assumed the ions is $\Ca$ based on the experiment of Mayer-Westbook (1987).

\begin{equation}
I_\Ca= \bar{g}_\Ca	B(V_m)u(V_m - E_\Ca)
\end{equation}
with $g_\Ca=\frac{1}{10}\bar{g}_\NMDA$, and 
\begin{equation*}
E_\Ca	= 0.04343 \times T \times \log\left( \frac{[\Ca]_o}{[\Ca]_i}\right)
\end{equation*}
with $[\Ca]_o=750 $mM (keep constant).

They also incorporate the effect of Ketamine on NMDAR by assuming (1)
steady-state, (2) uniform distribution. The effect is modeled simply a linear
change in conductance of the NMDAR
\begin{equation}
\begin{split}
\bar{g'}_\NMDAR &= (1-[\Ketamine]_o) \bar{g}_\NMDAR \\
\bar{g'}_\Ca &= (1-[\Ketamine]_o) \bar{g}_\Ca
\end{split} 
\end{equation}
with $[\Ketamine]_o$ is extracellular Ketamine dose.

