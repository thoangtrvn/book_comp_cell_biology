%%
%% Apdx_params.tex
%% Login : <hoang-trong@hoang-trong-laptop>
%% Started on  Tue Aug 25 23:57:35 2009 Hoang-Trong Minh Tuan
%% $Id$
%% 
%% Copyright (C) 2009 Hoang-Trong Minh Tuan
%%

\chapter{Parameters for models}
\label{chap:parameters-models}



In this chapter, we will discuss some experimental data. Myocytes can be
stimulated using either field stimulation (voltage-clamp 25-30V for 2ms) through
platinum electrodes or by AP (current injection with 2nA for 2.0ms) \citep{hagen2012}.

\section{Volume}

With rat, the estimated average myocyte volume is 36.8pL/cell and cytosolic
volume fraction is 0.65 \citep{bers2001ecc}.

\section{[$\Ca$]}
\label{sec:ceca2+}

\subsection{ER/SR}
\label{sec:ersr}


ER is the major internal $\Ca$ storage in most cells. At the ER
lumen, $\Ca$-ATPases accumulate a large amount of
$\Ca$. With a large amount of such $\Ca$-binding protein at
the lumen, total amount of $\Ca$ in the ER lumen may be $>
1mM$~\citep{foskett2007ip3r}.

The concentration of free $\Ca$ inside the ER has been estimated
between 100 and 700 $\mu M$~\citep{foskett2007ip3r}.

\subsection{Cytoplasm}
\label{sec:cytoplasm}

The concentration of $\Ca$ in the cytoplasm, in unstimulated
cells, is between 50 and 100 $nM$ (3 to 4 orders of magnitude lower
than in ER lumen)~\citep{foskett2007ip3r}. 

Using fluorescence, peak $\Delta F/F0$ during calcium transient (CaT) is
\citep{hagen2012}
\begin{enumerate}
  \item first beat: 9.4$\pm$0.5 ms (Fluo-3), 11.6$\pm$0.8 ms
  (Fluo-4), 12.2$\pm$0.6 ms (Fluo-2)
\end{enumerate} 

\section{Action Potential duration (APD)}

In rat, using different fluorescence, the detected APD defined as the elapsed
time between the 10\% rise and 90\% decay is:
APD90 =38.6$\pm$2.0 ms (with Fluo-4) and 39.2$\pm$ 1.6 ms (with Fluo-3), and
37.0$\pm$1.0 ms (with Fluo-2) \citep{hagen2012}. 

The time for the calcium transient (CaT) to decay from 90\% to 10\% of peak
amplitude (t$_{90-10}$), based on fluorescence, was
\begin{enumerate}
  \item at the first beat: 415$\pm$24 ms (Fluo-3) and
599$\pm$42 ms (Fluo-4), and 777$\pm$40 ms (Fluo-2) 
\item at steady-state: 426$\pm$24 ms (Fluo-3) and
631$\pm$29 ms (Fluo-4), and 682$\pm$40 ms (Fluo-2)
\end{enumerate} 
The slower CaT recovery was explained by an elevation of fluorescence baseline
at higher stimulation rate

\section{Calcium sparks}



%%% Local Variables: 
%%% mode: latex
%%% TeX-master: "mainfile"
%%% End: 
