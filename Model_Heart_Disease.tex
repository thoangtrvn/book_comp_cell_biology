
\chapter{Models of Heart Diseases}
\label{chap:models-heart-dise}

\section{Drug development}

Pharmacokinetics (PK)	 - what the body does to drug

Pharmacodynamics (PD) 	- what the drugs do to the body

Using mathematical/statistical modeling to integrate data from different source
into a model. Nonlinear mixed-effect model (NLME) is used to do the job, from
highly complicated/heterogeneous studied population. 


\subsection{GWAS}
\label{sec:GWAS}

\textcolor{red}{\bf CONTEXT}: Identify genetic risk factors for complex diseases
like schizophrenia (Sect.\ref{sec:schizophrenia}), type II diabetes.

\textcolor{red}{\bf CHOICES}: GWAS can be a good choice to identify such factors
by studying the (multiple) genetic variants across human genome from many
patients and found the common.

Genome-wide association studies (GWAS) method searches the genome for small
variations, called single nucleotide polymorphisms or SNPs (pronounced 'snips'),
that occur more frequently in people with a particular disease than in people
without the disease. Each study can look at hundreds or thousands of SNPs at the
same time. Researchers use data from this type of study to pinpoint genes that
may contribute to a person's risk of developing a certain disease. 
\url{https://ghr.nlm.nih.gov/primer/genomicresearch/gwastudies}

\begin{mdframed}

{\bf OTHER CHOICES}:
There have been many reported associations between traits and copy-number
variants (CNVs) and that there are known mechanisms by which CNVs can be
associated with disease. Results from other genome-wide surveys, including
exome and whole-genome sequencing (WGS) studies.

\end{mdframed}



\textcolor{blue}{\bf Examples of successful uses}:
\begin{enumerate}
  \item identification of the {\it Complement Factor H} gene as a major risk
  factor for age-related macular degeneration (AMD).
  DNA sequence variations in this gene associated with AMD.
  
  \item 
\end{enumerate}







Genome Wide Association Studies: use Logistic Regression based on univariate
model
\begin{equation}
\text{logit}(\pi_i) \sim \beta_0 + \beta_1 x_i
\end{equation}
There are different statistical methodologies to study the model

GWAS is data parallel computation. Packages; GenABEL, PyCUDA, \ldots 

Use SNP marker detectioin: 1.4 million SNPs from Phase III data of HapMap
dataset. 


How about PLINK? A popular open-source GWAS
toolset\footnote{\url{http://research.nesc.ac.uk/node/529}}. 



%%% Local Variables: 
%%% mode: latex
%%% TeX-master: "mainfile"
%%% End: 


