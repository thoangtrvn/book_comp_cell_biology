\chapter{Phenomenological models of STDP}
\label{chap:models-phenomenological_STDP}

Because of strong competition between inputs, the postsynaptic firing rate in
these models: Sect.\ref{sec:Blum-Abbott_1996},
Sect.\ref{sec:Gerstner-1996_neural-learning-rule}, Sect.\ref{sec:Eurich-1999},
Sect.\ref{sec:Kistler-vanHemmen_2000}, is independent of synaptic input rate.
Also, these learning rules are unstable and require hard (upper and lower)
bounds on the synaptic weight. The synaptic weights are driven to these bounds
and obtain a bimodal distribution, which is unlikely to reflect the weight
distribution in biological neurons.



\section{Blum-Abbott (1996)}
\label{sec:Blum-Abbott_1996}


\section{Gerstner (1996) - neural learning rule}
\label{sec:Gerstner-1996_neural-learning-rule}

The model was built based on {\it auditory system of the barn owl} with neurons
in the laminar nucleus. Barn owl uses auditory system to detect prey, i.e.
azimuthal sound with time resolution less than 5$\mus$.
This is achieved based on {\bf phase locking}: {\it spikes occur preferentially
at a certain phase - termed 'mean phase', at frequency up to 8 kHz}.

\section{Eurich et al. (1999)}
\label{sec:Eurich-1999}

\section{Kistler- van Hemmen (2000)}
\label{sec:Kistler-vanHemmen_2000}

\section{van Rossum (2000) - Hebbian learning}
\label{sec:vanRossum-2000}

van Rossum's learning rule claimed to generates a stable, unimodal, positively
skewed distribution of synaptic weights that closely resembles the distribution
of quantal amplitudes measured from central neuron.
\begin{itemize}
  \item the weights depend on their correlation with other inputs, so that learning occurs
through cooperation between inputs

  \item LTP if presynaptic spike occurs before EPSP
  
  \item LTD if presynaptic spike occurs after EPSP
  
  \item The conductance change decreases
  approximately exponentially to the time delay $\delta t$ between pre-synaptic
  and postsynaptic spikes.
   
  \item  competition is almost absent, and must be introduced independently by
  implementing {\bf activity-dependent synaptic scaling}.  The scaling does not
  change the shape or the stability of the weight distribution
\end{itemize}

\subsection{Hypothesis analysis}

Assumption:
\begin{itemize}
  \item symmetric time delay $\tau_{STDP}$
  
  \item independence between pairing events
  
  \item to reproduce fluctuation in the potentiation and depression weights:
  random variable of conductance is added. They assumed fluctuations are
  multiplicative
  
NOTE: The idea that weak synapse has strong
  fluctuation, and strong synapse has weak fluctuation was rejected as it does
  not seem to correspond to the data.
  
  
\end{itemize}


Model data: extracted from experimental data of Bi and Poo (1998)
\citep{bi1998} (Sect.\ref{sec:Bi-Poo_1998})
yet with some changes
\begin{itemize}
  \item In Bi and Poo (1998): the time delay is asymmetric: $34\pm 13$ ms for
  depression and $17\pm 9$ ms for potentiation.
  
  In van Rossum model: it is assumed symmetric with the same time constant
  $\tau_\STDP = 20$ ms.
  
  \item Based on 60 pairing times, the chage at one single pairing was
  calculated by dividing the recorded change for 60 (implicitly assuming
  independent between events)
\end{itemize}

\subsection{Mathematical model}

The synaptic strength (conductance) $w$
\begin{itemize}
  \item potentiation: $w <- w + w_p$
  
  \item depression: $w <- w + w_d$
\end{itemize}

INITIAL FORMULA:
\begin{gather}
w_p = c_p e^{-\delta t/\tau_\STDP} \\
w_d = -c_d e^{\delta t/tau_\STDP}
\end{gather}
with $c_p = 0.003$ (pS), and $c_d=7$ (pS) - the average amount of relative
potentiation and depression, respectively, after one pairing.

As fluctuation in the amount of depression and potentiation is important, i.e. a
noisy quantity. A random variable for conductance value is added to the weights
after every pairing.

{\bf REVISED FORMULA}
\begin{equation}
\begin{split}
\label{eq:vanRossum_formula-2000}
w_p &= (c_p + v. w) e^{-\delta t/\tau_\STDP} \\
w_d &= (-c_d + v.w) e^{\delta t/tau_\STDP}
\end{split}
\end{equation}
with $v$ is a Gaussian random variable with zero mean (SD is extracted from
data $\sigma = 0.015$)


{\bf activity-dependent synaptic scaling}: the precise mechanism behind activitydependent
scaling is not known. It is  a mechanism that adjusts the synaptic
weights to regulate the postsynaptic activity.
Activity-dependent scaling scales the weights to prevent too low or too high
activity levels. The scaling is thought to be multiplicative and independent
of presynaptic activity. Initially, a simple implementation was used
\begin{equation}
\frac{dw(t)}{dt} = \beta w(t) \left[ a_{goal} - a(t) \right]
\label{eq:active-dependent-scaling-simple}
\end{equation}
with $a_{goal}$ is the desired postsynaptic activity (set to at 20 Hz), and 
$\beta$ = a constant determine the strength of the scaling. $a(t)$ is a
slow-varying sensor which increases with every postsynaptic spike and decays 
exponentially between spikes
\begin{equation}
\tau \frac{da(t)}{dt} = -a(t) + \sum_i \delta(t-t_i)
\end{equation}
with $t_i$ are all spike times. The biological constant $\tau$ of the sensor is
unknown, yet it is expected to be slow, so $\tau = 100$ sec was used.

As the equations \ref{eq:vanRossum_formula-2000} pulls the weights, which may
lead to a residual deviation of the weight off the goal value. To avoid this
residual (such errors), an {\bf integral controller} is used to replace
eq.\ref{eq:active-dependent-scaling-simple}
\begin{multline}
\label{eq:active-dependent-scaling-integral-controller}
\frac{dw(t)}{dt} =  \beta w(t) \left[ a_{goal} - a(t) \right] +
    \gamma w(t) \int^t_0 dt' \left[a_{goal} - a(t')\right] 
\end{multline}
with $\gamma$ is another constant


\subsection{Numerical method}

The model was simulated by incorporating into a single cell neuron model with
receiving random inputs. 

The neuron model is {\bf leaky integrate and fire} neuron with 100 M$\Omega$
input resistance, 20 ms time constant, -60 mV resting potential, -50mV firing
threshold. NOTE: They reset the membrane potential back to resting potential
after each firing.

The neuron receives input from 25 inhibitory and 100 excitatory synapses.
\begin{itemize}
  \item inhibitory synapse: reversal potential -70 mV, with time constant 5ms
  
The inhibitory synapses are not plastic but are fixed at a conductance
of 2000 pS and are stimulated with Poisson trains (i.e. a burst of spike
following Poisson distribution) of 20 Hz.

  \item excitatory synapse: reversal potential 0 mV, with time constant 5ms.
  
The excitatory synapses also receive Poisson input.
Then to implement plasticity, a proper formula in equation
\ref{eq:vanRossum_formula-2000} was used, depending on the time synaptic event
before or after the postsynaptic spike at each excitatory synapse.

\end{itemize}

NOTE: To introduce correlation across synaptic input: 
Correlation is implemented by randomly distributing N Poisson trains among the
inputs. Every time step the Poisson trains are redistributed.
NOTE: For every synaptic event there is a chance 1/N, that it is shared by
another synapse, i.e. a cross-correlation coefficient between trains
\begin{equation}
C(\Delta t) = \frac{1}{N}\delta (\Delta t)
\end{equation}




\subsection{Code}

Code for generating Poisson spike trains:
\url{https://praneethnamburi.wordpress.com/2015/02/05/simulating-neural-spike-trains/}


\subsection{Analysis}

As with any feedback system, oscillations readily occur.

\section{Rubin et al. (2000)}

Rubin et al. (2000) has a similar approach to calculate weight distribution like
that of van Rossum (2000) - Sect.\ref{sec:vanRossum-2000}

\section{Pfister - Gerstner (2006)}

Previous STDP models are based on pair-based STDP learning rules, i.e. one
pre- and one post-synaptic spike. Thus, they cannot reproduce recent triplet and
quadruplet experiments.

The paper examine triplet experiments: a rule which considers sets of three
spikes, i.e., two pre and one post or one pre and two post) and compare it to
classical pair-based STDP learning rules.

 

\section{Experimental data}

\subsection{Bi-Poo (1998) - experiments}
\label{sec:Bi-Poo_1998}


STDP protocol: synaptic events and postsynaptic events are paired for 60 times.
