\chapter{Chemical reaction - Transportation}
\label{chap:chem-react-transp}

\def\A{{\text{A}}}
\def\B{{\text{B}}}
\def\C{{\text{C}}}
\def\D{{\text{D}}}

The previous chapter (Chap.\ref{chap:chem-bond-quant}) discussed the mechanism
of atom interactions at a lower level, the bonding energy of covalent and
non-covalent bonding.

In this chapter, we will look at the chemical interaction between molecules
within a single physical compartment/volume. The change in concentration of a
species in this compartment is the result of the reaction occurs that involve
the given species. When multiple compartments, i.e. spatial dimension, is
considered, then the change in concentration of the species is also under
another effect called the diffusion part (Sect.\ref{sec:diffusion}), represented
in the form of flux (Sect.\ref{sec:flux-across-compartments}).

{\bf Thermochemistry} is the field that study how a chemical reaction occurs
(Sect.\ref{sec:thermochemistry}).

% Chemical reaction is what happens at microscopic level.
% at a higher level. 
Important concepts involves the {\bf time scale} of a
process which tells whether a chemical reaction is fast, or slow. This
property is known as the {\bf rate of reaction} and is studied under
the field {\bf reaction kinetics}.

In a chemical reaction, the energy $E$ is the internal energy $U$
(Sect.\ref{sec:gibbs-fund-equat}), the chemical potential of the i-th species in
the system is defined as the partial derivative of the $U$ on the number of particles in species
i-th.
\begin{equation}
  \label{eq:chem_potential}
  \mu_i = \left( \frac{\partial U}{\partial N_i} \right)_{S,V,N_{j \ne i}}
\end{equation}

We have studied the macroscopic and microscopic viewpoint of a thermodynamic
system. At thermodynamic equilibrium, the system assumes the state with maximum
entropy (Sect.\ref{sec:entropy}). Finding such configuration, requires
estimating the partition function $Z$, and the Maxwell-Boltzmann distribution of
velocities of particles in a low density system. In such systems, the effect of
quantum mechanics were neglected.


% In the previous chapter we have learnt the kinetics of reactions, i.e.
% how fast different atoms/molecules interact to form a new species. In
% this chapter, we will study the underlying mechanism of chemical
% bonding. In order to do so, we need to know the structure of an atom
% and new technical concepts.

% \section{Wave function}
% \label{sec:wave-function}

% In quantum mechanics, the dynamics of all subatomic particles are
% described in terms of wave functions. It maps a possible state of the
% system into complex numbers. 
\section{Thermochemistry}
\label{sec:thermochemistry}


\textcolor{red}{Hess's law (Sect.\ref{sec:Hess-law}) and the Lavoisier \&
Laplace law (Sect.\ref{sec:Lavoisier-Laplace-law}) above are the bedrock for
thermochemistry}.

Both laws are preceded {\bf the law of conservation of energy}\footnote{[Wiki]
The conservation of energy stated that the energy is not created or destroyed
itself, it just transform from one form to another.}
(Sect.\ref{sec:conservation-of-energy}) and it can be shown that the two laws
are the direct consequences of it. However, {\it the conservation of energy does
NOT tell us what the various forms of energy are and how energy transforms from
one form to another}. This led to the concept of {\it affinity}
(Sect.\ref{sec:affinity}).

\subsection{-- Lavoisier and Laplace's law: {\it specific heat} and {\it latent
heat}}
\label{sec:Lavoisier-Laplace-law}

In the 18th century (1780) Antoine Lavoisier and Pierre-Simon Laplace laid the
foundation of {\it thermochemistry} - a branch of thermodynamics
(Sect.\ref{sec:thermodynamics-systems}) (See the book Thermodynamics) - which
studies the energy involved or absorbed in chemical reactions (heat into work
and vice verse) - by showing that {\it the same amount of heat must be supplied
to decompose a compound as would be produced on its formation}, i.e. heat
evolved in a reaction is equal to the heat absorbed in the reverse reaction.

They investigated the {\it specific} and {\it latent heat}
(Sect.\ref{sec:latent-heat}) of a number of substances, and amount of heats
involved in combustion.

\begin{itemize}

\item specific heat = heat received required to increase the system one degree
1$^o$ in temperature.
 
\item latent heat = heat receive required in order for a state transition
  to occur (e.g. solid \ce{->} liquid)
\end{itemize}

\subsection{-- Hess's law}
\label{sec:Hess-law}

In 1840 Swiss chemist Germain Hess formulated the principle that the evolution
of heat in a reaction is the same whether the process is accomplished in one
step or in a number of consecutive stages. This is known as {\bf Hess's law}.

Example: the heat of formation of \ce{CO2} is equal to the sum of heat of the
formation of CO and the heat of oxidation of that amount of CO to \ce{CO2}. Hess
employed this principle to determine indirectly the heat of formation of
compounds from their elements, when this magnitude, as is generally the case,
was inaccessible to direct measurement.



\subsection[Berthelot and Thomsen: affinity]{Affinity: driving-force for a
chemical reaction}
\label{sec:affinity}

To explain the microsopic aspect of chemical reaction, a theoretical foundation
is needed.  Ilya Prigogine - {\it ''As motion was explained by Newtonian concept
of force, chemists  want a similar concept of 'driving force' for a chemical
change. Why do chemical reactions occur, and why do they stop at certain points?
Chemists called the 'force' that caused chemical reactions {\bf affinity}, but
it lacked a clear definition."} 

{\bf Affinity} is the next stage to study the direction of energy transform,
after the concept of conservation of energy was formed
(Sect.\ref{sec:conservation-of-energy}).
{\bf Affinity} refers to the tendency for the reaction to occur in one direction
instead of the other\footnote{Affinity is the property of which dissimilar
substances are capable of entering into chemical combination with each other}.
\textcolor{red}{However, there is still no way to quantify "affinity''}.
With that effort, the French chemist Marcellin Berthelot and the Danish chemist
Julius Thomsen had attempted to quantify {\bf affinity} using heats of reaction
(Sect.\ref{sec:principal-of-maximum-work}).

However, this has so many exceptions (especially in reversible reactions) and
then had been abandoned by its authors. As a result, it is now of only
historical importance. \textcolor{red}{Nowadays, ``{\bf free energy}'' is a more
advanced and accurate term to replace the outdated term ``affinity''}
(Sect.\ref{sec:free-energy-direction-reaction}).
The concept of ``free-energy'' was originally derived from thermodynamics.

\subsection{-- principal of maximum work (obsolete)}
\label{sec:principal-of-maximum-work}

Berthelot proposed the {\it principal of maximum work} (in thermochemistry),
i.e. in a pure chemical reaction (a chemical changes occurring without the
intervention of outside energy), the reaction tends to occur in the direction
toward the production of bodies or a system of bodies that liberate heat.

Based on these and other ideas, Berthelot and Thomsen, as well as others,
considered the {\it heat given out in the formation of a compound as a measure
of the affinity}. 



\subsection{Affinity constants: binding strength}
\label{sec:affinity-constants}

{\bf Affinity constants} are numeric values that tells the strength
with which two molecules interact. Given an affinity constant, we can
know an interaction is ``tight binding'' or ``weak binding''. 

\subsection{-- binding constant, dissociation constant}
\label{sec:binding-constant}
\label{sec:dissociation-constant}

\begin{itemize}
  \item To tell the tendency in binding of two molecules, we use
{\bf binding constant} (association constant) $K_B$. 

  \item To tell the tendency in dissociation of a complex into two separate
  molecules, we use {\bf dissociation constant} $K_D$. $K_D$ is inversely
  related to affinity while $K_B$ is directly related to affinity.
\end{itemize}

So
\begin{equation}
  \label{eq:332}
  K_D = \frac{1}{K_B}
\end{equation}
The unit of $K_D$ is molar (M, moles/(litres-solution)), and the unit of $K_B$
is $M^{-1}$.  \textcolor{red}{For convenience, typically, $K_D$ is more widely
used}.

In a reaction with two {\bf rate constants} $k^+_{1}$ (binding) and
$k^-_{1}$ (dissociation)
\begin{equation}
  \label{eq:333}
  \ce{A + B <=>[k^+_1][k^-_{1}] AB}
\end{equation}
then 
\begin{equation}
  \label{eq:334}
  K_D = \frac{k^-_1}{k^+_{1}}
\end{equation}
Correspondingly, the unit of $k^-_{1}$ is $\text{time}^{-1}$ while
$k^+_{1}$ is Molar$^{-1}\text{time}^{-1}$.

\begin{framed}
  
  High affinity means large $K_B$, which in turns equivalent to (high
  $k^+_1$ and low $k^-_{1}$). This means that the complex formation occur
  rapidly, while the complex dissociate slowly.
\end{framed}


\subsection{-- estimate dissociation constant: $K_D$}
\label{sec:determining-k_d}

{\bf IMPORTANT}: We don't have to know the forward and backward rates
$k^+_1$, $k^-_{1}$ in order to compute $K_B$ or $K_D$. Instead, we can
compute them directly at the equilibrium state of the reaction.  This
is because of the fact that, at equilibrium, the concentration of each
reactants and the complex (product) follow this equation
\begin{equation}
  \label{eq:335}
  K_D = \frac{[A]_{eq}[B]_{eq}}{[AB]_{eq}}
\end{equation}
Based on the law of total mass conservation, $[A]_{total}=[A]+[AB]=a_t$,
then 
\begin{equation}
  \label{eq:336}
  \begin{split}
    K_D[AB]_{eq} &= (a_t-[AB]_{eq})[B]_{eq} \\
 \rightarrow\;\;   \frac{[AB]}{a_t} &= \frac{[B]}{[B]+K_D}
  \end{split}
\end{equation}

The {\bf fraction bound} $[AB]/a_t$ is the fraction of A involves into the
reaction at equilibrium. The upper bound is 1 (used all A) when [B] is
at a great amount, i.e. at infinity. 

\begin{framed}
  The idea behind using $K_D$ is that when [B] is equal to $K_D$, the
  fraction bound is 1/2.


  \begin{equation}
    \label{eq:337}
    \frac{[AB]}{a_t} = \frac{\frac{[B]}{K_D}}{\frac{[B]}{K_D}+1}  
  \end{equation}
\end{framed}

To determine $K_D$, the concentration of A, $a_t$ is held constant;
while they vary [B]. For each value of [B], they obtained a value of
[AB] at equilibrium. The curve plotting [B] vs. $[AB]/a_t$ is a hyperbola. 
Then, $K_D$ is the tangent coefficient of the line $1/[B]$ vs. $a_t/[AB]$
\begin{equation}
  \label{eq:338}
   \frac{a_t}{[AB]}= K_D\frac{1}{[B]} + 1
\end{equation}

{\bf EXPERIMENT TIPS}: To quickly measure $K_D$, it is often use a
small concentration of A compared to $K_D$ ($a_t\ll K_D$) - ideally
100-fold below the $K_D$. With this condition, at equilibrium, [AB] is
thus pretty much relatively small compared to $K_D$.  In addition,
$[B]_{total}$ covers a range from 10-fold below to 10-fold above
$K_D$.  Knowing that $[B]_{total}=b_t=[B]_{free}+[AB]$, then at
equilibrium, $[AB]$ is rather small compared to $[B]$ free. In the
end, we don't have to measure [B] free at equilbrium, but we can
assume [B] free to be equal $[B]_{total}$. Otherwise, we have to
measure [B] free at equilibrium every time we change the concentration
of $[B]_{total}$. 

Theoretically, the fraction bound approach 1 when [B] total go to
infinity. However, this is not true. At saturation, the fraction bound
approach a asymptotic line at $y=f_{max}$. Thus, we have to modify
eq.~\eqref{eq:336}
\begin{equation}
  \label{eq:339}
   \frac{[AB]}{a_t} = f_{max}\frac{[B]}{[B]+K_D}
\end{equation}
An equivalent form that is more commonly used is
\begin{equation}
  \label{eq:340}
    X_{AB}= {X_{AB}}_{max}\frac{[B]}{[B]+K_D}
\end{equation}
Here, $X_{AB}$ is not a concentration, instead it can be any thing
that denote the amount of AB complex, and ${X_{AB}}_{max}$ is the maximum
amount of $X_{AB}$ at equilibrium. 

\section{Direction of reactions}
%\section{Factors to determine the reaction}
\label{sec:fact-determ-react}
\label{sec:free-energy-direction-reaction}

Early efforts to quantify the direction of a (chemical) reaction used {\it
affinity} (Sect.\ref{sec:affinity}). Nowadays, to study the direction of the
reaction, {\bf free energy} is the modern measure to use, compared to the
obsolete one {\it affinity}.

However, in between, different concepts had also been used to study the
direction of a reaction
\begin{enumerate}
  \item entropy - Sect.\ref{sec:entropy}: reaction occurs in the direction of
  increasing entropy
  \item enthalpy - Sect.\ref{sec:enthalpy}
\end{enumerate}



\section{Speed of reactions}
\label{sec:speed-reaction}

