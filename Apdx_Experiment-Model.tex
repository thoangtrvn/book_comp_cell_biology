\chapter{Experiment data to building models}


\section{From dose-response data to kinetics model}
\label{chap:from-dose-response}

The dose-response data is the output of an interactive molecules,
interact with its ligands, and its internal regulation. When giving
the drugs to an organism, it's in a non-steady-state environment where
everything is dependent upon time, i.e. time is the only independent
variable. However, under lab condition, a great deal of information
can also be obtained where experiments occur at equilibrium,
i.e. ligand concentration is the sole independent variables. 


The various concept of ``ligand-receptor'' was developed during the
``classic era ending'' circa 1945 and post classic era, starting circa
1965. Typically, the receptor is modeled as {\it two-state mechanism}
with one state at free receptor and the second state at ligand-bound. 

Later, with the concept of conformational changes, there can be two or
more conformations for a free receptor unit. From that, the concept of
``efficacy'' was derived. This requires a clear distinction between
the two concepts below. Before modeling the reaction, it's important
to tell whether the dose-response data is
\begin{itemize}
\item binding response
\item functional response
\end{itemize}
of a ligand against the change in concentration of a ligand. This is
based on the fact that a conformational change when a ligand bind,
is not necessarily the one which activates the receptive unit for
function. In this case, we call it binding response; but not
functional response. 

In the functional response, when binding affect activation; then
activation must affect the binding process. This mutual interaction is
known as {\it reciprocity}.


%%% Local Variables: 
%%% mode: latex
%%% TeX-master: "mainfile"
%%% End: 
