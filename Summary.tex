
\chapter{Summary of landmarks}
\label{chap:summary-landmarks}

In 1950s, Hodgkin-Huxley was the first phenomenological model of excitable cell,
the axon of squid giant cell~\citep{hodgkin1990qdm}. This model try to replicate
the periodic depolarization of membrane potential - called the {\bf action
potential} (AP).  Different cell types, or cells at different regions, may have
different AP waveform.  Its shape depends upon the characteristics and types of
ionic fluxes across the membrane.  So, before the 90s, experimentalists try to
discover all types of ion channels and its kinetics that may contribute to the
AP of a cell. 

The kinetics of the ionic channels, and ionic pumps/exchangers (not only in this
stage, but later on) can be referenced in Chap. \ref{chap:voltage-gated-ionic}
(focusing on HH-based formula), Chap.\ref{chap:Na_models} ($\Na$),
Chap.\ref{chap:K_model} ($\K$), Chap.~\ref{chap:dhpr-models} (DHPR),
Chap.~\ref{chap:models-pumps} (NCX, NA/K, PMCA), Chap.~\ref{chap:ryr-models}
(RyR), Chap.~\ref{chap:ip3r-models} (IP3R). In whole-cell modelling, there are
two classes of model: (1) reduced one, i.e. use the minimum ionic currents that
still can reproduce the major properties of a cell; (2) detailed one, i.e.  try
to incorporate the kinetics of all possible different ion channels.


In cardiac cells, researchers realized that \ce{Ca^2+} play a major role to the
contractile function of the cells. However, the influx of calcium is not enough
to trigger the contraction~\citep{bassingthwaighte1972cme}. In 1975,
Fabiato-Fabiato confirmed the hypothesis that ``additional \ce{Ca^2+} is release
from internal store''~\citep{fabiato1975cic}, and there is only one calcium
internal storage - the sarcoplasmic reticulum (SR). The mechanism, known as {\bf
calcium-induced calcium release} (CICR), is described as follows: at the initial
phase of AP (mostly caused by the activation of $\Na$ channels), the
depolarization of membrane potential trigger an influx of \ce{Ca^2+} via the LCC
will cause a local elevation of calcium concentration; these calcium bind to the
RyR in proximity and activate the RyRs, triggering the release of \ce{Ca^2+}
from SR~\citep{fabiato1989apr}. \textcolor{red}{``It is the trans-sarcolemmal
calcium flux, rather than
  the depolarization of the sarcolemma, per se, that triggers SR calcium
  release''}
(Stein, 1992).  Unlike AP as an all-or-none process, CICR has two important
properties: {\it high-gain} (amplification), i.e. the smaller influx of
\ce{Ca^2+} triggers the much higher \ce{Ca^2+} release, and {\it gradeness},
i.e. the calcium release is proportional to the graded increase of calcium
influx. However, it was not until 1998 that computational models incorporate
CICR explicitly~\citep{jafri1998cad}.

Early models, involving those which incorporate CICR mechanism, are
``common-pool models''. The essential feature of a 'common pool' model
of E-C coupling is that the entering \ce{Ca^2+} and released
\ce{Ca^2+} occupy a common cytosolic space (i.e. there is no spatial
localization of calcium). \citep{stern1992tec} proved that
such models cannot reproduce ``gradeness'' property. 
 

In 1993, Cheng-Lederer discovered that the graded increase is due to
the release of \ce{Ca^2+} at different local region which, under
normal condition, cannot trigger calcium wave.  So, the calcium
elevation in the whole-cell is the summation of all these local
calcium transience.  The elementary events of \ce{Ca^2} elevations in
cardiac cells is called {\bf \ce{Ca^2+} spark}~\citep{cheng1993cse}.
The \ce{Ca^2+} spark allows us to explain the important property:
gradeness, i.e. the release of \ce{Ca^2+} from junctional SR (JSR) is
not all-or-none but in gradeness, which can occur while \ce{Ca^2+}
wave cannot occur. Experimental studies also confirmed that frequency
of the sparks, per se, cannot trigger calcium wave, but the SR \ce{Ca^2+}
- or {\bf luminal calcium} - overload can trigger calcium wave as it
modify the sensitivities of RyRs to \ce{Ca^2+}. As a result, many
models, though can reproduce many frequency-dependent phenomenon, need
to be revised.

It's important to incorporate a large number of calcium release site/unit (CRUs)
into the model. Experiments show this range from 10,000 to 20,000. Since then,
there are several models to replicate the high-gain and gradeness properties of
\ce{Ca^2+} signalling, with the number of CRUs vary from a few hundreds to a
realistic amount (20,000).

Given that CICR is a regenerative process, it turns into the question how the
local \ce{Ca^2+} release from SR can be terminated. There are different
hypothesis, yet the widely accepted until nowadays is the negative feedback of
luminal calciumm depletion. There is a bunch of models try to resolved this by
proposing mechanism for terminating \ce{Ca^2+} sparks.

Since 2000, there are demands to developed new, detailed models that
take into the high amount of calcium release units (CRUs).  Most of
the models are built at physiological conditions, e.g. normal heart
rate. 

Two major fluxes, $I_\Ca$, $I_\NaCa$ are modulated by intracellular
system. Thus, a calcium system is driven by AP waveform which in turns
depend on the dynamics of the $[\ce{Ca^2+}]_i$ itself. So, it's important to
build the model that can reproduce the results at different AP
waveform. This is the {\bf bidirectional coupling} between the $V_m$ and the
calcium.

\section{Apoptosis}

Components of the previously mysterious process were being discovered almost
weekly, frequent scientific meetings had little overlap in their contents, and
it seemed that every issue of Cell, Nature, or Science had to have at least one
paper on apoptosis.



%%% Local Variables: 
%%% mode: latex
%%% TeX-master: "mainfile"
%%% End: 
