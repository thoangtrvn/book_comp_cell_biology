\chapter{Endoplasmic Reticulum (ER)}
\label{chap:ER}

The endoplasmic reticulum (ER) is an extensive intracellular membrane system.
It is important for a number of cellular functions including translocation of
secretory proteins across the membrane (from ER to plasma membrane), insertion
of membrane proteins, lipid synthesis, calcium storage and signaling, and
separation of nucleoplasm from cytoplasm.

\section{Introduction}

ER was first discovered in 1945 by K.R. Porter. He first called ER as "a
lace-like reticulum" \citep{porter1945}. ER contains sacs and tubules that
extend across the cell; connect to the nuclear envelope; and form a dynamic,
constantly reorganizing, continuous membrane that separates the ER lumen from
the cytosol.


Structurally the ER is a network that is present throughout the cytoplasm, and
consists of
\begin{enumerate}
  \item cisternae
  
  \item flattened sheets
  
  \item 60-100 nm diameter tubules that form  
  irregular polygons with a common luminal space connected via three-way
  junctions
\end{enumerate}

The shape of the ER is heterogeneous,and varies between cell types and
cell stages, but can be divided into three domains
\begin{enumerate}
  \item smooth ER (ribosome-free ER): predominantly composed of tubules

smooth ER is characterized by a much lower density of ribosomes and extends more
uniformly across the cytosolic space.
  
  \item rough ER (ribosome-bound ER: primarily composed of sheets and cisternae
  
  The rough microscopic appearance reflects the presence of membraneassociated
  ribosomes.

  SER and RER were initially identified by electron microscopy; the RER is
  decorated with ribosomes, whereas the SER is not.

  \item NE (nuclear envelope) is a different ER domain; it is a double-membrane
structure in which the outer membrane is connected to the peripheral ER and the
inner membrane is connected to the outer at the nuclear pore. In animal cells it
is distinguished from the rest of the ER by nuclear pores and a set of membrane
proteins enriched in the inner NE.

Most of the rough ER is localized around the nucleus as a continuation of the external
nuclear membrane. It is the primary place where membrane and secreted proteins are
synthesized.
\end{enumerate}

ER also forms the junction with the plasma membrane
(Sect.\ref{sec:junction-plasmamembrane-ER}) which is known to play an important
role in regulating Ca2+ signaling.

\section{Regulators}

Several recently identified proteins are known to regulate
the structure and stability of the ER.

{\bf Reticulons} and {\bf DP1} are two families of ubiquitous and
structurally related eukaryotic proteins associated with
ER membranes, and these are responsible for maintaining
the tubular shape of the ER.

{\bf Atlastin-1}, a dynamin-like GTPase, interacts with reticulon proteins to
promote fusion and the formation of the tubular network.

{\bf CLIMP-63} is a microtubule-binding protein that regulates
the abundance of interaction sites between the ER and the
microtubule cytoskeleton, effectively stabilizing the network.

Fusion of membrane tubules also requires NSF/ $\alpha$, $\gamma$-SNAP, the
p97/p47/VCIP135 complex, syntaxin 18, and BNIP1/sec20.

Other candidates for the maintenance of ER structure include huntingtin, the
EF-hand Ca2+-binding protein p22, spastin, and kinectin.


\section{Nuclear envelope (NE)}
\label{sec:nuclear-envelope}



\section{Rough endoplasmic reticulum}
\label{sec:rough-ER}

RER has ribosome.
RER must be present in all cells, as the RER forms the lipids and proteins that
make up cell membranes, thanks to the presence of ribosome.

To study the distribution of RER in neurons:
\begin{enumerate}
  \item  In invertebrate neurons, e.g. C. elegans, ribosomes were abundant in
  the cell body and rare in neurites, similar to RER membrane protein
  localization; while general ER markers were present throughout the cell body
  and neurites \citep{rolls2002}.  

RER membrane proteins diffuse rapidly, and their mobility is more comparable to
that of general ER membrane proteins than NE membrane proteins.

NOTE: C. elegans neurons generally project only one or two unbranched neurites,
which are often both pre- and postsynaptic, and so are very different from these
mammalian neurons.

  \item  
\end{enumerate}


\section{Smooth endoplasmic reticulum (sarcoplasmic reticulum)}
\label{sec:smooth-ER}
\label{sec:sarcoplasmic-reticulum}

SER doesn't have ribosome.
SER aids in the formation of plasma membranes through production of lipids
(cholesterol and triglycerides). SER functions in steroid hormone synthesis in
endocrine cells. In hepatocytes, SER aids in detoxification pathway.

SER is prominent in certain cell types, such as liver, steroid-synthesizing
cells, muscle, and neurons.

To study the distribution of SER in neurons:
\begin{enumerate}
  \item In invertebrate neurons, e.g. C. elegans,
  general ER markers were present throughout the cell body
  and neurites \citep{rolls2002}. 
  
NOTE: C. elegans neurons generally project only one or two unbranched neurites,
which are often both pre- and postsynaptic, and so are very different from these
mammalian neurons.

 \citep{ramirez2000}  
\end{enumerate}
